\chapter{Introducción}
\label{cap:introduccion}

\chapterquote{Los jóvenes deben enseñarse a sí mismos, entrenarse a sí mismos, con infinita paciencia, intentarlo una y otra y otra vez hasta que salga bien}{William Faulkner}


\section{Motivación}

La \textbf{Inteligencia Artificial} es una de las ramas de la computación que más interés ha generado, tanto entre expertos de la materia como en otro tipo de público, más interesado en la parte lúdica de este concepto. Dotar a una máquina de la capacidad de realizar funciones asociadas sólo al intelecto humano ha sido siempre considerado ciencia ficción. Y sin embargo esta misma idea ha supuesto un enorme avance tecnológico en los últimos años.

Para conocer mejor este concepto desde sus orígenes, es necesario hacer referencia al \textbf{Test de Turing}, propuesto por Alan \citet{Turing1950-TURCMA}. Éste consiste en realizar una serie de preguntas a un ente y el test se considerará superado si el interrogador no es capaz de discernir si las respuestas provienen de una máquina o una persona.

Intentar definir la Inteligencia Artificial nos lleva directamente a conceptos como el \textit{proceso del pensamiento} o el \textit{razonamiento}, los cuales terminan por conducir a otros más complejos como es el \textit{comportamiento}. A partir de estas ideas podemos encontrar otras definiciones, clasificadas en la matriz~\ref{fig:tabla_IA}.

\figura{Bitmap/Introduccion/tabla_IA}{width=1\textwidth}{fig:tabla_IA}%
       {Definiciones de Inteligencia Artificial, \citet{Russell:2009:AIM:1671238}}

En base a estas clasificaciones podríamos diferenciar dos corrientes de interpretación:
\begin{itemize}
    \item Una visión empírica (columna izquierda) con el ser humano como centro de la investigación. Involucra principalmente observaciones y hipótesis sobre cómo debería comportarse un humano. Si nos paramos a pensarlo, esta vertiente tiene cierta relevancia a día de hoy, sobre todo gracias al auge de proyectos como el coche autónomo. Estos agentes autónomos, puestos en una situación límite, podrían verse obligados a decidir entre dos malas opciones. ¿Qué haría un conductor real en esa situación?
    \item Una visión racionalista (columna derecha) que involucra una combinación de matemáticas e ingeniería. En esta vertiente se engloban proyectos como los asistentes de voz, los cuales no necesitan valorar las órdenes o información que les das a un nivel ``humano'', no necesitan ser asertivos, sólo necesitan ser capaces de procesar la información de forma correcta para actuar consecuentemente. También irían aquí los robots de Boston Dynamics, los cuales deben ser capaces de aprender a adaptarse a cualquier terreno con el fin de realizar su función de forma correcta.
\end{itemize}

Multitud de expertos han abordado ambos acercamientos de distintas formas. Nosotros buscaremos que nuestro \textit{agente} tome las ``decisiones correctas'' en función del conocimiento que posea. Para llevar a cabo esto, nos especializaremos en un campo de la Inteligencia Artificial llamado Aprendizaje por Refuerzo, así como una combinación del mismo con Redes Neuronales, denominado Aprendizaje por Refuerzo Profundo.


\section{Objetivos}

\begin{enumerate}
    \item Comprender en qué es el Aprendizaje por Refuerzo distinto a otros métodos de Aprendizaje Automático, y en qué situaciones puede ser usado.
    \item Estudiar los fundamentos del Aprendizaje por Refuerzo, entendiendo sus características, componentes y limitaciones.
    \item Poner a prueba lo aprendido con simulaciones prácticas, en las que implementemos algoritmos de Aprendizaje por Refuerzo y estudiemos sus resultados.
    \item Adentrarnos en el campo del Aprendizaje Profundo, donde veremos los fundamentos de las Redes Neuronales.
    \item Estudiar cómo es posible combinar las Redes Neuronales y el Aprendizaje por Refuerzo con el objetivo de sortear las limitaciones de ambos. Será un camino en el que, paso a paso, encontraremos soluciones a los problemas que surjan hasta lograr un modelo estable de Aprendizaje por Refuerzo Profundo.
    \item En cada hito del camino evaluaremos los resultados obtenidos, para comprobar que las soluciones mejoran en rendimiento y estabilidad.
\end{enumerate}


\section{Estructura de la memoria}

Nuestro trabajo intercala capítulos teóricos con aplicaciones prácticas de lo visto en esos capítulos.

\begin{itemize}
    \item \textbf{Capítulo 1, Introducción.} Motivación y objetivos de nuestro proyecto.
    \item \textbf{Capítulo 2, Aprendizaje por Refuerzo.} El capítulo comienza con comparación del Aprendizaje por Refuerzo con otros métodos de Aprendizaje Automático. Después, se proporciona toda la base teórica necesaria para comprender el Aprendizaje por Refuerzo. Para terminar, explicamos en profundidad Q-Learning, el algoritmo que utilizaremos en nuestras pruebas.
    \item \textbf{Capítulo 3, Q-Learning en acción.} Introduciremos OpenAI Gym, la herramienta utilizada durante nuestras pruebas. Después aplicaremos los conocimientos de Aprendizaje por Refuerzo en CartPole, un problema que nos permitirá evaluar resultados y limitaciones.
    \item \textbf{Capítulo 4, Redes Neuronales y Q-Learning.} Proporcionaremos el marco teórico necesario para comprender las Redes Neuronales. Después explicaremos cómo es posible aplicarlas a técnicas de Aprendizaje por Refuerzo, obteniendo las DQN.
    \item \textbf{Capítulo 5, DQNs en acción.} Volveremos a resolver CartPole y nos enfrentaremos a MountainCar, un problema que presenta nuevos retos para nuestro agente.
    \item \textbf{Capítulo 6, Conclusiones.} Síntesis de todo lo aprendido durante el camino. Puntos clave de las distintas etapas, limitaciones y oportunidades de cara al futuro.
    
\end{itemize}
