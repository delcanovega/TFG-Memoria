\chapter{Introducción}
\label{cap:introduccion}

\chapterquote{Frase célebre dicha por alguien inteligente}{Autor}

% \section{Motivación}
% Introducción al tema del TFM.

\section{Contexto}
La \textbf{Inteligencia Artificial} es una de las ramas de la computación que más interés ha generado, tanto entre expertos de la materia como en otro tipo de público, más interesado en la parte lúdica de este concepto. Dotar a una máquina de la capacidad de realizar funciones asociadas sólo al intelecto humano ha sido siempre considerado ciencia ficción. Y sin embargo esta misma idea ha supuesto un enorme avance tecnológico en los últimos años.

Para conocer mejor este concepto desde sus orígenes, es necesario hacer referencia al \textbf{Test de Turing}, propuesto por Alan \citet{Turing1950-TURCMA}. Éste consiste en realizar una serie de preguntas a un ente y el test se considerará superado si el interrogador no es capaz de discernir si las respuestas provienen de una máquina o una persona.

Intentar definir la Inteligencia Artificial nos lleva directamente a conceptos como el \textit{proceso del pensamiento} o el \textit{razonamiento}, los cuales terminan por conducir a otros más complejos como es el \textit{comportamiento}. A partir de estas ideas podemos encontrar otras definiciones, clasificadas en la matriz~\ref{fig:tabla_IA}.

\figura{Bitmap/Capitulo2/tabla_IA}{width=1\textwidth}{fig:tabla_IA}%
       {Definiciones de Inteligencia Artificial, \citet{Russell:2009:AIM:1671238}}

En base a estas clasificaciones podríamos diferenciar dos corrientes de interpretación:
\begin{itemize}
    \item Una visión empírica (columna izquierda) con el ser humano como centro de la investigación.
    \item Una visión racionalista (columna derecha) que involucra una combinación de matemáticas e ingeniería.
\end{itemize}

Multitud de expertos han abordado ambos acercamientos de distintas formas. Nosotros nos centraremos en la visión empírica, buscando que nuestro \textit{agente} tome las ``decisiones correctas'' en función del conocimiento que posea.

\subsection{Aprendiendo a aprender}
Una de las funciones humanas de las que necesitaremos dotar a nuestra máquina en busca de este \textit{rendimiento ideal} es el \textbf{aprendizaje}.

El campo del \textbf{Aprendizaje Automático} (o \textit{Machine Learning}) se encarga de esta tarea, a través de la generalización y la búsquedas de patrones en experiencias pasadas. En función del tipo de \textit{realimentación} obtenido existen diferentes técnicas de aprendizaje, diseñadas para distintos casos y objetivos:
\begin{itemize}
    \item \textbf{Aprendizaje Supervisado} (\textit{Supervised Learning}): El agente observa una serie de ejemplos de entradas y salidas, aprendiendo una función que es capaz de asignar a una entrada su salida correspondiente. Se llama \textbf{supervisado} porque esta serie de ejemplos, llamado conjunto de entrenamiento, deben estar correctamente clasificada desde un primer momento (es decir, supervisado por un humano). Podría decirse que el agente aprende en base a estas experiencias, hasta que llegado un punto es capaz de clasificar una entrada completamente nueva. La exactitud de esta clasificación dependerá del entrenamiento recibido; más adelante en el proyecto haremos uso de esta técnica de aprendizaje.
    \item \textbf{Aprendizaje no Supervisado} (\textit{Unsupervised Learning}): A diferencia del supervisado, en el aprendizaje no supervisado los ejemplos no cuentan con una etiqueta que los clasifica inequívocamente. En su lugar el agente busca patrones en el conjunto de entrada, intentando extraer características comunes de sus elementos. Uno de los usos más comunes del Aprendizaje No Supervisado es la agrupación o \textbf{clustering}: La unión de ejemplos como grupos o \textit{clústeres} que comparten características comunes. Por ejemplo, la agrupación de películas con características similares, de modo que si a un usuario le resultan interesantes varias de un mismo \textit{clúster}, es muy probable que también disfrute de otros elementos de esa misma agrupación. Estas ideas se usan, por ejemplo, en algunos sistemas de recomendación.   
    \item \textbf{Aprendizaje por Refuerzo} (\textit{Reinforcement Learning}): El agente aprende a partir de una serie de refuerzos: Recompensas si son positivos y penalizaciones en caso de ser negativos. Por ejemplo, cuando una mascota cumple una orden y recibe una galleta como recompensa; o un músico desafina en directo y recibe un abucheo por parte del público. Además el agente puede "pensar" a largo plazo, de forma que su comportamiento le permita conseguir una recompensa mayor en un futuro, así como adaptarse a entornos totalmente nuevos para él. Esta técnica de aprendizaje será el punto de partida de nuestro proyecto y explicaremos su funcionamiento a lo largo de este capítulo. 
\end{itemize}
