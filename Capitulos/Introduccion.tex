\chapter{Introducción}
\label{cap:introduccion}

\chapterquote{Los jóvenes deben enseñarse a sí mismos, entrenarse a sí mismos, con infinita paciencia, intentarlo una y otra y otra vez hasta que salga bien}{William Faulkner}


\section{Motivación}

La \textbf{inteligencia artificial} es una de las ramas de la computación que más interés ha generado, tanto entre expertos de la materia como en otro tipo de público, más interesado en la parte lúdica de este concepto. Dotar a una máquina de la capacidad de realizar funciones asociadas sólo al intelecto humano ha sido siempre considerado ciencia ficción. Y sin embargo esta misma idea ha supuesto un enorme avance tecnológico en los últimos años.

Para conocer mejor este concepto desde sus orígenes, es necesario hacer referencia al \textbf{test de Turing}, propuesto por Alan \citet{Turing1950-TURCMA}. Éste consiste en realizar una serie de preguntas a un ente y el test se considerará superado si el interrogador no es capaz de discernir si las respuestas provienen de una máquina o una persona.

Intentar definir la inteligencia artificial nos lleva directamente a conceptos como el \textit{proceso del pensamiento} o el \textit{razonamiento}, los cuales terminan por conducir a otros más complejos como es el \textit{comportamiento}. A partir de estas ideas podemos encontrar otras definiciones, clasificadas en la matriz~\ref{fig:tabla_IA}.

\figura{Bitmap/Introduccion/tabla_IA}{width=1\textwidth}{fig:tabla_IA}%
       {Definiciones de inteligencia artificial, \citet{Russell:2009:AIM:1671238}}

En base a estas clasificaciones podríamos diferenciar dos corrientes de interpretación:
\begin{itemize}
    \item Una visión empírica (columna izquierda) con el ser humano como centro de la investigación. Involucra principalmente observaciones e hipótesis sobre cómo debería comportarse un humano. Esta vertiente tiene cierta relevancia a día de hoy, sobre todo gracias al auge de proyectos como el coche autónomo. Estos agentes autónomos, puestos en una situación límite, podrían verse obligados a decidir entre dos opciones que pongan en peligro vidas humanas. En este caso, deberíamos basar nuestra respuesta en qué elegiría un conductor real.
    \item Una visión racionalista (columna derecha), que implica una combinación de matemáticas e ingeniería. En esta vertiente se engloban proyectos como los asistentes de voz o los robots de Boston Dynamics. Ninguno de ellos necesita valorar las órdenes e información que se les provee a un nivel ``humano''. Los asistentes no necesitan ser asertivos, sólo necesitan ser capaces de procesar la información de forma correcta para actuar consecuentemente, mientras que los robots sólo deben ser capaces de aprender a adaptarse a cualquier terreno con el fin de realizar la función que se les ha encomendado.
\end{itemize}

Multitud de expertos han abordado ambos acercamientos de distintas formas. Nosotros buscaremos que nuestro \textit{agente} tome las ``decisiones correctas'' en función del conocimiento que posea. En particular buscaremos que un agente sin ningún conocimiento previo sea capaz de aprender a realizar tareas sencillas mediante la interacción constante con el entorno, quien le proporcionará nuevas experiencias de las que extraer conocimiento.

Esta forma de aprendizaje es un campo de la inteligencia artificial llamado \textbf{aprendizaje por refuerzo}. También investigaremos una combinación del mismo con \textbf{redes neuronales}, resultando en el llamado \textbf{aprendizaje por refuerzo profundo}. La popularidad de estas técnicas no ha parado de crecer en los últimos años. Koray Kavukcuoglu, director de investigación en Deepmind, explica su potencial de la siguiente forma:

\begin{quote}
    El aprendizaje por refuerzo es un sistema muy general para aprender a tomar decisiones secuenciales. Por otra parte, el aprendizaje profundo es el mejor conjunto de algoritmos disponibles para aprender representaciones. Combinar estos dos modelos diferentes es la mejor opción que tenemos disponible para lograr buenas representaciones de estados en tareas complejas, no sólo para resolver juegos sencillos si no también complicados problemas reales.
\end{quote}

En definitiva, el potencial y las posibilidades de esta metodología convierten al aprendizaje por refuerzo profundo en un campo muy interesante, que puede que nos lleve un paso más cerca al mundo de la inteligencia artificial general.


\section{Objetivos}

\begin{enumerate}
    \item Comprender en qué es el aprendizaje por refuerzo distinto a otros métodos de aprendizaje automático, y en qué situaciones puede ser usado.
    \item Estudiar los fundamentos del aprendizaje por refuerzo, entendiendo sus características, componentes y limitaciones.
    \item Poner a prueba lo aprendido con simulaciones prácticas, en las que implementemos algoritmos de aprendizaje por refuerzo y estudiemos sus resultados.
    \item Adentrarnos en el campo del aprendizaje profundo, donde veremos los fundamentos de las redes neuronales.
    \item Estudiar cómo es posible combinar las redes neuronales y el aprendizaje por refuerzo con el objetivo de sortear las limitaciones de ambos. Será un camino en el que, paso a paso, encontraremos soluciones a los problemas que surjan hasta lograr un modelo estable de aprendizaje por refuerzo profundo.
    \item En cada hito del camino evaluaremos los resultados obtenidos, para comprobar que las soluciones mejoran en rendimiento y estabilidad.
\end{enumerate}


\section{Estructura de la memoria}

Nuestro trabajo intercala capítulos teóricos con aplicaciones prácticas de lo visto en dichos capítulos, resultando en dos bloques diferenciables: el primero sobre aprendizaje por refuerzo y el segundo sobre aprendizaje por refuerzo profundo.

\begin{itemize}
    \item \textbf{Capítulo 1, Introducción.} Motivación y objetivos de nuestro proyecto.
    \item \textbf{Capítulo 2, Aprendizaje por Refuerzo.} El capítulo comienza con una comparación del aprendizaje por refuerzo con otros métodos de aprendizaje automático. Después, se proporciona toda la base teórica necesaria para comprender el aprendizaje por refuerzo. Para terminar, explicaremos en profundidad Q-Learning, el algoritmo que utilizaremos en nuestras pruebas.
    \item \textbf{Capítulo 3, Q-Learning en acción.} Introduciremos OpenAI Gym, la herramienta utilizada durante nuestras pruebas. Aplicaremos los conocimientos de aprendizaje por refuerzo en CartPole, un problema que nos permitirá evaluar resultados y limitaciones.
    \item \textbf{Capítulo 4, Redes Neuronales y Q-Learning.} Proporcionaremos el marco teórico necesario para comprender las redes neuronales. Después explicaremos cómo es posible aplicarlas a técnicas de aprendizaje por refuerzo, obteniendo las DQN.
    \item \textbf{Capítulo 5, DQNs en acción.} Volveremos a resolver CartPole y nos enfrentaremos a MountainCar, un problema que presenta nuevos retos para nuestro agente.
    \item \textbf{Capítulo 6, Conclusiones.} Síntesis de todo lo aprendido durante el camino. Puntos clave de las distintas etapas, limitaciones y oportunidades de cara al futuro.
    
\end{itemize}
