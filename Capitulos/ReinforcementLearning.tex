\chapter{Reinforcement Learning}
\label{cap:reinforcementLearning}

En el que estudiaremos cómo un agente puede aprender a través del éxito y el fracaso, mediante recompensas y penalizaciones.

\chapterquote{El comportamiento es modelado y mantenido por sus consecuencias}{B. F. Skinner}

\section{Contexto}
La \textbf{inteligencia artificial} es una de las ramas de la computación que más interés genera en las personas, sean o no expertas en la informática. Dotar a una maquina de la capacidad de realizar funciones asociadas sólo al intelecto humano ha sido siempre considerado ciencia ficción. Pero, ¿cómo podemos definir la inteligencia artificial dejando de lado la ciencia ficción?

Es común hacer referencia al \textbf{Test de Turing}, propuesto por Alan Turing (1950) \todo{referencia}. En él una persona realiza un interrogatorio, el cual se considera superado si el interrogador no es capaz de discernir si las respuestas provienen de una máquina o una persona.

Intentar definir la inteligencia artificial nos lleva rápidamente a conceptos como el \textit{proceso del pensamiento} o el \textit{razonamiento}, los cuales terminan por conducir a otros más complejos como el \textit{comportamiento}. A partir de estas ideas podemos encontrar otras definiciones, clasificadas en la matriz~\ref{fig:tabla_IA}.

\figura{Bitmap/Capitulo2/tabla_IA}{width=1\textwidth}{fig:tabla_IA}%
       {Definiciones de inteligencia artificial}

En base a estas clasificaciones podríamos diferenciar dos corrientes de interpretación:
\begin{itemize}
    \item Una visión empírica (columna izquierda) con el ser humano como centro de investigación.
    \item Una visión racionalista (columna derecha) que involucra una combinación de matemáticas e ingeniería.
\end{itemize}

Multitud de expertos han abordado ambos acercamientos de distintas formas. Nosotros nos centraremos en la visión empírica, buscando que nuestro \textit{agente} tome las ``decisiones correctas'' en función del conocimiento que posea.

\subsection{Aprendiendo a aprender}
Una de las funciones humanas de las que necesitaremos dotar a nuestra máquina en busca de este \textit{rendimiento ideal} es el \textbf{aprendizaje}.

El campo del \textbf{aprendizaje automático} (o \textit{machine learning}) se encarga de esta tarea, a través de la generalización y la búsquedas de patrones en experiencias pasadas. Existen diferentes técnicas de aprendizaje, diseñadas para distintos casos y objetivos:
\begin{itemize}
    \item \textbf{Aprendizaje supervisado} (\textit{supervised learning}): \todo{hacer}
    \item \textbf{Aprendizaje no supervisado} (\textit{unsupervised learning}): \todo{hacer}
    \item \textbf{Aprendizaje por refuerzo} (\textit{reinforcement learning}): 
\end{itemize}
