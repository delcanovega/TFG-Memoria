\chapter{DQNs en acción}
\label{cap:dqnEnAccion}

\chapterquote{La vida sólo puede entenderse hacia atrás, pero debe vivirse hacia adelante}{Søren Kierkegaard}


\section{CartPole}
\label{sec:cartpoleDQN}

El primer problema que trataremos al trabajar con DQNs se corresponde con el de CartPole, ya descrito en el punto \ref{cartpole-sec}, en el que jugamos con un carro, cuyo objetivo es moverse a derecha o izquierda para evitar que el poste vertical que hay sobre él se caiga.


\subsection{Reemplazar la tabla-Q: \texttt{SimpleAgent}}

Como primer paso para la resolución del problema, se sustituyó la tabla-Q y su consecuente discretización de estados por una red neuronal \ref{fig:nn_cartpole}, abandonando por fin los problemas y limitaciones que habíamos observado anteriormente. La red neuronal recibe como entrada el \textbf{estado} actual (posición del carro, velocidad del carro, posición del péndulo y velocidad del péndulo) y produce dos salidas: el resultado (aproximado por la red neuronal) de la partida si el agente ejerciese fuerza hacia la izquierda y el resultado hacia la derecha. Después de ejecutar cada movimiento, se aplicaba la ecuación de Bellman para entrenar la red e intentar que el agente aprendiese a jugar.

\figura{Bitmap/DQNEnAccion/nn_cartpole.png}{width=1\textwidth}{fig:nn_cartpole}%
       {Estructura de la red usada}

En el fragmento de código \ref{code:dqn} podemos ver que la implementación del algoritmo es muy similar a la usada en \ref{code:q-learning}, con unas modificaciones que describiremos a continuación.

\begin{minipage}{0.9\linewidth}%
    \begin{lstlisting}[frame=tb, caption=Pseudocódigo SimpleAgent, inputencoding=latin1, label=code:dqn]
    
    Inicializar red, estados y acciones
    
    Repetir (para cada episodio):
        Inicializar estado-actual (s)
        
        Repetir (para cada paso del episodio):
            Predecir accion
            Tomar la accion
            Entrenar red
            estado-actual = estado-siguiente
        Hasta que el episodio termine
    \end{lstlisting}%
\end{minipage}

Para \textbf{predecir acciones} utilizamos la red neuronal, a la cual introducimos el estado actual para que nos devuelva los resultados de ambas acciones, de los cuales nos quedaremos con el mejor. Una vez ejecutada la acción disponemos del estado siguiente y la recompensa, lo cual podemos usar pare \textbf{entrenar la red neuronal} de nuevo.

Como cabía esperar con una modificación tan ingenua, los resultados no son buenos, como podemos apreciar en la imagen \ref{fig:CartPole-SimpleAgent}. El agente no logra aprender de una forma estable. Si miramos atrás, en nuestra solución con una tabla-Q, el único estado que es modificado tras un entrenamiento es aquél en el que nos encontramos. Por otra parte cuando trabajamos con una red neuronal esto no es posible. Entrenar una red neuronal implica que toda la red se modificará, buscando satisfacer el nuevo caso de entrenamiento proporcionado \citep{Buduma:backprop}. En otras palabras, el agente mejora su rendimiento en el estado actual, pero al mismo tiempo empeora en todos los demás.

Los hiperparámetros usados también son una ligera modificación de los usados en la solución de Aprendizaje por Refuerzo. Como puede verse en \ref{code:cartpole_drl} el mayor cambio es en la tasa de aprendizaje, la cual deberá ser mucho menor para intentar evitar las grandes modificaciones en la red mencionadas en el párrafo anterior.

\begin{minipage}{0.9\linewidth}%
    \begin{lstlisting}[frame=tb, language=Python, caption=Hiperparámetros, label=code:cartpole_drl]
    LEARNING_RATE = 0.0001  # Alpha
    DISCOUNT_FACTOR = 0.95  # Gamma
    EXPLORATION = 0.5       # Epsilon (initial)
    MIN_EXPLORATION = 0.01  # Epsilon (final)
    \end{lstlisting}%
\end{minipage}

\figura{Bitmap/DQNEnAccion/Results-SimpleAgent.png}{width=0.7\textwidth}{fig:CartPole-SimpleAgent}%
       {Resultados del \texttt{SimpleAgent}}

\subsection{Añadir memoria al agente: \texttt{BatchAgent}}

Otra manera de conseguir estabilidad es añadiendo una memoria al agente. Para hacerlo recurriremos al reproductor de experiencias (o \textit{Experience Replay} \citep{Lin1992}), también conocidas como valores-Q, al que nos referiremos como \textbf{replay}. Éste método consiste en almacenar múltiples experiencias pasadas del agente y, antes de cada entrenamiento, tomar una muestra del conjunto de experiencias acumuladas a lo largo de varios episodios.

En nuestra memoria guardamos todos los datos necesarios para la perfecta representación de lo sucedido en pasos de ejecución anteriores, lo que se traduce en: \textbf{estado-actual}, \textbf{acción-tomada}, \textbf{recompensa}, \textbf{estado-siguiente} y \textbf{finalizado}.

\begin{minipage}{0.9\linewidth}%
    \begin{lstlisting}[frame=tb, caption=Pseudocódigo BatchAgent, inputencoding=latin1, label=code:cartpole_drl2]
    
    Inicializar red, estados y acciones
    
    Repetir (para cada episodio):
        Inicializar estado-actual (s)
        
        Repetir (para cada paso del episodio):
            Predecir accion
            Tomar la accion
            Almacenar resultados en memoria
            Entrenar red con datos de la memoria
            estado-actual = estado-siguiente
        Hasta que el episodio termine
    \end{lstlisting}%
\end{minipage}

Tras la ejecución de cada acción, y con la información proporcionada por el entorno en ese momento, guardamos todos estos valores en la memoria para ser utilizados posteriormente. De esta forma contamos con un conjunto ($batch$) de los últimos $N$ pasos guardados en la memoria. De esta forma podemos entrenar la red neuronal con un mayor número de casos de prueba, intentando evitar así la inestabilidad del \texttt{SimpleAgent}.

El problema es que los pasos tomados dentro de estos $batch$ están fuertemente cohesionados entre sí, ya que al ser consecutivos la información aportada por cada uno de ellos era muy similar. La memoria de estados consecutivos proveía de más información a la red pero las similitudes entre ellos provocaban que el agente aprendiese muy lentamente, de forma insuficiente para la resolución del problema, tal y como se puede ver en la imagen \ref{fig:CartPole-BatchAgent}.

\figura{Bitmap/DQNEnAccion/Results-BatchAgent.png}{width=0.7\textwidth}{fig:CartPole-BatchAgent}%
       {Resultados del \texttt{BatchAgent}}


\subsection{Romper la cohesión de la memoria: \texttt{RandomBatchAgent}}

Por fortuna solucionar el problema de la cohesión en el \texttt{BatchAgent} tiene fácil solución. Tan sólo necesitaremos que la memoria de nuestro agente sea lo bastante grande como par albergar experiencias lo bastante variadas, después sólo tendremos que seleccionar elementos aleatorios de la memoria. Con esta solución, el agente debería tener una muestra más amplia respecto a los estados posibles y sería mucho más fácil que encontrase una solución con prontitud.

\begin{minipage}{0.9\linewidth}%
    \begin{lstlisting}[frame=tb, caption=Pseudocódigo RandomBatchAgent, inputencoding=latin1, label=code:cartpole_drl3]
    
    Inicializar red, estados y acciones
    
    Repetir (para cada episodio):
        Inicializar estado-actual (s)
        
        Repetir (para cada paso del episodio):
            Predecir accion
            Tomar la accion
            Almacenar resultados en memoria
            Entrenar con datos aleatorios de la memoria
            estado-actual = estado-siguiente
        Hasta que el episodio termine
    \end{lstlisting}%
\end{minipage}

El resultado es notoriamente mejor, como muestra la imagen \ref{fig:CartPole-RandomBatchAgent}. De esta forma el agente consigue resolver el problema de forma rápida, además de puntuaciones mucho más altas en general. Sin embargo, entre unas simulaciones y otras los resultados son muy dispares, desde encontrar rápidamente y reforzar un camino ``bueno'' hasta lograr la resolución del problema, hasta no encontrar nunca un camino válido. En éste último caso podría deberse a que las muestras tomadas de la memoria no son lo suficiente variadas y el estado de la red neuronal se degrada progresivamente.

El problema del volumen de datos estaba solucionado, ya que en varias ocasiones el agente había conseguido aprender una solución, y en las que no, al menos daba la impresión de que iba por buen camino. No obstante, la volatilidad de los resultados nos daba a entender que el modelo propuesto aún no estaba completo.

\figura{Bitmap/DQNEnAccion/Results-RandomBatchAgent.png}{width=0.7\textwidth}{fig:CartPole-RandomBatchAgent}%
       {Resultados del \texttt{RandomBatchAgent}}


\subsection{Estabilizar la red: \texttt{DoubleAgent}}
\label{sec:DA}

Con el fin de en estabilizar el aprendizaje del agente, intentando evitar esos escollos que el agente puede experimentar en algunas ejecuciones, surge por parte de \textit{DeepMind} el concepto de la \textbf{Target Q-Network} \citep{NIPS2010_3964}. Éste implica la utilización de una segunda red objetivo (\textit{target}), como copia de la red principal de predicción. La red objetivo sólo se actualiza cada cierto número de pasos, ahorrándose los cambios en los pesos que sufre constantemente la red de predicción y consiguiendo así más estabilidad en los resultados.

$$Y^Q_{t} \equiv R_{t+1} + \gamma max_{a} Q(S_{t+1}, a; \theta_{t})$$

Como derivación de este método existe otro conocido como \textbf{Double Q-Network}, donde las redes se dividen entre ``agente'' y ``aprendiz''. La red agente se utiliza para guardar los estados con los que tomar las decisiones en cuanto a los movimientos a ejecutar, mientras que la red de aprendizaje se encarga de guardar los valores de la ejecución actualizados: cada cierto número de pasos ambas redes son comparadas, guardando la que haya dado mejores resultados como agente, para la toma de decisiones, y utilizando la otra para el aprendizaje.

$$Y^Q_{t} = R_{t+1} + \gamma Q(S_{t+1}, argmax_{a} Q(S_{t+1}, a; \theta_{t}); \theta_{t})$$

El principal inconveniente que mostraba nuestro agente hasta el momento se debía a la gran diferencia entre las ejecuciones, dando unos resultados que era necesario estabilizar. Tomando como referencia las soluciones consideradas en el capítulo anterior, optamos por el uso de dos redes diferentes.

\begin{minipage}{0.9\linewidth}%
    \begin{lstlisting}[frame=tb, caption=Pseudocódigo DoubleAgent, inputencoding=latin1, label=code:cartpole_drl4]
    
    Inicializar red, estados y acciones
    
    Repetir (para cada episodio):
        Inicializar estado-actual (s)
        
        Repetir (para cada paso del episodio):
            Predecir accion con agente
            Tomar la accion
            Almacenar resultados en memoria
            Entrenar aprendiz con datos aleatorios
            estado-actual = estado-siguiente
        Hasta que el episodio termine

        Si aprendiz mejor que agente:
            agente = aprendiz
    \end{lstlisting}%
\end{minipage}

Tras la implementación de este método, comprobamos que lográbamos una mayor consistencia en los datos entre unas simulaciones y otras, obteniendo resultados menos dispares que con el entrenamiento por conjunto aleatorio como podemos observar en la imagen \ref{fig:CartPole-DoubleAgent}.

\figura{Bitmap/DQNEnAccion/Results-DoubleAgent.png}{width=0.7\textwidth}{fig:CartPole-DoubleAgent}%
       {Resultados del \texttt{DoubleNetworkAgent}}

Gracias a este método se han logrado grandes hitos en el campo del Aprendizaje por Refuerzo Profundo. Por ejemplo, hacer que una red neuronal sea capaz de jugar a juegos de Atari a nivel profesional utilizando tan solo los píxeles como entrada \citep{mnih2013playing}


\section{MountainCar}

Con el fin de comprobar la versatilidad de nuestro agente lo pondremos a prueba en un entorno distinto, este es su enunciado:

\begin{quote}
    Un coche se encuentra en una pista unidimensional, posicionado entre dos "montañas". El objetivo es alcanzar la cima de la montaña derecha; sin embargo, el motor del coche no es lo bastante potente como para alcanzar la cima de la montaña con un único impulso. Por ello, la única forma de lograrlo es conducir hacia delante y hacia atrás aumentando el impulso.
\end{quote}

\figura{Bitmap/DQNEnAccion/MountainCar_01}{width=0.7\textwidth}{fig:mountaincar_01}%
       {Entorno de simulación MountainCar, \citet{brockman2016openai}}

Este problema fue descrito por primera vez por A. \citet{Moore90efficientmemory-based}.

\subsection{Especificación del problema}

\begin{itemize}
    \item \textbf{Estado:} el estado de este problema nos viene dado a través de dos observaciones: la posición respecto al eje de abscisas y la velocidad del coche.
    \item \textbf{Recompensa:} la recompensa recibida será de \texttt{-1} por cada paso en el que el coche no alcance la meta.
    \item \textbf{Terminación:} el episodio terminará con éxito cuando el coche alcance la meta, o fracasará si no lo logra tras 200 pasos.
    \item \textbf{Acciones:} dispondremos de tres acciones disponibles, impulsarse hacia la derecha, impulsarse hacia la izquierda y no hacer nada. El cada situación el agente deberá elegir la opción que más impulso le ayude a acumular.
\end{itemize}

A primera vista puede parecer que es un problema muy parecido a CartPole, no obstante tiene detalles que lo convierten en un problema especialmente interesante. En particular su función de recompensa, recibir \texttt{-1} en cada paso, combinado con el hecho de que la simulación sólo termine tras 200 pasos (si no se alcanza la meta), hace que prácticamente todas las simulaciones terminen con un resultado de -200. Esto elimina el factor de progresión del que disponíamos en CartPole, donde el agente generalmente iba obteniendo recompensas pequeñas pero cada vez mayores.

Para resolver este problema partiremos de la versión más estable de nuestro agente desarrollado en \ref{sec:cartpoleDQN}.


\subsection{Enfoques de resolución}

Ejecutando el problema con el agente descrito en \ref{sec:DA} y los parámetros usados en \ref{code:cartpole_drl}, obtendremos como resultado que el agente no aprende en absoluto. La topología interna de la red también se mantiene igual, por supuesto adaptando el número de entradas y salidas a las del problema; ya que CartPole describía su estado a través de cuatro variables y tenía dos acciones disponibles, frente a las dos variables y tres acciones de las que MountainCar dispone.

Los resultados tan pobres obtenidos puede que sorprendan viniendo del rendimiento visto en CartPole, pero recordemos que el tamaño de las capas internas de la red se eligió para ese problema en concreto. Es de esperar que distintos escenarios necesiten redes de distinto tamaño que sean capaces de interpretar las observaciones de manera efectiva. En definitiva, nuestra red actual \textbf{no es lo bastante descriptiva} para la complejidad del problema.


\subsubsection{Nueva topología e hiperparámetros}

Modificando el agente para usar la siguiente red e hiperparámetros:

((Imagen de la topología))

\begin{minipage}{0.9\linewidth}%
    \begin{lstlisting}[frame=tb, language=Python, caption=Hiperparámetros MountainCar, label=code:cartpole_drl]
    LEARNING_RATE = 0.0001  # Alpha
    DISCOUNT_FACTOR = 0.95  # Gamma
    EXPLORATION = 0.5       # Epsilon (initial)
    MIN_EXPLORATION = 0.01  # Epsilon (final)
    \end{lstlisting}%
\end{minipage}

Gracias a la nueva topología y mediante un entrenamiento bastante largo, en la imagen \ref{fig:mountaincar_02} podemos comprobar que el agente llega a ser capaz de resolver el problema. No obstante, creemos que el resultado tiene mucho margen de mejora, por lo que seguiremos explorando otras soluciones que permitan al agente aprender de una forma más eficiente

\figura{Bitmap/DQNEnAccion/mountainCar_02}{width=0.7\textwidth}{fig:mountaincar_02}%
       {Resultados de la nueva topología}


\subsubsection{Función de recompensa}

El primer obstáculo a superar para acelerar la curva de aprendizaje es la función de la recompensa. La recompensa proporcionada por el entorno no es lo bastante útil para el agente. Imagina que haces un examen y, sin importar las veces que lo intentes o las respuestas que des, tu puntuación siempre es un cero. De esta forma es muy difícil identificar qué puntos en particular debe reforzar primero.

Nuestra solución pasa por sustituir esa recompensa dada por el entorno \citep{mediumTim}, reemplazando ese resultado binario de \textit{meta} o \textit{no meta} por algo más descriptivo, que empuje al agente a mejorar su funcionamiento. Utilizando la \textbf{velocidad} o la \textbf{posición} del agente como recompensa motiva al mismo a tomar las decisiones que lo llevan a moverse más lejos, hasta que finalmente termine alcanzando la meta.

\figura{Bitmap/DQNEnAccion/mountainCar_03}{width=0.8\textwidth}{fig:mountaincar_03}%
       {Visualización de la recompensa modificada}

Como podemos ver en la imagen \ref{fig:mountaincar_03}, usando una recompensa modificada favorecemos que el coche se mueva e intente llegar cada vez más lejos, a diferencia de la recompensa original con la cual el coche suele quedarse estancado en el fondo.

A continuación, evaluaremos en mayor detalle cada una de las opciones disponibles a la hora de reemplazar la recompensa, para ver cuál se adapta mejor al problema y nos proporciona mejores resultados.


\subsubsection*{Posición}

La posición es el primer candidato para utilizar como recompensa. Al fin y al cabo, si conseguimos que el coche avance lo bastante, terminará llegando a la meta. Hicimos una serie de pruebas en las que directamente reemplazábamos el valor de la recompensa por el valor de la posición, obteniendo los resultados de \ref{fig:mountaincar_04}. Para ser capaces de apreciar en qué momento el agente alcanzaba la meta, le otorgamos una recompensa significativamente más grande cuando éste la alcanzaba, con la esperanza también de que esto reforzase mucho más ese comportamiento.

\figura{Bitmap/DQNEnAccion/mountaincar_04}{width=0.7\textwidth}{fig:mountaincar_04}%
       {Visualización de la recompensa modificada}

Como podemos observar, el rendimiento sigue siendo muy parecido al visto en la gráfica \ref{fig:mountaincar_02}. No es hasta los 1400 episodios cuando el agente consigue alcanzar la meta. A partir de los nuevos resultados podemos hacer las siguientes observaciones:

\begin{itemize}
    % \item En primer lugar, dimos una recompensa exageradamente alta al alcanzar la meta, no solo para que ese camino se refuerce más que con la recompensa original, para acelerar el aprendizaje, sino para además permitirnos distinguir más fácilmente (en la gráfica) las partidas que llegan a la meta de las que no.  
    \item Nuestra nueva recompensa sólo empuja al agente a intentar escalar la rampa de la derecha, ya que es la dirección que hace crecer el valor de la \texttt{posición}. No obstante, para conseguir escalar por la derecha necesita también tomar impulso desde la rampa de la izquierda, lo que se corresponde a valores negativos de la posición. Por ello, tendríamos que considerar usar la posición en valores absolutos para buscar un mejor resultado.
    \item Otro detalle es que la posición central no se corresponde con \texttt{0}, si no con \texttt{-0.5}, por lo que también tendremos que corregir esta desviación.
    % \item Por último, el agente no recibe ningún estímulo que lo lleve a resolver el problema cuanto antes, por lo que volveremos a introducir una penalización por cada paso tomado, de forma similar a la recompensa original del entorno.
\end{itemize}

\figura{Bitmap/DQNEnAccion/mountaincar_05}{width=0.7\textwidth}{fig:mountaincar_05}%
       {Resultados finales de la posición como recompensa}

Teniendo en cuenta estos nuevos factores, los resultados obtenidos pueden verse en la gráfica \ref{fig:mountaincar_05}. Finalmente conseguimos que el problema se resuelva mucho más rápido: 500 episodios frente a los 1400 anteriores. No obstante, la estabilidad sigue lejos de lo visto en otros problemas como CartPole. Seguimos teniendo situaciones en las que el agente consigue resolver el problema en algunas partidas, pero luego vuelve a fracasar en las siguientes.

% La grafica representa claramente como la puntuación converge a un valor, esto se debe a que la penalización crece demasiado.
% Penalización inicial empieza en 1, sin penalización, y a cada Step esta se multiplica por 0.995, de forma que la penalización del último Step es de 0.36695. Esto hace que las puntuaciones de los últimos Steps de la partida sean despreciables para el sistema, de tal forma que se optimicen solo las recompensas iniciales e ignorando las de los últimos. Así llegando a la convergencia. 
% Además, nos dimos cuenta de que este tipo de penalización incumplía lo establecido de basar la recompensa únicamente en el estado, sin tener en cuenta factores externos como, en el caso de la penalización, el número de Steps que lleva la partida.
% Ante los inconvenientes de la penalización optamos por descartar ese método y en su lugar añadir una pequeña disminución de la recompensa, en caso de alcanzar la meta, según el número de pasos realizados al final del Episodio para reforzar las soluciones rápidas frente a las lentas.

% RESULTADOS: mainPosicion


\subsubsection*{Velocidad}

Otra opción explorada es la de usar la velocidad como recompensa. Al fin y al cabo, cuanta más velocidad consiga acumular el agente, más alto llegará escalando la rampa.

\figura{Bitmap/DQNEnAccion/mountaincar_06}{width=0.7\textwidth}{fig:mountaincar_06}%
       {Resultados finales de la posición como recompensa}

A diferencia de la posición, la velocidad devuelta por el entorno tiene unos valores muy pequeños, por lo que basar la recompensa en la velocidad recibida sería semejante a darle una recompensa constante, para cuando se alcance la meta darle una notablemente mayor, lo cual nos dejaría en una situación muy similar a la vista en \ref{fig:mountaincar_02}. Por ello decidimos multiplicarla por 100, haciéndola más visible y relevantes para el agente. 

Como vemos en la imagen \ref{fig:mountaincar_06} nos encontramos con un comportamiento muy similar al visto usando la posición en valores absolutos. El agente consigue resolver el problema en varias ocasiones pero no se estabiliza.

% RESULTADO: main3.2 y  


\subsubsection*{Otras pruebas}

Puesto que los resultados obtenidos por la posición y velocidad parecían haberse estancado, decidimos investigar otras piezas clave de nuestro modelo, como el optimizador. Empezamos a cuestionarnos si Adam \citep{kingma2014adam} era nuestra mejor opción, e hicimos algunas pruebas con Adadelta \citep{NIPS2017_7003}.

Para poder comparar lo mejor posible los resultamos decidimos crear una serie de gráficas para cada ejemplo en el que no solo se viera mejor la evolución general de los resultados, sino que añadiera otros datos de interés como la recompensa original del entorno, la velocidad máxima la posición máxima alcanzada. Así seríamos capaces de vigilar en una sola simulación todas las variables.

\figura{Bitmap/DQNEnAccion/mountaincar_07}{width=1\textwidth}{fig:mountaincar_07}%
       {Resultados ¿?}

SIN REVISAR A PARTIR DE AQUÍ

Por último, decidimos añadir al agente una modificación para favorecer la convergencia de los resultados del sistema y reducir las caídas vistas en las gráficas.
Esto lo conseguimos añadiendo un tercer modelo al agente, siendo su composición final de:
\begin{itemize}
 \item Modelo Mentor: se limitará a responder a los estímulos del entorno sin hacer replay de sus jugadas, es decir, sin aprender y simplemente almacenando en la memoria los estados por los que ha pasado y sus correspondientes recompensas.
 \item Modelo Aprendiz: se dedicará a aprender, haciendo los replays de la batería de estados-recompensa que va generando el Modelo Mentor.
 \item Modelo Best: será el mejor de los modelos encontrados hasta el momento. 
\end{itemize}
El desarrollo de la ejecución se divide en 3 fases:

Primera fase, fase Inicial, durante los primeros episodios, equivalente a dos veces la periodicidad de las competiciones (COMP), el modelo mentor aprende del entorno tal y como hacen los agentes predecesores, jugando la partida y aprendiendo mediante backtracking. Mientras que los otros dos, Aprendiz y Best no hacen nada, se limitan a inicializarse.

La segunda fase, fase de Competiciones, Mentor se limita a interactuar con el entorno, es decir, toma las decisiones que controlan al avatar durante la partida, y guarda los resultados obtenidos en la memoria. Por otro lado, Aprendiz comienza a entrenar con dicha memoria, aprendiendo de las acciones de Mentor.
Cada COMP episodios, se produce una competición, que somete a Mentor y Aprendiz a una prueba, que no tiene repercusión alguna en la memoria y permite comparar los resultados para ver quien reacciona mejor en el entorno. La que salga victoriosa se copia sobre la vencida, y prosigue el proceso.
Así, si Aprendiz aprende de las experiencias de Mentor, y es capaz de superarle en la prueba, pasará a ser Mentor, dejando una copia en Aprendiz para seguir desarrollándose. Por otro lado, si Aprendiz aprende “mal”, es decir, si los resultados de la prueba indican que Mentor es mejor que Aprendiz, Mentor trasmitirá su configuración de los pesos a Aprendiz para que este proceda a mejorarlo y aprenda de Mentor hasta la siguiente Competición.  

Fase final, fase de Mantenimiento, es semejante a Competición, salvo que ahora Best guarda el mejor modelo que se haya generado durante la simulación. Cada 100 episodios, los tres modelos se pondrán a prueba con 100 partidas de testeo, que al igual que durante las competiciones normales, no tendrán ninguna repercusión en la memoria y por tanto serán invisibles para el entrenamiento. El mejor modelo se guardará en Best y se pondrá como Mentor, para así generar los mejores resultados posibles de los que Aprendiz pueda aprender.  

RESULTADOS: 	main7SustituyendoModelPorBestEXT, main7SustituyendoModelPorBestEXT2, main7SustituyendoModelPorBestEXT4


% \subsubsection*{Variantes}
% Nos planteamos otros aderezos para las soluciones ya planteadas para optimizar el aprendizaje. 
% Consideramos añadir pequeñas recompensas como dar una bonificación extra al alcanzar cierta posición o al llevar una velocidad positiva.
% Algunas de estas fueron descartadas por la premisa de solo basarnos en el estado para definir la recompensa, pero otras pudimos llevarlas a cabo.

%  Dar bonus +10 cada vez que alcanza un multiplo de 0.1

%  Dar bonus al llevar velocidad positiva al pasar por 0

%  Dar bonus al superar su velocidad máxima al pasar por cero(se dedicaba a pasar muchas veces por cero)

%  Dar una bonificación cada vez que supera mu mayor altura por 0.1, a primera vista puede parecer que de elementos externos al estado pero... teniendo en cuenta que se trata de solo posiciones de 0.1, al final solo se recompensan 0, 0.1, 0.2... sería equivalente la primera propuesta pero alternando la ladera. 

% ((algoritmo de modificacion))

% ((Gráfica de resultados))

% \subsubsection*{Otros enfoques}
% Hablar sobre randomwalks y transfer learning

% \paragraph{Randomwalks}
% A grandes rasgos Randomwalks, consiste en generar una gran memoria de partidas jugadas totalmente al azar para luego seleccionar los mejores resultados de entre estos para entrenar el agente. Por último, poner el agente en el entorno real.
% Este planteamiento ha resultado ser con diferencia el más rapido de todos, dado que al ahorarse el entrenamiento intermedio de todas las partidas se ahora gran parte del tiempo de ejecución.


