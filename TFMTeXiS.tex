% ----------------------------------------------------------------------
%
%                            TFMTesis.tex
%
%----------------------------------------------------------------------
%
% Este fichero contiene el "documento maestro" del documento. Lo único
% que hace es configurar el entorno LaTeX e incluir los ficheros .tex
% que contienen cada sección.
%
%----------------------------------------------------------------------
%
% Los ficheros necesarios para este documento son:
%
%       TeXiS/* : ficheros de la plantilla TeXiS.
%       Cascaras/* : ficheros con las partes del documento que no
%          son capítulos ni apéndices (portada, agradecimientos, etc.)
%       Capitulos/*.tex : capítulos de la tesis
%       Apendices/*.tex: apéndices de la tesis
%       constantes.tex: constantes LaTeX
%       config.tex : configuración de la "compilación" del documento
%       guionado.tex : palabras con guiones
%
% Para la bibliografía, además, se necesitan:
%
%       *.bib : ficheros con la información de las referencias
%
% ---------------------------------------------------------------------

\documentclass[11pt,a4paper,twoside]{book}

%
% Definimos  el   comando  \compilaCapitulo,  que   luego  se  utiliza
% (opcionalmente) en config.tex. Quedaría  mejor si también se definiera
% en  ese fichero,  pero por  el modo  en el  que funciona  eso  no es
% posible. Puedes consultar la documentación de ese fichero para tener
% más  información. Definimos también  \compilaApendice, que  tiene el
% mismo  cometido, pero  que se  utiliza para  compilar  únicamente un
% apéndice.
%
%
% Si  queremos   compilar  solo   una  parte  del   documento  podemos
% especificar mediante  \includeonly{...} qué ficheros  son los únicos
% que queremos  que se incluyan.  Esto  es útil por  ejemplo para sólo
% compilar un capítulo.
%
% El problema es que todos aquellos  ficheros que NO estén en la lista
% NO   se  incluirán...  y   eso  también   afecta  a   ficheros  de
% la plantilla...
%
% Total,  que definimos  una constante  con los  ficheros  que siempre
% vamos a querer compilar  (aquellos relacionados con configuración) y
% luego definimos \compilaCapitulo.
\newcommand{\ficherosBasicosTeXiS}{%
TeXiS/TeXiS_pream,TeXiS/TeXiS_cab,TeXiS/TeXiS_bib,TeXiS/TeXiS_cover%
}
\newcommand{\ficherosBasicosTexto}{%
constantes,guionado,Cascaras/bibliografia,config%
}
\newcommand{\compilaCapitulo}[1]{%
\includeonly{\ficherosBasicosTeXiS,\ficherosBasicosTexto,Capitulos/#1}%
}

\newcommand{\compilaApendice}[1]{%
\includeonly{\ficherosBasicosTeXiS,\ficherosBasicosTexto,Apendices/#1}%
}

%- - - - - - - - - - - - - - - - - - - - - - - - - - - - - - - - - - -
%            Preámbulo del documento. Configuraciones varias
%- - - - - - - - - - - - - - - - - - - - - - - - - - - - - - - - - - -

% Define  el  tipo  de  compilación que  estamos  haciendo.   Contiene
% definiciones  de  constantes que  cambian  el  comportamiento de  la
% compilación. Debe incluirse antes del paquete TeXiS/TeXiS.sty
%---------------------------------------------------------------------
%
%                          config.tex
%
%---------------------------------------------------------------------
%
% Contiene la  definición de constantes  que determinan el modo  en el
% que se compilará el documento.
%
%---------------------------------------------------------------------
%
% En concreto, podemos  indicar si queremos "modo release",  en el que
% no  aparecerán  los  comentarios  (creados  mediante  \com{Texto}  o
% \comp{Texto}) ni los "por  hacer" (creados mediante \todo{Texto}), y
% sí aparecerán los índices. El modo "debug" (o mejor dicho en modo no
% "release" muestra los índices  (construirlos lleva tiempo y son poco
% útiles  salvo  para   la  versión  final),  pero  sí   el  resto  de
% anotaciones.
%
% Si se compila con LaTeX (no  con pdflatex) en modo Debug, también se
% muestran en una esquina de cada página las entradas (en el índice de
% palabras) que referencian  a dicha página (consulta TeXiS_pream.tex,
% en la parte referente a show).
%
% El soporte para  el índice de palabras en  TeXiS es embrionario, por
% lo  que no  asumas que  esto funcionará  correctamente.  Consulta la
% documentación al respecto en TeXiS_pream.tex.
%
%
% También  aquí configuramos  si queremos  o  no que  se incluyan  los
% acrónimos  en el  documento final  en la  versión release.  Para eso
% define (o no) la constante \acronimosEnRelease.
%
% Utilizando \compilaCapitulo{nombre}  podemos también especificar qué
% capítulo(s) queremos que se compilen. Si no se pone nada, se compila
% el documento  completo.  Si se pone, por  ejemplo, 01Introduccion se
% compilará únicamente el fichero Capitulos/01Introduccion.tex
%
% Para compilar varios  capítulos, se separan sus nombres  con comas y
% no se ponen espacios de separación.
%
% En realidad  la macro \compilaCapitulo  está definida en  el fichero
% principal tesis.tex.
%
%---------------------------------------------------------------------


% Comentar la línea si no se compila en modo release.
% TeXiS hará el resto.
% ¡¡¡Si cambias esto, haz un make clean antes de recompilar!!!
\def\release{1}


% Descomentar la linea si se quieren incluir los
% acrónimos en modo release (en modo debug
% no se incluirán nunca).
% ¡¡¡Si cambias esto, haz un make clean antes de recompilar!!!
%\def\acronimosEnRelease{1}


% Descomentar la línea para establecer el capítulo que queremos
% compilar

% \compilaCapitulo{01Introduccion}
% \compilaCapitulo{02EstructuraYGeneracion}
% \compilaCapitulo{03Edicion}
% \compilaCapitulo{04Imagenes}
% \compilaCapitulo{05Bibliografia}
% \compilaCapitulo{06Makefile}

% \compilaApendice{01AsiSeHizo}

% Variable local para emacs, para  que encuentre el fichero maestro de
% compilación y funcionen mejor algunas teclas rápidas de AucTeX
%%%
%%% Local Variables:
%%% mode: latex
%%% TeX-master: "./Tesis.tex"
%%% End:


% Paquete de la plantilla
\usepackage{TeXiS/TeXiS}

% Incluimos el fichero con comandos de constantes
%---------------------------------------------------------------------
%
%                          constantes.tex
%
%---------------------------------------------------------------------
%
% Fichero que  declara nuevos comandos LaTeX  sencillos realizados por
% comodidad en la escritura de determinadas palabras
%
%---------------------------------------------------------------------

%%%%%%%%%%%%%%%%%%%%%%%%%%%%%%%%%%%%%%%%%%%%%%%%%%%%%%%%%%%%%%%%%%%%%%
% Comando: 
%
%       \titulo
%
% Resultado: 
%
% Escribe el título del documento.
%%%%%%%%%%%%%%%%%%%%%%%%%%%%%%%%%%%%%%%%%%%%%%%%%%%%%%%%%%%%%%%%%%%%%%
\def\titulo{\textsc{TeXiS}: Una plantilla de \LaTeX\
  para Tesis y otros documentos}

%%%%%%%%%%%%%%%%%%%%%%%%%%%%%%%%%%%%%%%%%%%%%%%%%%%%%%%%%%%%%%%%%%%%%%
% Comando: 
%
%       \autor
%
% Resultado: 
%
% Escribe el autor del documento.
%%%%%%%%%%%%%%%%%%%%%%%%%%%%%%%%%%%%%%%%%%%%%%%%%%%%%%%%%%%%%%%%%%%%%%
\def\autor{Marco Antonio y Pedro Pablo G\'omez Mart\'in}

% Variable local para emacs, para  que encuentre el fichero maestro de
% compilación y funcionen mejor algunas teclas rápidas de AucTeX

%%%
%%% Local Variables:
%%% mode: latex
%%% TeX-master: "tesis.tex"
%%% End:


% Sacamos en el log de la compilación el copyright
%\typeout{Copyright Marco Antonio and Pedro Pablo Gomez Martin}

%
% "Metadatos" para el PDF
%
\ifpdf\hypersetup{%
    pdftitle = {\titulo},
    pdfsubject = {Plantilla de Tesis},
    pdfkeywords = {Plantilla, LaTeX, tesis, trabajo de
      investigación, trabajo de Master},
    pdfauthor = {\textcopyright\ \autor},
    pdfcreator = {\LaTeX\ con el paquete \flqq hyperref\frqq},
    pdfproducer = {pdfeTeX-0.\the\pdftexversion\pdftexrevision},
    }
    \pdfinfo{/CreationDate (\today)}
\fi


%- - - - - - - - - - - - - - - - - - - - - - - - - - - - - - - - - - -
%                        Documento
%- - - - - - - - - - - - - - - - - - - - - - - - - - - - - - - - - - -
\begin{document}

% Incluimos el  fichero de definición de guionado  de algunas palabras
% que LaTeX no ha dividido como debería
%----------------------------------------------------------------
%
%                          guionado.tex
%
%----------------------------------------------------------------
%
% Fichero con algunas divisiones de palabras que LaTeX no
% hace correctamente si no se le da alguna ayuda.
%
%----------------------------------------------------------------

\hyphenation{
% a
abs-trac-to
abs-trac-tos
abs-trac-ta
abs-trac-tas
ac-tua-do-res
a-gra-de-ci-mien-tos
ana-li-za-dor
an-te-rio-res
an-te-rior-men-te
apa-rien-cia
a-pro-pia-do
a-pro-pia-dos
a-pro-pia-da
a-pro-pia-das
a-pro-ve-cha-mien-to
a-que-llo
a-que-llos
a-que-lla
a-que-llas
a-sig-na-tu-ra
a-sig-na-tu-ras
a-so-cia-da
a-so-cia-das
a-so-cia-do
a-so-cia-dos
au-to-ma-ti-za-do
% b
batch
bi-blio-gra-fía
bi-blio-grá-fi-cas
bien
bo-rra-dor
boo-l-ean-expr
% c
ca-be-ce-ra
call-me-thod-ins-truc-tion
cas-te-lla-no
cir-cuns-tan-cia
cir-cuns-tan-cias
co-he-ren-te
co-he-ren-tes
co-he-ren-cia
co-li-bri
co-men-ta-rio
co-mer-cia-les
co-no-ci-mien-to
cons-cien-te
con-si-de-ra-ba
con-si-de-ra-mos
con-si-de-rar-se
cons-tan-te
cons-trucción
cons-tru-ye
cons-tru-ir-se
con-tro-le
co-rrec-ta-men-te
co-rres-pon-den
co-rres-pon-dien-te
co-rres-pon-dien-tes
co-ti-dia-na
co-ti-dia-no
crean
cris-ta-li-zan
cu-rri-cu-la
cu-rri-cu-lum
cu-rri-cu-lar
cu-rri-cu-la-res
% d
de-di-ca-do
de-di-ca-dos
de-di-ca-da
de-di-ca-das
de-rro-te-ro
de-rro-te-ros
de-sa-rro-llo
de-sa-rro-llos
de-sa-rro-lla-do
de-sa-rro-lla-dos
de-sa-rro-lla-da
de-sa-rro-lla-das
de-sa-rro-lla-dor
de-sa-rro-llar
des-cri-bi-re-mos
des-crip-ción
des-crip-cio-nes
des-cri-to
des-pués
de-ta-lla-do
de-ta-lla-dos
de-ta-lla-da
de-ta-lla-das
di-a-gra-ma
di-a-gra-mas
di-se-ños
dis-po-ner
dis-po-ni-bi-li-dad
do-cu-men-ta-da
do-cu-men-to
do-cu-men-tos
% e
edi-ta-do
e-du-ca-ti-vo
e-du-ca-ti-vos
e-du-ca-ti-va
e-du-ca-ti-vas
e-la-bo-ra-do
e-la-bo-ra-dos
e-la-bo-ra-da
e-la-bo-ra-das
es-co-llo
es-co-llos
es-tu-dia-do
es-tu-dia-dos
es-tu-dia-da
es-tu-dia-das
es-tu-dian-te
e-va-lua-cio-nes
e-va-lua-do-res
exis-ten-tes
exhaus-ti-va
ex-pe-rien-cia
ex-pe-rien-cias
% f
for-ma-li-za-do
% g
ge-ne-ra-ción
ge-ne-ra-dor
ge-ne-ra-do-res
ge-ne-ran
% h
he-rra-mien-ta
he-rra-mien-tas
% i
i-dio-ma
i-dio-mas
im-pres-cin-di-ble
im-pres-cin-di-bles
in-de-xa-do
in-de-xa-dos
in-de-xa-da
in-de-xa-das
in-di-vi-dual
in-fe-ren-cia
in-fe-ren-cias
in-for-ma-ti-ca
in-gre-dien-te
in-gre-dien-tes
in-me-dia-ta-men-te
ins-ta-la-do
ins-tan-cias
% j
% k
% l
len-gua-je
li-be-ra-to-rio
li-be-ra-to-rios
li-be-ra-to-ria
li-be-ra-to-rias
li-mi-ta-do
li-te-ra-rio
li-te-ra-rios
li-te-ra-ria
li-te-ra-rias
lo-tes
% m
ma-ne-ra
ma-nual
mas-que-ra-de
ma-yor
me-mo-ria
mi-nis-te-rio
mi-nis-te-rios
mo-de-lo
mo-de-los
mo-de-la-do
mo-du-la-ri-dad
mo-vi-mien-to
% n
na-tu-ral
ni-vel
nues-tro
% o
obs-tan-te
o-rien-ta-do
o-rien-ta-dos
o-rien-ta-da
o-rien-ta-das
% p
pa-ra-le-lo
pa-ra-le-la
par-ti-cu-lar
par-ti-cu-lar-men-te
pe-da-gó-gi-ca
pe-da-gó-gi-cas
pe-da-gó-gi-co
pe-da-gó-gi-cos
pe-rio-di-ci-dad
per-so-na-je
plan-te-a-mien-to
plan-te-a-mien-tos
po-si-ción
pre-fe-ren-cia
pre-fe-ren-cias
pres-cin-di-ble
pres-cin-di-bles
pri-me-ra
pro-ble-ma
pro-ble-mas
pró-xi-mo
pu-bli-ca-cio-nes
pu-bli-ca-do
% q
% r
rá-pi-da
rá-pi-do
ra-zo-na-mien-to
ra-zo-na-mien-tos
re-a-li-zan-do
re-fe-ren-cia
re-fe-ren-cias
re-fe-ren-cia-da
re-fe-ren-cian
re-le-van-tes
re-pre-sen-ta-do
re-pre-sen-ta-dos
re-pre-sen-ta-da
re-pre-sen-ta-das
re-pre-sen-tar-lo
re-qui-si-to
re-qui-si-tos
res-pon-der
res-pon-sa-ble
% s
se-pa-ra-do
si-guien-do
si-guien-te
si-guien-tes
si-guie-ron
si-mi-lar
si-mi-la-res
si-tua-ción
% t
tem-pe-ra-ments
te-ner
trans-fe-ren-cia
trans-fe-ren-cias
% u
u-sua-rio
Unreal-Ed
% v
va-lor
va-lo-res
va-rian-te
ver-da-de-ro
ver-da-de-ros
ver-da-de-ra
ver-da-de-ras
ver-da-de-ra-men-te
ve-ri-fi-ca
% w
% x
% y
% z
}
% Variable local para emacs, para que encuentre el fichero
% maestro de compilación
%%%
%%% Local Variables:
%%% mode: latex
%%% TeX-master: "./Tesis.tex"
%%% End:


% Marcamos  el inicio  del  documento para  la  numeración de  páginas
% (usando números romanos para esta primera fase).
\frontmatter
\pagestyle{empty}

%---------------------------------------------------------------------
%
%                          configCover.tex
%
%---------------------------------------------------------------------
%
% cover.tex
% Copyright 2009 Marco Antonio Gomez-Martin, Pedro Pablo Gomez-Martin
%
% This file belongs to the TeXiS manual, a LaTeX template for writting
% Thesis and other documents. The complete last TeXiS package can
% be obtained from http://gaia.fdi.ucm.es/projects/texis/
%
% Although the TeXiS template itself is distributed under the 
% conditions of the LaTeX Project Public License
% (http://www.latex-project.org/lppl.txt), the manual content
% uses the CC-BY-SA license that stays that you are free:
%
%    - to share & to copy, distribute and transmit the work
%    - to remix and to adapt the work
%
% under the following conditions:
%
%    - Attribution: you must attribute the work in the manner
%      specified by the author or licensor (but not in any way that
%      suggests that they endorse you or your use of the work).
%    - Share Alike: if you alter, transform, or build upon this
%      work, you may distribute the resulting work only under the
%      same, similar or a compatible license.
%
% The complete license is available in
% http://creativecommons.org/licenses/by-sa/3.0/legalcode
%
%---------------------------------------------------------------------
%
% Fichero que contiene la configuración de la portada y de la 
% primera hoja del documento.
%
%---------------------------------------------------------------------


% Pueden configurarse todos los elementos del contenido de la portada
% utilizando comandos.

%%%%%%%%%%%%%%%%%%%%%%%%%%%%%%%%%%%%%%%%%%%%%%%%%%%%%%%%%%%%%%%%%%%%%%
% Título del documento:
% \tituloPortada{titulo}
% Nota:
% Si no se define se utiliza el del \titulo. Este comando permite
% cambiar el título de forma que se especifiquen dónde se quieren
% los retornos de carro cuando se utilizan fuentes grandes.
%%%%%%%%%%%%%%%%%%%%%%%%%%%%%%%%%%%%%%%%%%%%%%%%%%%%%%%%%%%%%%%%%%%%%%
\tituloPortada{%
Deep Reinforcement Learning en juegos
}

%%%%%%%%%%%%%%%%%%%%%%%%%%%%%%%%%%%%%%%%%%%%%%%%%%%%%%%%%%%%%%%%%%%%%%
% Autor del documento:
% \autorPortada{Nombre}
% Se utiliza en la portada y en el valor por defecto del
% primer subtítulo de la segunda portada.
%%%%%%%%%%%%%%%%%%%%%%%%%%%%%%%%%%%%%%%%%%%%%%%%%%%%%%%%%%%%%%%%%%%%%%
\autorPortada{Ricardo Arranz Janeiro\\
Lidia Concepción Echeverría\\
Juan Ramón Del Caño Vega\\
Francisco Ponce Belmonte\\
Juan Luis Romero Sánchez}

%%%%%%%%%%%%%%%%%%%%%%%%%%%%%%%%%%%%%%%%%%%%%%%%%%%%%%%%%%%%%%%%%%%%%%
% Fecha de publicación:
% \fechaPublicacion{Fecha}
% Puede ser vacío. Aparece en la última línea de ambas portadas
%%%%%%%%%%%%%%%%%%%%%%%%%%%%%%%%%%%%%%%%%%%%%%%%%%%%%%%%%%%%%%%%%%%%%%
\fechaPublicacion{\today}

%%%%%%%%%%%%%%%%%%%%%%%%%%%%%%%%%%%%%%%%%%%%%%%%%%%%%%%%%%%%%%%%%%%%%%
% Imagen de la portada (y escala)
% \imagenPortada{Fichero}
% \escalaImagenPortada{Numero}
% Si no se especifica, se utiliza la imagen TODO.pdf
%%%%%%%%%%%%%%%%%%%%%%%%%%%%%%%%%%%%%%%%%%%%%%%%%%%%%%%%%%%%%%%%%%%%%%
\imagenPortada{Imagenes/Vectorial/escudoUCM}
\escalaImagenPortada{.2}

%%%%%%%%%%%%%%%%%%%%%%%%%%%%%%%%%%%%%%%%%%%%%%%%%%%%%%%%%%%%%%%%%%%%%%
% Tipo de documento.
% \tipoDocumento{Tipo}
% Para el texto justo debajo del escudo.
% Si no se indica, se utiliza "TESIS DOCTORAL".
%%%%%%%%%%%%%%%%%%%%%%%%%%%%%%%%%%%%%%%%%%%%%%%%%%%%%%%%%%%%%%%%%%%%%%
\tipoDocumento{Trabajo de Fin de Grado}

%%%%%%%%%%%%%%%%%%%%%%%%%%%%%%%%%%%%%%%%%%%%%%%%%%%%%%%%%%%%%%%%%%%%%%
% Institución/departamento asociado al documento.
% \institucion{Nombre}
% Puede tener varias líneas. Se utiliza en las dos portadas.
% Si no se indica aparecerá vacío.
%%%%%%%%%%%%%%%%%%%%%%%%%%%%%%%%%%%%%%%%%%%%%%%%%%%%%%%%%%%%%%%%%%%%%%
\institucion{%
Grado en Ingeniería Informática\\[0.2em]
Facultad de Informática\\[0.2em]
Universidad Complutense de Madrid
}

%%%%%%%%%%%%%%%%%%%%%%%%%%%%%%%%%%%%%%%%%%%%%%%%%%%%%%%%%%%%%%%%%%%%%%
% Director del trabajo.
% \directorPortada{Nombre}
% Se utiliza para el valor por defecto del segundo subtítulo, donde
% se indica quién es el director del trabajo.
% Si se fuerza un subtítulo distinto, no hace falta definirlo.
%%%%%%%%%%%%%%%%%%%%%%%%%%%%%%%%%%%%%%%%%%%%%%%%%%%%%%%%%%%%%%%%%%%%%%
\directorPortada{Antonio Alejandro Sánchez Ruiz-Granados}

%%%%%%%%%%%%%%%%%%%%%%%%%%%%%%%%%%%%%%%%%%%%%%%%%%%%%%%%%%%%%%%%%%%%%%
% Texto del primer subtítulo de la segunda portada.
% \textoPrimerSubtituloPortada{Texto}
% Para configurar el primer "texto libre" de la segunda portada.
% Si no se especifica se indica "Memoria que presenta para optar al
% título de Doctor en Informática" seguido del \autorPortada.
%%%%%%%%%%%%%%%%%%%%%%%%%%%%%%%%%%%%%%%%%%%%%%%%%%%%%%%%%%%%%%%%%%%%%%
\textoPrimerSubtituloPortada{%
\textbf{Trabajo de Fin de Grado en Ingeniería Informática}  \\ [0.3em]
\textbf{Departamento de Ingeniería de Software e Inteligencia Artificial} \\ [0.3em]
}

%%%%%%%%%%%%%%%%%%%%%%%%%%%%%%%%%%%%%%%%%%%%%%%%%%%%%%%%%%%%%%%%%%%%%%
% Texto del segundo subtítulo de la segunda portada.
% \textoSegundoSubtituloPortada{Texto}
% Para configurar el segundo "texto libre" de la segunda portada.
% Si no se especifica se indica "Dirigida por el Doctor" seguido
% del \directorPortada.
%%%%%%%%%%%%%%%%%%%%%%%%%%%%%%%%%%%%%%%%%%%%%%%%%%%%%%%%%%%%%%%%%%%%%%
\textoSegundoSubtituloPortada{%
\textbf{Convocatoria: }\textit{Junio \the\year} \\ [0.2em]
\textbf{Calificación: }\textit{}
}

%%%%%%%%%%%%%%%%%%%%%%%%%%%%%%%%%%%%%%%%%%%%%%%%%%%%%%%%%%%%%%%%%%%%%%
% \explicacionDobleCara
% Si se utiliza, se aclara que el documento está preparado para la
% impresión a doble cara.
%%%%%%%%%%%%%%%%%%%%%%%%%%%%%%%%%%%%%%%%%%%%%%%%%%%%%%%%%%%%%%%%%%%%%%
\explicacionDobleCara

%%%%%%%%%%%%%%%%%%%%%%%%%%%%%%%%%%%%%%%%%%%%%%%%%%%%%%%%%%%%%%%%%%%%%%
% \isbn
% Si se utiliza, aparecerá el ISBN detrás de la segunda portada.
%%%%%%%%%%%%%%%%%%%%%%%%%%%%%%%%%%%%%%%%%%%%%%%%%%%%%%%%%%%%%%%%%%%%%%
%\isbn{978-84-692-7109-4}


%%%%%%%%%%%%%%%%%%%%%%%%%%%%%%%%%%%%%%%%%%%%%%%%%%%%%%%%%%%%%%%%%%%%%%
% \copyrightInfo
% Si se utiliza, aparecerá información de los derechos de copyright
% detrás de la segunda portada.
%%%%%%%%%%%%%%%%%%%%%%%%%%%%%%%%%%%%%%%%%%%%%%%%%%%%%%%%%%%%%%%%%%%%%%
\copyrightInfo{\autor}


%%
%% Creamos las portadas
%%
\makeCover

% Variable local para emacs, para que encuentre el fichero
% maestro de compilación
%%%
%%% Local Variables:
%%% mode: latex
%%% TeX-master: "../Tesis.tex"
%%% End:

\chapter*{Autorización de difusión}

   
El abajo firmante, matriculado en el Máster en Ingeniería en Informática de la Facultad de Informática, autoriza a la Universidad Complutense de Madrid (UCM) a difundir y utilizar con fines académicos, no comerciales y mencionando expresamente a su autor el presente Trabajo Fin de Máster: ``TITULO DEL TRABAJO'', realizado durante el curso académico CURSO bajo la dirección de DIRECTORES en el Departamento de XXXXXXXXXXXXXXXXXXXXXXXX, y a la Biblioteca de la UCM a depositarlo en el Archivo Institucional E-Prints Complutense con el objeto de incrementar la difusión, uso e impacto del trabajo en Internet y garantizar su preservación y acceso a largo plazo.

\vspace{5cm}

% +--------------------------------------------------------------------+
% | On the line below, replace "Enter Your Name" with your name
% | Use the same form of your name as it appears on your title page.
% | Use mixed case, for example, Lori Goetsch.
% +--------------------------------------------------------------------+
\begin{center}
	\large Nombre Del Alumno\\
	
	\vspace{0.5cm}
	
	% +--------------------------------------------------------------------+
	% | On the line below, replace Fecha
	% |
	% +--------------------------------------------------------------------+
	
	\today\\
	
\end{center}

% +--------------------------------------------------------------------+
% | Dedication Page (Optional)
% +--------------------------------------------------------------------+

\chapter*{Dedicatoria}


Texto de la dedicatoria...
% +--------------------------------------------------------------------+
% | Acknowledgements Page (Optional)                                   |
% +--------------------------------------------------------------------+

\chapter*{Agradecimientos}

Texto de los agradecimientos












\chapter*{Resumen}

Resumen en español del trabajo


\section*{Palabras clave}
   
\noindent Máximo 10 palabras clave separadas por comas

   



\begin{otherlanguage}{english}
\chapter*{Abstract}

In this project we will study the Deep Reinforcement Learning field in order to achieve an stable application for classic control problems. To do this we will investigate its fundamentals: Reinforcement Learning and Neural Networks, learning which are their strengths and weaknesses. Finally, we will merge both to progressivly improve our agent's performance and stability.

In order to gain a better insight we will personally implement the agents and algorithms. All of this will be tested through the popular framework OpenAI Gym. 

This project's source code can be found in the repository:

\url{https://github.com/delcanovega/TFG-DRL}

\section*{Keywords}

\begin{itemize}
    \item Reinforcement Learning
    \item Q-Learning
    \item Markov decision process
    \item Neural Networks
    \item Deep Reinforcement Learning
    \item DeepMind 
    \item OpenAI
\end{itemize}
   




% Si el trabajo se escribe en inglés, comentar esta línea y descomentar
% otra igual que hay justo antes de \end{document}
\end{otherlanguage}

\ifx\generatoc\undefined
\else
%---------------------------------------------------------------------
%
%                          TeXiS_toc.tex
%
%---------------------------------------------------------------------
%
% TeXiS_toc.tex
% Copyright 2009 Marco Antonio Gomez-Martin, Pedro Pablo Gomez-Martin
%
% This file belongs to TeXiS, a LaTeX template for writting
% Thesis and other documents. The complete last TeXiS package can
% be obtained from http://gaia.fdi.ucm.es/projects/texis/
%
% This work may be distributed and/or modified under the
% conditions of the LaTeX Project Public License, either version 1.3
% of this license or (at your option) any later version.
% The latest version of this license is in
%   http://www.latex-project.org/lppl.txt
% and version 1.3 or later is part of all distributions of LaTeX
% version 2005/12/01 or later.
%
% This work has the LPPL maintenance status `maintained'.
% 
% The Current Maintainers of this work are Marco Antonio Gomez-Martin
% and Pedro Pablo Gomez-Martin
%
%---------------------------------------------------------------------
%
% Contiene  los  comandos  para  generar los  índices  del  documento,
% entendiendo por índices las tablas de contenidos.
%
% Genera  el  índice normal  ("tabla  de  contenidos"),  el índice  de
% figuras y el de tablas. También  crea "marcadores" en el caso de que
% se esté compilando con pdflatex para que aparezcan en el PDF.
%
%---------------------------------------------------------------------


% Primero un poquito de configuración...


% Pedimos que inserte todos los epígrafes hasta el nivel \subsection en
% la tabla de contenidos.
\setcounter{tocdepth}{2} 

% Le  pedimos  que nos  numere  todos  los  epígrafes hasta  el  nivel
% \subsubsection en el cuerpo del documento.
\setcounter{secnumdepth}{3} 


% Creamos los diferentes índices.

% Lo primero un  poco de trabajo en los marcadores  del PDF. No quiero
% que  salga una  entrada  por cada  índice  a nivel  0...  si no  que
% aparezca un marcador "Índices", que  tenga dentro los otros tipos de
% índices.  Total, que creamos el marcador "Índices".
% Antes de  la creación  de los índices,  se añaden los  marcadores de
% nivel 1.

\ifpdf
   \pdfbookmark{Índices}{indices}
\fi

% Tabla de contenidos.
%
% La  inclusión  de '\tableofcontents'  significa  que  en la  primera
% pasada  de  LaTeX  se  crea   un  fichero  con  extensión  .toc  con
% información sobre la tabla de contenidos (es conceptualmente similar
% al  .bbl de  BibTeX, creo).  En la  segunda ejecución  de  LaTeX ese
% documento se utiliza para  generar la verdadera página de contenidos
% usando la  información sobre los  capítulos y demás guardadas  en el
% .toc
\ifpdf
   \pdfbookmark[1]{Tabla de Contenidos}{tabla de contenidos}
\fi

\cabeceraEspecial{\'Indice}

\tableofcontents

\newpage 

% Índice de figuras
%
% La idea es semejante que para  el .toc del índice, pero ahora se usa
% extensión .lof (List Of Figures) con la información de las figuras.

\ifpdf
   \pdfbookmark[1]{Índice de figuras}{indice de figuras}
\fi

\cabeceraEspecial{\'Indice de figuras}

\listoffigures

\newpage

% Índice de tablas
% Como antes, pero ahora .lot (List Of Tables)

\ifpdf
   \pdfbookmark[1]{Índice de tablas}{indice de tablas}
\fi

\cabeceraEspecial{\'Indice de tablas}

\listoftables

\newpage

% Variable local para emacs, para  que encuentre el fichero maestro de
% compilación y funcionen mejor algunas teclas rápidas de AucTeX

%%%
%%% Local Variables:
%%% mode: latex
%%% TeX-master: "../Tesis.tex"
%%% End:

\fi

% Marcamos el  comienzo de  los capítulos (para  la numeración  de las
% páginas) y ponemos la cabecera normal
\mainmatter

\pagestyle{fancy}
\restauraCabecera

%%%%%%%%%%%%%%%%%%%%%%%%%%%%%%%%%%%%%%%%%%%%%%%%%%%%%%%%%%%%%%%%%%%%%%%%%%%
% Si el TFM se escribe en ingles, comentar las siguientes líneas 
% porque no hace falta incluir nuevamente la Introducción en inglés
\begin{otherlanguage}{english}
\chapter{Introduction}

\chapterquote{The young man or the young woman must possess or teach himself, train himself, in infinite patience, which is to try and to try and to try until it comes right}{William Faulkner}


\section{Motivation}

\textbf{Artificial Intelligence} is one of the computation fields that most interest has generated, both among experts and general public, who is more interested in the leisure side of it. Been able to grant a machine the hability to reason and perform functions only associated with the human intellect has always been considered Sci-Fi. And nonetheless this very idea has experienced a huge technological leap in the last years.

To better understand this concept from its roots, we need to reference the \textbf{Turing Test}, formulated by Alan Turing \citep{Turing1950-TURCMA}, which consists in performing a series of questions to a machine. The test is considered passed if the evaluator cannot discern if the answers were given by a human or a machine.

Trying to define Artificial Intelligence quickly leads us to concepts like \textit{thought process} or \textit{reasoning}, which end up driving to more complex ones like \textit{behaviour}. From this ideas we can find other definitions, classified in the matrix \ref{fig:tabla_IA_EN}.

\figura{Bitmap/Introduccion/tabla_IA_EN}{width=1\textwidth}{fig:tabla_IA_EN}%
       {Artificial Intelligence definitions, \citet{Russell:2009:AIM:1671238}}

% TODO JCA TRADUCIR UNA VEZ AMPLIADO
En base a estas clasificaciones podríamos diferenciar dos corrientes de interpretación:
\begin{itemize}
    \item Una visión empírica (columna izquierda) con el ser humano como centro de la investigación.
    \item Una visión racionalista (columna derecha) que involucra una combinación de matemáticas e ingeniería.
\end{itemize}

Lots of experts have studied both approaches in different ways. We will focus in the empiric approach, pursuing the goal of our \textit{agent} being able to take the ``right choices'' with its available knowledge. In order to achieve this, we will study an Artificial Intelligence's field in particular, called Reinforcement Learning. We will also combine this field with Neural Networks, resulting in Deep Reinforcement Learning.


\section{Goals}

\begin{enumerate}
    \item Understand why is Reinforcement Learning different from other Machine Learning methods, and in which situations it can be applied.
    \item Study the fundamentals of Reinforcement Learning, understanding its components, implementations and limits.
    \item Test what we have learned with practical simulations, on which we will implement Reinforcement Learning algorithms, and study the results.
    \item Dive into the Deep Learning field, where we will see the fundamentals of Neural Networks.
    \item Learn how is it possible to combine Neural Networks with Reinforcement Learning techniques, trying to avoid the limitations of both. It will be a journey where, step by step, we will find solutions to the problems that arise, until we come with a stable Deep Reinforcement Learning solution.
    \item After every milestone we will evaluate the results, contrasting if our solutions improve stability and performance.
\end{enumerate}


\section{Structure of the memory}

Our project intercalates theorical chapters with practical applications of what we have seen, resulting in two big blocks: One about Reinforcement Learning and other about Deep Reinforcement Learning.

\begin{itemize}
    \item \textbf{Chapter 1, Introduction.} Motivation and goals of our proyect.
    \item \textbf{Chapter 2, Reinforcement Learning.} The chapter begins with a comparison between different Machine Learning techniques. Afterwards, the needed Reinforcement Learning theorical background is provided. Finally, we will thoroughly explain Q-Learning, one of the most commonly used Reinforcement Learning algorithms.
    \item \textbf{Chapter 3, Q-Learning in action.} We will introduce OpenAI Gym, the framework used in our tests. Later we will apply the acquired Reinforcement Learning knowledge into CartPole, an environment that will allow us to measure results and experience Reinforcement Learning's limitations.
    \item \textbf{Chapter 4, Neural Networks and Q-Learning.} We will give the necessary theorical background to understand Neural Networks, followed by how is possible to combine them with Reinforcement Learning techniques, obtaining the so called Deep Q-Networks.
    \item \textbf{Chapter 5, DQNs in action.} We will solve CartPole again, this time applying the new learned approaches. Then we will face MountainCar, a new and challenging environment for our agent.
    \item \textbf{Chapter 6, Conlusions.} Summary of everything we have achieved so far. Lessons learned, highlights and future work.    
\end{itemize}

\end{otherlanguage}
\addtocounter{chapter}{-1} 
%%%%%%%%%%%%%%%%%%%%%%%%%%%%%%%%%%%%%%%%%%%%%%%%%%%%%%%%%%%%%%%%%%%%%%%%%%%

\chapter{Introducción}
\label{cap:introduccion}

\chapterquote{Los jóvenes deben enseñarse a sí mismos, entrenarse a sí mismos, con infinita paciencia, intentarlo una y otra y otra vez hasta que salga bien}{William Faulkner}


\section{Motivación}

La \textbf{inteligencia artificial} es una de las ramas de la computación que más interés ha generado, tanto entre expertos de la materia como en otro tipo de público, más interesado en la parte lúdica de este concepto. Dotar a una máquina de la capacidad de realizar funciones asociadas sólo al intelecto humano ha sido siempre considerado ciencia ficción. Y sin embargo esta misma idea ha supuesto un enorme avance tecnológico en los últimos años.

Para conocer mejor este concepto desde sus orígenes, es necesario hacer referencia al \textbf{test de Turing}, propuesto por Alan \citet{Turing1950-TURCMA}. Éste consiste en realizar una serie de preguntas a un ente y el test se considerará superado si el interrogador no es capaz de discernir si las respuestas provienen de una máquina o una persona.

Intentar definir la inteligencia artificial nos lleva directamente a conceptos como el \textit{proceso del pensamiento} o el \textit{razonamiento}, los cuales terminan por conducir a otros más complejos como es el \textit{comportamiento}. A partir de estas ideas podemos encontrar otras definiciones, clasificadas en la matriz~\ref{fig:tabla_IA}.

\figura{Bitmap/Introduccion/tabla_IA}{width=1\textwidth}{fig:tabla_IA}%
       {Definiciones de inteligencia artificial, \citet{Russell:2009:AIM:1671238}}

En base a estas clasificaciones podríamos diferenciar dos corrientes de interpretación:
\begin{itemize}
    \item Una visión empírica (columna izquierda) con el ser humano como centro de la investigación. Involucra principalmente observaciones e hipótesis sobre cómo debería comportarse un humano. Esta vertiente tiene cierta relevancia a día de hoy, sobre todo gracias al auge de proyectos como el coche autónomo. Estos agentes autónomos, puestos en una situación límite, podrían verse obligados a decidir entre dos opciones que pongan en peligro vidas humanas. En este caso, deberíamos basar nuestra respuesta en qué elegiría un conductor real.
    \item Una visión racionalista (columna derecha), que implica una combinación de matemáticas e ingeniería. En esta vertiente se engloban proyectos como los asistentes de voz o los robots de Boston Dynamics. Ninguno de ellos necesita valorar las órdenes e información que se les provee a un nivel ``humano''. Los asistentes no necesitan ser asertivos, sólo necesitan ser capaces de procesar la información de forma correcta para actuar consecuentemente, mientras que los robots sólo deben ser capaces de aprender a adaptarse a cualquier terreno con el fin de realizar la función que se les ha encomendado.
\end{itemize}

Multitud de expertos han abordado ambos acercamientos de distintas formas. Nosotros buscaremos que nuestro \textit{agente} tome las ``decisiones correctas'' en función del conocimiento que posea. En particular buscaremos que un agente sin ningún conocimiento previo sea capaz de aprender a realizar tareas sencillas mediante la interacción constante con el entorno, quien le proporcionará nuevas experiencias de las que extraer conocimiento.

Esta forma de aprendizaje es un campo de la inteligencia artificial llamado \textbf{aprendizaje por refuerzo}. También investigaremos una combinación del mismo con \textbf{redes neuronales}, resultando en el llamado \textbf{aprendizaje por refuerzo profundo}. La popularidad de estas técnicas no ha parado de crecer en los últimos años. Koray Kavukcuoglu, director de investigación en Deepmind, explica su potencial de la siguiente forma:

\begin{quote}
    El aprendizaje por refuerzo es un sistema muy general para aprender a tomar decisiones secuenciales. Por otra parte, el aprendizaje profundo es el mejor conjunto de algoritmos disponibles para aprender representaciones. Combinar estos dos modelos diferentes es la mejor opción que tenemos disponible para lograr buenas representaciones de estados en tareas complejas, no sólo para resolver juegos sencillos si no también complicados problemas reales.
\end{quote}

En definitiva, el potencial y las posibilidades de esta metodología convierten al aprendizaje por refuerzo profundo en un campo muy interesante, que puede que nos lleve un paso más cerca al mundo de la inteligencia artificial general.


\section{Objetivos}

\begin{enumerate}
    \item Comprender en qué es el aprendizaje por refuerzo distinto a otros métodos de aprendizaje automático, y en qué situaciones puede ser usado.
    \item Estudiar los fundamentos del aprendizaje por refuerzo, entendiendo sus características, componentes y limitaciones.
    \item Poner a prueba lo aprendido con simulaciones prácticas, en las que implementemos algoritmos de aprendizaje por refuerzo y estudiemos sus resultados.
    \item Adentrarnos en el campo del aprendizaje profundo, donde veremos los fundamentos de las redes neuronales.
    \item Estudiar cómo es posible combinar las redes neuronales y el aprendizaje por refuerzo con el objetivo de sortear las limitaciones de ambos. Será un camino en el que, paso a paso, encontraremos soluciones a los problemas que surjan hasta lograr un modelo estable de aprendizaje por refuerzo profundo.
    \item En cada hito del camino evaluaremos los resultados obtenidos, para comprobar que las soluciones mejoran en rendimiento y estabilidad.
\end{enumerate}


\section{Estructura de la memoria}

Nuestro trabajo intercala capítulos teóricos con aplicaciones prácticas de lo visto en dichos capítulos, resultando en dos bloques diferenciables: el primero sobre aprendizaje por refuerzo y el segundo sobre aprendizaje por refuerzo profundo.

\begin{itemize}
    \item \textbf{Capítulo 1, Introducción.} Motivación y objetivos de nuestro proyecto.
    \item \textbf{Capítulo 2, Aprendizaje por Refuerzo.} El capítulo comienza con una comparación del aprendizaje por refuerzo con otros métodos de aprendizaje automático. Después, se proporciona toda la base teórica necesaria para comprender el aprendizaje por refuerzo. Para terminar, explicaremos en profundidad Q-Learning, el algoritmo que utilizaremos en nuestras pruebas.
    \item \textbf{Capítulo 3, Q-Learning en acción.} Introduciremos OpenAI Gym, la herramienta utilizada durante nuestras pruebas. Aplicaremos los conocimientos de aprendizaje por refuerzo en CartPole, un problema que nos permitirá evaluar resultados y limitaciones.
    \item \textbf{Capítulo 4, Redes Neuronales y Q-Learning.} Proporcionaremos el marco teórico necesario para comprender las redes neuronales. Después explicaremos cómo es posible aplicarlas a técnicas de aprendizaje por refuerzo, obteniendo las DQN.
    \item \textbf{Capítulo 5, DQNs en acción.} Volveremos a resolver CartPole y nos enfrentaremos a MountainCar, un problema que presenta nuevos retos para nuestro agente.
    \item \textbf{Capítulo 6, Conclusiones.} Síntesis de todo lo aprendido durante el camino. Puntos clave de las distintas etapas, limitaciones y oportunidades de cara al futuro.
    
\end{itemize}

\chapter{Estado de la Cuestión}
\label{cap:estadoDeLaCuestion}

En el estado de la cuestión es donde aparecen gran parte de las referencias bibliográficas del trabajo. Una de las formas más cómodas de gestionar la bibliografía en {\LaTeX} es utilizando \textbf{bibtex}. Las entradas bibliográficas deben estar en un fichero con extensión \textit{.bib} (con esta plantilla se proporcionan 3, dos de los cuales están vacíos). Cada entrada bibliográfica tiene una clave que permite referenciarla desde cualquier parte del texto con los siguiente comandos:

\begin{itemize}
\item Referencia bibliografica con cite: \cite{ldesc2e}
\item Referencia bibliográfica con citep: \citep{notsoshort}
\item Referencia bibliográfica con citet: \citet{latexAPrimer}
\end{itemize}

Es posible citar más de una fuente, como por ejemplo \citep{latexCompanion,LaTeXLamport,texKnuth}

Después, latex se ocupa de rellenar la sección de bibliografía con las entradas \textbf{que hayan sido citadas} (es decir, no con todas las entradas que hay en el .bib, sino sólo con aquellas que se hayan citado en alguna parte del texto).

Bibtex es un programa separado de latex, pdflatex o cualquier otra cosa que se use para compilar los .tex, de manera que para que se rellene correctamente la sección de bibliografía es necesario compilar primero el trabajo (a veces es necesario compilarlo dos veces), compilar después con bibtex, y volver a compilar otra vez el trabajo (de nuevo, puede ser necesario compilarlo dos veces). 

%\include{Capitulos/Capitulo3}
%\include{Capitulos/Capitulo4}
%\include{Capitulos/Capitulo5}
\chapter{Conclusiones y Trabajo Futuro}
\label{cap:conclusiones}

Conclusiones del trabajo y líneas de trabajo futuro.




%%%%%%%%%%%%%%%%%%%%%%%%%%%%%%%%%%%%%%%%%%%%%%%%%%%%%%%%%%%%%%%%%%%%%%%%%%%
% Si el TFM se escribe en inglés, comentar las siguientes líneas 
% porque no es necesario incluir nuevamente las Conclusiones en inglés
\setcounter{chapter}{\thechapter-1} 
\begin{otherlanguage}{english}
\chapter{Conclusions and future work}
\label{cap:conclusions}

\chapterquote{We are what we do repeatedly. Excellence, then, is not an act, it is a habit}{Aristotle}

\section{Conclusions}

This project has served us as an extensive study about the origins and evolution of one of the most explored technologies within the Artificial Inteligence: Machine Learning, concretely the Deep Learning field.

When we started this project, we just had a slight idea about this term's complexity involvements or the utilities's scope that can be reached with this technology. But fortunately, our goal was clear and our research was aimed for a single goal: To make it possible for an AI to play a videogame on its own.

We began by studying the basics of Machine Learning, taking as our starting point the technique of Reinforcement Learning. This one fitted perfectly our project, since when it comes to videogames, it is easy to find rewards in that environment. Investigating about it, we discovered the Q-Learning method, which seemed to be the best suited to our case. It allowed us to make decisions at every moment, maintaining a reliable representation of any possible environment, in a way that the AI could play normally regardless of the game.

After the previous study, it was time to do our first tests and learn how to use the environment. We started with a simple game: CartPole, available in the OpenAI Gym library. The environment was simple and the actions to be taken by the agent were reduced. The results were good, but we needed to go one step further. The model used in Q-Learning could give us a lot of problems in larger environments, even though the idea was good at first.

Almost simultaneously, we started looking for other ways to develop our project, finally opting for Neural Networks. We verified with several examples that these were compatible with our problem, concluding that it was a good point to advance. Exploring beyond what we already knew about them, we found the perfect way to combine the idea and results of Q-Learning with the speed and comfort that Neural Networks provide: DQNs.

Testing DQNs was somehow more tedious to perform, as it was something totally new for us and required extra research time. The results were not good at the beginning, but after stabilizing them with a second network and adding a memory for them to learn from their own mistakes, we achieve the correct learning of our agent, giving us better results.

The last step we reached in our project was to transfer everything we learned and accomplished in CartPole to another environment to check that the agent was still learning correctly. This time it was the MountainCar game, also available in the OpenAI Gym library. After several adaptations and some new minor components, we got successful results, so we concluded that we had managed to develop an AI capable of playing.

What has been achieved in this project has shown us the amount of possibilities that exist when it comes to solving a problem of this type. The research and time invested, has led us to understand the work involved in building an AI, even having a small initial base on which to stand, and the amount of information we still have to learn about this topic.

\section{Achievement of objectives}

At the beginning of this document, we proposed some objectives to explore during the course of this project, which have helped us to organize the work and its presentation in this document:

\begin{enumerate}
        \item The first objective was to understand what Reinforcement Learning is and how it differs from other branches of Machine Learning. Throughout the chapter \ref{cap:reinforcementLearning} we saw an explanation of this concept, the ideas on which it was based and their differences from other branches of Machine Learning, aswell as why we chose it as a starting point. Continuing with this idea, we explain all the elements that compose it and the way in which they relate to each other.
        \item Having seen how all the elements of Reinforcement Learning are related, we decided to test these concepts during the chapter \ref{cap:q-learning}, in which we explained the OpenAI Gym library. We chose the CartPole environment to carry out our experiments, managing to overtake the objective set for this environment and experimenting the problems involved in the implementation of this algorithm, in reference to the exponential increase of the Q-table and the complexity involved in keeping it completely updated, and achieving the correct learning of our agent.
        \item Another objective was to get into the field of Deep Learning and the fundamentals of Neural Networks. We discussed it throughout the chapter \ref{cap:deepLearning}, during which we explained the principles of Neural Networks, their composition, structure and behaviour. This way, we played with different problems of classification and regression that served us to have a first contact that later would help us to manage problems within the scope of Deep Reinforcement Learning.
        \item Our last goal was to combine Neural Networks and Reinforcement Learning, in a way that we could get the benefits of both, as well as finding solutions to their different issues, as we saw during the chapter \ref{cap:dqnEnAccion}. We studied step by step how to combine the two scopes, first replacing the Q-table of Reinforcement Learning with a Neuronal Network, which we saw that it didn't give our agent enough learning skills. After that, we introduced the concept of experience replay and the use of two networks, thanks to which we managed to stabilize learning and convergence.
        \item Having fulfilled all the objectives, and in view of the obtained results, we have been able to see the limitations of this field of study as of today, which is quickly evolving with new techniques every day. There definetly are lots of ways to follow if you are eager and interested in it.
\end{enumerate}

\section{Future work}

The project was aimed for teaching an AI to play and, although it has learned successfully, it has hardly been tested in a couple of environments with few variables. The path of our research could be the right one, but without further progress, or testing in other environments, we cannot be sure of that.

That said, if we could continue our project, we would focus on continuing with more tests on different games. After verifying that our AI is adaptable, we would start by adding more options to the possible actions that the player can perform, updating it to make it scalable, in case it already wasn't.

From there, work could follow several paths. On one hand we could experiment with other types of algorithms, looking for a greater efficiency. Proof of this is ``Baselines'', the OpenAI repository that we have already mentioned \citep{baselines}. Another option would be to go a step further and start interpreting images directly as input, so as not to depend on observations in the form of a variable as we have done so far.

DeepMind already demonstrated this a few years ago. In \citet{mnih2013playing} they proved that their AI was able to learn to play some Atari games at a professional level, all of this only through the pixels of each image. To achieve this we would have to specialize in Neural Convolutional Networks, which would be quite a challenge. Not only because they are more complex, but also because, since they need more resources when it comes to training, the system must be much more precise to be viable.

But every effort pays off: Once you have an AI capable of recognizing images, the possibilities soar. By not being tied to observations provided through variables, it is no longer necessary to rely on frameworks or tedious manual implementations to perform tests. One could work at a much more ``real'' level, in the sense that perhaps it would no longer be necessary to describe a problem perfectly, with all its physical laws, in order to work on it. Perhaps it would simply be necessary to provide the agent with a series of images so that it would be able to learn about them, understanding behaviors and patterns and learning in a much more humane way. However, this idea is still a long way off for us.

\end{otherlanguage}
%%%%%%%%%%%%%%%%%%%%%%%%%%%%%%%%%%%%%%%%%%%%%%%%%%%%%%%%%%%%%%%%%%%%%%%%%%%


% Apéndices
\appendix

\chapter{Título}
\label{Appendix:Key1}

En continuación describiremos cronológicamente el desarrollo de prototipos durante el ejercicio de MountainCar:

\section{Prototipo 0}
Este fue nuestro primer intento en el que simplemente probamos a entrenar directamente con la recompensa del entorno.

\figura{Bitmap/ApendiceA/Primer_paso.PNG}{width=1\textwidth}{fig:mountaincar_00}%
       {Prototipo 0}

En las primeras ejecuciones de 200 y 500 episodios no conseguimos ningún resultado \ref{fig:mountaincar_00} así que probamos con 1500 y obtuvimos nuestro primer paso, era posible resolver el problema.

\figura{Bitmap/ApendiceA/laFuerzaBrutaSiempreGana.jpg}{width=1\textwidth}{fig:mountaincar_001}%
{Prototipo 00}

En este momento iniciamos nuestra investigación y análisis del entorno. Identificamos como posible causa la recompensa del entorno, y decidimos cambiarla por otra modificada para favorecer el aprendizaje.

\section{Prototipo 1}
Desarrollamos nuestra primera función de recompensa, ``ourReward'', que recibe un estado y devuelve la recompensa en función de la posición absoluta añadiendo una bonificación de +1 en caso de que la posición sea mayor a 0.5 y la velocidad sea positiva. Inicialmente planteamos este prototipo con un factor de descuento \ref{fig:mountaincar_01} que para penalizar las soluciones lentas frente a las rápidas.

\figura{Bitmap/ApendiceA/CapturaCambiandoRewardAPosicion.PNG}{width=1\textwidth}{fig:mountaincar_01}%
{Prototipo 1}

Más tarde nos percatamos de que este factor de descuento entraba en conflicto con nuestro principio de basar la recompensa enteramente en el estado. De todas formas, pudimos comprobar un factor de penalización menor de 0.997 era excesivo. Y por otro lado caímos en la cuenta de que la posición absoluta no reflejaba el comportamiento que queríamos premiar. Nuestro objetivo era premiar al agente por alejar el avatar del centro, el punto más bajo entre las montañas, pero dado que este centro se encuentra en la posición -0.5, estábamos premiando la distancia respecto a la mitad de la montaña derecha. Lo cual claramente conlleva a evitar subir la ladera derecha.





\section{Prototipo 2}
Este prototipo fue desarrollado al mismo tiempo que el Prototipo 1 y más tarde descartado por las mismas razones.
En este caso basamos la recompensa en la velocidad absoluta, añadiendo una bonificación si la posición absoluta es mayor que 0.5. Como ya explicamos antes, el centro no es 0 sino -0.5 por lo que esa bonificación se aplicará siempre que no alcance la mitad de la ladera derecha. Además, lo insignificantemente pequeña que es la velocidad comparada con el valor de la bonificación, hace que reciba siempre la misma recompensa. Por otro lado, al igual que Prototipo 1, esta cuenta con un factor de descuento.

\figura{Bitmap/ApendiceA/FigureMain2.png}{width=1\textwidth}{fig:mountaincar_02}%
{Prototipo 2}

Observando más detenidamente los resultados y el comportamiento del agente son percatamos de que esta recompensa se ajustaba bastante al comportamiento de la recompensa original del problema, la razón de esto es que las recompensas intermedias son invariantes durante toda la ejecución.

\section{Prototipo 3}
Este prototipo fue creado junto a Prototipo 1 y Prototipo 2, pero fuimos actualizando lo a medida que íbamos detectando fallos en los dos anteriores y puntos que mejorar en los siguiente.
Comenzó como otro prototipo basado en la Velocidad absoluta, sin factor de penalización y con una bonificación de +10 en caso de superar su mayor posición centro, de esta forma pretendíamos que al quedarse sin impulso hacia un lado buscara superar dicha altura por la ladera contraria y así favorecer el balanceo \ref{fig:mountaincar_03}.

\figura{Bitmap/ApendiceA/Main_3SinPenalizacionAlcanzadoIndistin.png}{width=1\textwidth}{fig:mountaincar_030}%
{Prototipo 30}

En este punto nos dimos cuenta de que, aunque en ocasiones si llegaba a alcanzar la meta, estas ejecuciones pasaban inadvertidas para el aprendizaje dado que en estos casos la recompensa no era especialmente mayor que las demás partidas. Incluso en otros prototipos que desarrollábamos al mismo tiempo, llegamos a ver ejemplos que priorizaban seguir con el balanceo y así ganar más puntos.
Por ello decidimos dar una recompensa desproporcionada a las partidas que alcanzaran la meta \ref{fig:mountaincar_031}.

\figura{Bitmap/ApendiceA/Main3SinPenalAlcanzaDisting.png}{width=1\textwidth}{fig:mountaincar_031}%
{Prototipo 31}

Tras las observaciones del Prototipo 2, optamos por dar más visibilidad a la velocidad, esto lo haríamos en primera instancia multiplicando la por 10.

\figura{Bitmap/ApendiceA/main3_rewardX10.png}{width=1\textwidth}{fig:mountaincar_032}%
{Prototipo 32}

Tras los buenos resultados del Prototipo 4, probamos a aplicar el mismo método en el Prototipo 3, pero esta vez le daríamos muchos más episodios para aprender \ref{fig:moutaincar_033}

\figura{Bitmap/ApendiceA/1500itMain3RewardX100.png}{width=1\textwidth}{fig:mountaincar_033}%
{Prototipo 33}

\section{Prototipo 4}
Desarrollado poco después de comenzar con las pruebas del Prototipo 3 e implementando las correcciones de los errores de los prototipos 1 y 2.
La función de recompensa consiste en la velocidad absoluta por 100, para así darle más visibilidad a la velocidad frente a las bonificaciones. Por otro lado, y al igual que Prototipo 3, este premia con un +10 cada vez que el avatar supera su posición más alejada del centro. Además, este fue nuestro primer prototipo que no tenía el factor de descuento que invalidaba casi todas las pruebas anteriores.

\figura{Bitmap/ApendiceA/Main_4SolorecompSiGTmaxMas0_1.png}{width=1\textwidth}{fig:mountaincar_04}%
{Prototipo 4}

\section{Prototipo 5}
Viendo el poco éxito que habíamos tenido hasta el momento, decidimos regresar a los orígenes, para ver si con pequeñas modificaciones de la recompensa original conseguíamos alguna mejora notable.
La recompensa volvía a ser -1 para todo estado no final, pero si supera su posición más alejada del centro en 0.1, la recompensa pasa a ser 1. De esta forma planteamos una serie de recompensas intermedias como los prototipos anteriores, pero reduciendo el número de estas. Esta estrategia premia solo superar las metas y penaliza por el paso del tiempo.

\figura{Bitmap/ApendiceA/main5PremiarGTmaxMas0_1else-1.png}{width=1\textwidth}{fig:mountaincar_05}%
{Prototipo 5}

En este momento fue cuando caímos en la cuenta de que todos los bonus, que habíamos estado basando en la posición máxima alcanzada, eran erróneos, dado que la posición máxima alcanzada varía según el momento de la partida, es decir, la recompensa no se basaba únicamente en el estado y por tanto no era valida.

\section{Prototipo 6}
Ante las últimas revelaciones optamos por volver a modelos mucho más simples. Comenzando como una nueva variante del Prototipo 3, basado en la velocidad absoluta y dando una recompensa exageradamente alta para reforzar los caminos correctos. Además de empotrar el ``Doble Agente'' en este ejercicio.

\figura{Bitmap/ApendiceA/main3_1_prueba3ExageraRecompensa.png}{width=1\textwidth}{fig:mountaincar_06}%
{Prototipo 6}

Comenzamos a explorar las posibles causas de las bajadas en el aprendizaje que se ven en algunas de las gráficas.
Hasta el momento siempre habíamos utilizado el Optimizador Adam\ref{fig:mountaincar_061} pero hicimos algunas pruebas con Adadelta\ref{fig:mountaincar_062}.

\figura{Bitmap/ApendiceA/main_3_2FinalSinBonusAdam200eps.png}{width=1\textwidth}{fig:mountaincar_061}%
{RELLENAR}

\figura{Bitmap/ApendiceA/main_3_2FinalSinBonusAdadelta200Mejor.png}{width=1\textwidth}{fig:mountaincar_062}%
{RELLENAR}

\section{Prototipo 7}
Ahora pasamos a crear unas gráficas más completas para poder ver bien la evolución de las ejecuciones.
Retomamos la recompensa basada en la posición respecto al centro \ref{fig:mountaincar_070}. Y hacemos la correspondiente versión para la velocidad \ref{fig:mountaincar_071}.

\figura{Bitmap/ApendiceA/mainPosCentradaGraficaCompleta.png}{width=1\textwidth}{fig:mountaincar_070}%
{Prototipo Gráfica mejorada Posición Centro}

\figura{Bitmap/ApendiceA/mainVelocidadGraficaCompleta.png}{width=1\textwidth}{fig:mountaincar_071}%
{Prototipo Gráfica mejorada velocidad absoluta}

La primera grafica representa los datos referentes a la recompensa, tanto la recompensa del \textbf{entorno} en \textbf{rojo} y como la que nosotros le asignamos en \textbf{azul}.
Las gráficas \textbf{moradas} muestran respectivamente la \textbf{velocidad} máxima alcanzada y la velocidad acumulada durante toda cada episodio.
Las gráficas \textbf{verdes} muestran la información referente a la \textbf{posición} siendo la primera la posición máxima alcanzada, si es 0.5 implica terminar la partida y la posición acumulada, para así medir claramente el desplazamiento del avatar durante sus episodios.

\subsection{Prototipo 8}
Se puede ver claramente como el aprendizaje de los prototipos anteriores fluctúa bastante. Por ello creamos un nuevo agente más estable que añade una tercera red neuronal en la que guardara la mejor configuración encontrada hasta el momento de esa forma podemos prevenir la regresión en el aprendizaje.

\figura{Bitmap/ApendiceA/MAIN10AGENT6.png}{width=1\textwidth}{fig:mountaincar_08}%
{RELLENAR}

La gráfica inferior izquierda muestra los resultados de las competiciones entre las redes neuronales, las marcas verdes indican el resultado de las grandes competiciones, en las que se renueva la mejor red guardada.
Además, cambiamos la configuración de las gráficas para prestar información más interesante.
También añadimos una competición final al terminar la ejecución para comparar los resultados finales.

\figura{Bitmap/ApendiceA/mountainCar_08.png}{width=1\textwidth}{fig:mountaincar_081}%
{RELLENAR}

\figura{Bitmap/ApendiceA/mountainCar_09.png}{width=1\textwidth}{fig:mountaincar_082}%
{RELLENAR}

\figura{Bitmap/ApendiceA/mountainCar_10.png}{width=1\textwidth}{fig:mountaincar_083}%
{RELLENAR}

En estas gráficas se puede ver la evolución de los resultados de las tres redes neuronales, así como la posición máxima alcanzada en cada episodio y por ultimo los resultados de las tres redes en una prueba para evaluar su aprendizaje.
Las dos primeras corresponden al doble agente, y la siguiente muestra el resultado obtenido de Prototipo 8.




\chapter{Título}
\label{Appendix:Key2}

%\include{Apendices/appendixC}
%\include{...}
%\include{...}
%\include{...}
\backmatter

%
% Bibliografía
%
% Si el TFM se escribe en inglés, editar TeXiS/TeXiS_bib para cambiar el
% estilo de las referencias
%---------------------------------------------------------------------
%
%                      configBibliografia.tex
%
%---------------------------------------------------------------------
%
% bibliografia.tex
% Copyright 2009 Marco Antonio Gomez-Martin, Pedro Pablo Gomez-Martin
%
% This file belongs to the TeXiS manual, a LaTeX template for writting
% Thesis and other documents. The complete last TeXiS package can
% be obtained from http://gaia.fdi.ucm.es/projects/texis/
%
% Although the TeXiS template itself is distributed under the 
% conditions of the LaTeX Project Public License
% (http://www.latex-project.org/lppl.txt), the manual content
% uses the CC-BY-SA license that stays that you are free:
%
%    - to share & to copy, distribute and transmit the work
%    - to remix and to adapt the work
%
% under the following conditions:
%
%    - Attribution: you must attribute the work in the manner
%      specified by the author or licensor (but not in any way that
%      suggests that they endorse you or your use of the work).
%    - Share Alike: if you alter, transform, or build upon this
%      work, you may distribute the resulting work only under the
%      same, similar or a compatible license.
%
% The complete license is available in
% http://creativecommons.org/licenses/by-sa/3.0/legalcode
%
%---------------------------------------------------------------------
%
% Fichero  que  configura  los  parámetros  de  la  generación  de  la
% bibliografía.  Existen dos  parámetros configurables:  los ficheros
% .bib que se utilizan y la frase célebre que aparece justo antes de la
% primera referencia.
%
%---------------------------------------------------------------------


%%%%%%%%%%%%%%%%%%%%%%%%%%%%%%%%%%%%%%%%%%%%%%%%%%%%%%%%%%%%%%%%%%%%%%
% Definición de los ficheros .bib utilizados:
% \setBibFiles{<lista ficheros sin extension, separados por comas>}
% Nota:
% Es IMPORTANTE que los ficheros estén en la misma línea que
% el comando \setBibFiles. Si se desea utilizar varias líneas,
% terminarlas con una apertura de comentario.
%%%%%%%%%%%%%%%%%%%%%%%%%%%%%%%%%%%%%%%%%%%%%%%%%%%%%%%%%%%%%%%%%%%%%%
\setBibFiles{%
nuestros,latex,otros%
}

%%%%%%%%%%%%%%%%%%%%%%%%%%%%%%%%%%%%%%%%%%%%%%%%%%%%%%%%%%%%%%%%%%%%%%
% Definición de la frase célebre para el capítulo de la
% bibliografía. Dentro normalmente se querrá hacer uso del entorno
% \begin{FraseCelebre}, que contendrá a su vez otros dos entornos,
% un \begin{Frase} y un \begin{Fuente}.
%
% Nota:
% Si no se quiere cita, se puede eliminar su definición (en la
% macro setCitaBibliografia{} ).
%%%%%%%%%%%%%%%%%%%%%%%%%%%%%%%%%%%%%%%%%%%%%%%%%%%%%%%%%%%%%%%%%%%%%%
% \setCitaBibliografia{
% \begin{FraseCelebre}
% \begin{Frase}
%   Y así, del mucho leer y del poco dormir, se le secó el celebro de
%   manera que vino a perder el juicio.
% \end{Frase}
% \begin{Fuente}
%   Miguel de Cervantes Saavedra
% \end{Fuente}
% \end{FraseCelebre}
% }

%%
%% Creamos la bibliografia
%%
\makeBib

% Variable local para emacs, para  que encuentre el fichero maestro de
% compilación y funcionen mejor algunas teclas rápidas de AucTeX

%%%
%%% Local Variables:
%%% mode: latex
%%% TeX-master: "../Tesis.tex"
%%% End:

%
% Índice de palabras
%

% Sólo  la   generamos  si  está   declarada  \generaindice.  Consulta
% TeXiS.sty para más información.

% En realidad, el soporte para la generación de índices de palabras
% en TeXiS no está documentada en el manual, porque no ha sido usada
% "en producción". Por tanto, el fichero que genera el índice
% *no* se incluye aquí (está comentado). Consulta la documentación
% en TeXiS_pream.tex para más información.
\ifx\generaindice\undefined
\else
%%---------------------------------------------------------------------
%
%                        TeXiS_indice.tex
%
%---------------------------------------------------------------------
%
% TeXiS_indice.tex
% Copyright 2009 Marco Antonio Gomez-Martin, Pedro Pablo Gomez-Martin
%
% This file belongs to TeXiS, a LaTeX template for writting
% Thesis and other documents. The complete last TeXiS package can
% be obtained from http://gaia.fdi.ucm.es/projects/texis/
%
% This work may be distributed and/or modified under the
% conditions of the LaTeX Project Public License, either version 1.3
% of this license or (at your option) any later version.
% The latest version of this license is in
%   http://www.latex-project.org/lppl.txt
% and version 1.3 or later is part of all distributions of LaTeX
% version 2005/12/01 or later.
%
% This work has the LPPL maintenance status `maintained'.
% 
% The Current Maintainers of this work are Marco Antonio Gomez-Martin
% and Pedro Pablo Gomez-Martin
%
%---------------------------------------------------------------------
%
% Contiene  los  comandos  para  generar  el índice  de  palabras  del
% documento.
%
%---------------------------------------------------------------------
%
% NOTA IMPORTANTE: el  soporte en TeXiS para el  índice de palabras es
% embrionario, y  de hecho  ni siquiera se  describe en el  manual. Se
% proporciona  una infraestructura  básica (sin  terminar)  para ello,
% pero  no ha  sido usada  "en producción".  De hecho,  a pesar  de la
% existencia de  este fichero, *no* se incluye  en Tesis.tex. Consulta
% la documentación en TeXiS_pream.tex para más información.
%
%---------------------------------------------------------------------


% Si se  va a generar  la tabla de  contenidos (el índice  habitual) y
% también vamos a  generar el índice de palabras  (ambas decisiones se
% toman en  función de  la definición  o no de  un par  de constantes,
% puedes consultar modo.tex para más información), entonces metemos en
% la tabla de contenidos una  entrada para marcar la página donde está
% el índice de palabras.

\ifx\generatoc\undefined
\else
   \addcontentsline{toc}{chapter}{\indexname}
\fi


% Generamos el índice
\printindex

% Variable local para emacs, para  que encuentre el fichero maestro de
% compilación y funcionen mejor algunas teclas rápidas de AucTeX

%%%
%%% Local Variables:
%%% mode: latex
%%% TeX-master: "./tesis.tex"
%%% End:

\fi

%
% Lista de acrónimos
%

% Sólo  lo  generamos  si  está declarada  \generaacronimos.  Consulta
% TeXiS.sty para más información.


\ifx\generaacronimos\undefined
\else
%---------------------------------------------------------------------
%
%                        TeXiS_acron.tex
%
%---------------------------------------------------------------------
%
% TeXiS_acron.tex
% Copyright 2009 Marco Antonio Gomez-Martin, Pedro Pablo Gomez-Martin
%
% This file belongs to TeXiS, a LaTeX template for writting
% Thesis and other documents. The complete last TeXiS package can
% be obtained from http://gaia.fdi.ucm.es/projects/texis/
%
% This work may be distributed and/or modified under the
% conditions of the LaTeX Project Public License, either version 1.3
% of this license or (at your option) any later version.
% The latest version of this license is in
%   http://www.latex-project.org/lppl.txt
% and version 1.3 or later is part of all distributions of LaTeX
% version 2005/12/01 or later.
%
% This work has the LPPL maintenance status `maintained'.
% 
% The Current Maintainers of this work are Marco Antonio Gomez-Martin
% and Pedro Pablo Gomez-Martin
%
%---------------------------------------------------------------------
%
% Contiene  los  comandos  para  generar  el listado de acrónimos
% documento.
%
%---------------------------------------------------------------------
%
% NOTA IMPORTANTE:  para que la  generación de acrónimos  funcione, al
% menos  debe  existir  un  acrónimo   en  el  documento.  Si  no,  la
% compilación  del   fichero  LaTeX  falla  con   un  error  "extraño"
% (indicando  que  quizá  falte  un \item).   Consulta  el  comentario
% referente al paquete glosstex en TeXiS_pream.tex.
%
%---------------------------------------------------------------------


% Redefinimos a español  el título de la lista  de acrónimos (Babel no
% lo hace por nosotros esta vez)

\def\listacronymname{Lista de acrónimos}

% Para el glosario:
% \def\glosarryname{Glosario}

% Si se  va a generar  la tabla de  contenidos (el índice  habitual) y
% también vamos a  generar la lista de acrónimos  (ambas decisiones se
% toman en  función de  la definición  o no de  un par  de constantes,
% puedes consultar config.tex  para más información), entonces metemos
% en la  tabla de contenidos una  entrada para marcar  la página donde
% está el índice de palabras.

\ifx\generatoc\undefined
\else
   \addcontentsline{toc}{chapter}{\listacronymname}
\fi


% Generamos la lista de acrónimos (en realidad el índice asociado a la
% lista "acr" de GlossTeX)

\printglosstex(acr)

% Variable local para emacs, para  que encuentre el fichero maestro de
% compilación y funcionen mejor algunas teclas rápidas de AucTeX

%%%
%%% Local Variables:
%%% mode: latex
%%% TeX-master: "../Tesis.tex"
%%% End:

\fi

%
% Final
%
%---------------------------------------------------------------------
%
%                      fin.tex
%
%---------------------------------------------------------------------
%
% fin.tex
% Copyright 2009 Marco Antonio Gomez-Martin, Pedro Pablo Gomez-Martin
%
% This file belongs to the TeXiS manual, a LaTeX template for writting
% Thesis and other documents. The complete last TeXiS package can
% be obtained from http://gaia.fdi.ucm.es/projects/texis/
%
% Although the TeXiS template itself is distributed under the 
% conditions of the LaTeX Project Public License
% (http://www.latex-project.org/lppl.txt), the manual content
% uses the CC-BY-SA license that stays that you are free:
%
%    - to share & to copy, distribute and transmit the work
%    - to remix and to adapt the work
%
% under the following conditions:
%
%    - Attribution: you must attribute the work in the manner
%      specified by the author or licensor (but not in any way that
%      suggests that they endorse you or your use of the work).
%    - Share Alike: if you alter, transform, or build upon this
%      work, you may distribute the resulting work only under the
%      same, similar or a compatible license.
%
% The complete license is available in
% http://creativecommons.org/licenses/by-sa/3.0/legalcode
%
%---------------------------------------------------------------------
%
% Contiene la última página
%
%---------------------------------------------------------------------


% Ponemos el marcador en el PDF
\ifpdf
   \pdfbookmark{Fin}{fin}
\fi

\thispagestyle{empty}\mbox{}

\vspace*{4cm}

\small

\hfill \emph{--¿Qué te parece desto, Sancho? -- Dijo Don Quijote --}

\hfill \emph{Bien podrán los encantadores quitarme la ventura,}

\hfill \emph{pero el esfuerzo y el ánimo, será imposible.}

\hfill 

\hfill \emph{Segunda parte del Ingenioso Caballero} 

\hfill \emph{Don Quijote de la Mancha}

\hfill \emph{Miguel de Cervantes}

\vfill%space*{4cm}

\hfill \emph{--Buena está -- dijo Sancho --; fírmela vuestra merced.}

\hfill \emph{--No es menester firmarla -- dijo Don Quijote--,}

\hfill \emph{sino solamente poner mi rúbrica.}

\hfill 

\hfill \emph{Primera parte del Ingenioso Caballero} 

\hfill \emph{Don Quijote de la Mancha}

\hfill \emph{Miguel de Cervantes}


\newpage
\thispagestyle{empty}\mbox{}

\newpage

% Variable local para emacs, para  que encuentre el fichero maestro de
% compilación y funcionen mejor algunas teclas rápidas de AucTeX

%%%
%%% Local Variables:
%%% mode: latex
%%% TeX-master: "../Tesis.tex"
%%% End:

%\end{otherlanguage}
\end{document}
