% ----------------------------------------------------------------------
%
%                            TFMTesis.tex
%
%----------------------------------------------------------------------
%
% Este fichero contiene el "documento maestro" del documento. Lo único
% que hace es configurar el entorno LaTeX e incluir los ficheros .tex
% que contienen cada sección.
%
%----------------------------------------------------------------------
%
% Los ficheros necesarios para este documento son:
%
%       TeXiS/* : ficheros de la plantilla TeXiS.
%       Cascaras/* : ficheros con las partes del documento que no
%          son capítulos ni apéndices (portada, agradecimientos, etc.)
%       Capitulos/*.tex : capítulos de la tesis
%       Apendices/*.tex: apéndices de la tesis
%       constantes.tex: constantes LaTeX
%       config.tex : configuración de la "compilación" del documento
%       guionado.tex : palabras con guiones
%
% Para la bibliografía, además, se necesitan:
%
%       *.bib : ficheros con la información de las referencias
%
% ---------------------------------------------------------------------

\documentclass[11pt,a4paper,twoside]{book}

%
% Definimos  el   comando  \compilaCapitulo,  que   luego  se  utiliza
% (opcionalmente) en config.tex. Quedaría  mejor si también se definiera
% en  ese fichero,  pero por  el modo  en el  que funciona  eso  no es
% posible. Puedes consultar la documentación de ese fichero para tener
% más  información. Definimos también  \compilaApendice, que  tiene el
% mismo  cometido, pero  que se  utiliza para  compilar  únicamente un
% apéndice.
%
%
% Si  queremos   compilar  solo   una  parte  del   documento  podemos
% especificar mediante  \includeonly{...} qué ficheros  son los únicos
% que queremos  que se incluyan.  Esto  es útil por  ejemplo para sólo
% compilar un capítulo.
%
% El problema es que todos aquellos  ficheros que NO estén en la lista
% NO   se  incluirán...  y   eso  también   afecta  a   ficheros  de
% la plantilla...
%
% Total,  que definimos  una constante  con los  ficheros  que siempre
% vamos a querer compilar  (aquellos relacionados con configuración) y
% luego definimos \compilaCapitulo.
\newcommand{\ficherosBasicosTeXiS}{%
TeXiS/TeXiS_pream,TeXiS/TeXiS_cab,TeXiS/TeXiS_bib,TeXiS/TeXiS_cover%
}
\newcommand{\ficherosBasicosTexto}{%
constantes,guionado,Cascaras/bibliografia,config%
}
\newcommand{\compilaCapitulo}[1]{%
\includeonly{\ficherosBasicosTeXiS,\ficherosBasicosTexto,Capitulos/#1}%
}

\newcommand{\compilaApendice}[1]{%
\includeonly{\ficherosBasicosTeXiS,\ficherosBasicosTexto,Apendices/#1}%
}

%- - - - - - - - - - - - - - - - - - - - - - - - - - - - - - - - - - -
%            Preámbulo del documento. Configuraciones varias
%- - - - - - - - - - - - - - - - - - - - - - - - - - - - - - - - - - -

% Define  el  tipo  de  compilación que  estamos  haciendo.   Contiene
% definiciones  de  constantes que  cambian  el  comportamiento de  la
% compilación. Debe incluirse antes del paquete TeXiS/TeXiS.sty
%---------------------------------------------------------------------
%
%                          config.tex
%
%---------------------------------------------------------------------
%
% Contiene la  definición de constantes  que determinan el modo  en el
% que se compilará el documento.
%
%---------------------------------------------------------------------
%
% En concreto, podemos  indicar si queremos "modo release",  en el que
% no  aparecerán  los  comentarios  (creados  mediante  \com{Texto}  o
% \comp{Texto}) ni los "por  hacer" (creados mediante \todo{Texto}), y
% sí aparecerán los índices. El modo "debug" (o mejor dicho en modo no
% "release" muestra los índices  (construirlos lleva tiempo y son poco
% útiles  salvo  para   la  versión  final),  pero  sí   el  resto  de
% anotaciones.
%
% Si se compila con LaTeX (no  con pdflatex) en modo Debug, también se
% muestran en una esquina de cada página las entradas (en el índice de
% palabras) que referencian  a dicha página (consulta TeXiS_pream.tex,
% en la parte referente a show).
%
% El soporte para  el índice de palabras en  TeXiS es embrionario, por
% lo  que no  asumas que  esto funcionará  correctamente.  Consulta la
% documentación al respecto en TeXiS_pream.tex.
%
%
% También  aquí configuramos  si queremos  o  no que  se incluyan  los
% acrónimos  en el  documento final  en la  versión release.  Para eso
% define (o no) la constante \acronimosEnRelease.
%
% Utilizando \compilaCapitulo{nombre}  podemos también especificar qué
% capítulo(s) queremos que se compilen. Si no se pone nada, se compila
% el documento  completo.  Si se pone, por  ejemplo, 01Introduccion se
% compilará únicamente el fichero Capitulos/01Introduccion.tex
%
% Para compilar varios  capítulos, se separan sus nombres  con comas y
% no se ponen espacios de separación.
%
% En realidad  la macro \compilaCapitulo  está definida en  el fichero
% principal tesis.tex.
%
%---------------------------------------------------------------------


% Comentar la línea si no se compila en modo release.
% TeXiS hará el resto.
% ¡¡¡Si cambias esto, haz un make clean antes de recompilar!!!
\def\release{1}


% Descomentar la linea si se quieren incluir los
% acrónimos en modo release (en modo debug
% no se incluirán nunca).
% ¡¡¡Si cambias esto, haz un make clean antes de recompilar!!!
%\def\acronimosEnRelease{1}


% Descomentar la línea para establecer el capítulo que queremos
% compilar

% \compilaCapitulo{01Introduccion}
% \compilaCapitulo{02EstructuraYGeneracion}
% \compilaCapitulo{03Edicion}
% \compilaCapitulo{04Imagenes}
% \compilaCapitulo{05Bibliografia}
% \compilaCapitulo{06Makefile}

% \compilaApendice{01AsiSeHizo}

% Variable local para emacs, para  que encuentre el fichero maestro de
% compilación y funcionen mejor algunas teclas rápidas de AucTeX
%%%
%%% Local Variables:
%%% mode: latex
%%% TeX-master: "./Tesis.tex"
%%% End:


% Paquete de la plantilla
\usepackage{TeXiS/TeXiS}

% Incluimos el fichero con comandos de constantes
%---------------------------------------------------------------------
%
%                          constantes.tex
%
%---------------------------------------------------------------------
%
% Fichero que  declara nuevos comandos LaTeX  sencillos realizados por
% comodidad en la escritura de determinadas palabras
%
%---------------------------------------------------------------------

%%%%%%%%%%%%%%%%%%%%%%%%%%%%%%%%%%%%%%%%%%%%%%%%%%%%%%%%%%%%%%%%%%%%%%
% Comando: 
%
%       \titulo
%
% Resultado: 
%
% Escribe el título del documento.
%%%%%%%%%%%%%%%%%%%%%%%%%%%%%%%%%%%%%%%%%%%%%%%%%%%%%%%%%%%%%%%%%%%%%%
\def\titulo{\textsc{TeXiS}: Una plantilla de \LaTeX\
  para Tesis y otros documentos}

%%%%%%%%%%%%%%%%%%%%%%%%%%%%%%%%%%%%%%%%%%%%%%%%%%%%%%%%%%%%%%%%%%%%%%
% Comando: 
%
%       \autor
%
% Resultado: 
%
% Escribe el autor del documento.
%%%%%%%%%%%%%%%%%%%%%%%%%%%%%%%%%%%%%%%%%%%%%%%%%%%%%%%%%%%%%%%%%%%%%%
\def\autor{Marco Antonio y Pedro Pablo G\'omez Mart\'in}

% Variable local para emacs, para  que encuentre el fichero maestro de
% compilación y funcionen mejor algunas teclas rápidas de AucTeX

%%%
%%% Local Variables:
%%% mode: latex
%%% TeX-master: "tesis.tex"
%%% End:


% Sacamos en el log de la compilación el copyright
%\typeout{Copyright Marco Antonio and Pedro Pablo Gomez Martin}

%
% "Metadatos" para el PDF
%
\ifpdf\hypersetup{%
    pdftitle = {\titulo},
    pdfsubject = {Plantilla de Tesis},
    pdfkeywords = {Plantilla, LaTeX, tesis, trabajo de
      investigación, trabajo de Master},
    pdfauthor = {\textcopyright\ \autor},
    pdfcreator = {\LaTeX\ con el paquete \flqq hyperref\frqq},
    pdfproducer = {pdfeTeX-0.\the\pdftexversion\pdftexrevision},
    }
    \pdfinfo{/CreationDate (\today)}
\fi


%- - - - - - - - - - - - - - - - - - - - - - - - - - - - - - - - - - -
%                        Documento
%- - - - - - - - - - - - - - - - - - - - - - - - - - - - - - - - - - -
\begin{document}

% Incluimos el  fichero de definición de guionado  de algunas palabras
% que LaTeX no ha dividido como debería
%----------------------------------------------------------------
%
%                          guionado.tex
%
%----------------------------------------------------------------
%
% Fichero con algunas divisiones de palabras que LaTeX no
% hace correctamente si no se le da alguna ayuda.
%
%----------------------------------------------------------------

\hyphenation{
% a
abs-trac-to
abs-trac-tos
abs-trac-ta
abs-trac-tas
ac-tua-do-res
a-gra-de-ci-mien-tos
ana-li-za-dor
an-te-rio-res
an-te-rior-men-te
apa-rien-cia
a-pro-pia-do
a-pro-pia-dos
a-pro-pia-da
a-pro-pia-das
a-pro-ve-cha-mien-to
a-que-llo
a-que-llos
a-que-lla
a-que-llas
a-sig-na-tu-ra
a-sig-na-tu-ras
a-so-cia-da
a-so-cia-das
a-so-cia-do
a-so-cia-dos
au-to-ma-ti-za-do
% b
batch
bi-blio-gra-fía
bi-blio-grá-fi-cas
bien
bo-rra-dor
boo-l-ean-expr
% c
ca-be-ce-ra
call-me-thod-ins-truc-tion
cas-te-lla-no
cir-cuns-tan-cia
cir-cuns-tan-cias
co-he-ren-te
co-he-ren-tes
co-he-ren-cia
co-li-bri
co-men-ta-rio
co-mer-cia-les
co-no-ci-mien-to
cons-cien-te
con-si-de-ra-ba
con-si-de-ra-mos
con-si-de-rar-se
cons-tan-te
cons-trucción
cons-tru-ye
cons-tru-ir-se
con-tro-le
co-rrec-ta-men-te
co-rres-pon-den
co-rres-pon-dien-te
co-rres-pon-dien-tes
co-ti-dia-na
co-ti-dia-no
crean
cris-ta-li-zan
cu-rri-cu-la
cu-rri-cu-lum
cu-rri-cu-lar
cu-rri-cu-la-res
% d
de-di-ca-do
de-di-ca-dos
de-di-ca-da
de-di-ca-das
de-rro-te-ro
de-rro-te-ros
de-sa-rro-llo
de-sa-rro-llos
de-sa-rro-lla-do
de-sa-rro-lla-dos
de-sa-rro-lla-da
de-sa-rro-lla-das
de-sa-rro-lla-dor
de-sa-rro-llar
des-cri-bi-re-mos
des-crip-ción
des-crip-cio-nes
des-cri-to
des-pués
de-ta-lla-do
de-ta-lla-dos
de-ta-lla-da
de-ta-lla-das
di-a-gra-ma
di-a-gra-mas
di-se-ños
dis-po-ner
dis-po-ni-bi-li-dad
do-cu-men-ta-da
do-cu-men-to
do-cu-men-tos
% e
edi-ta-do
e-du-ca-ti-vo
e-du-ca-ti-vos
e-du-ca-ti-va
e-du-ca-ti-vas
e-la-bo-ra-do
e-la-bo-ra-dos
e-la-bo-ra-da
e-la-bo-ra-das
es-co-llo
es-co-llos
es-tu-dia-do
es-tu-dia-dos
es-tu-dia-da
es-tu-dia-das
es-tu-dian-te
e-va-lua-cio-nes
e-va-lua-do-res
exis-ten-tes
exhaus-ti-va
ex-pe-rien-cia
ex-pe-rien-cias
% f
for-ma-li-za-do
% g
ge-ne-ra-ción
ge-ne-ra-dor
ge-ne-ra-do-res
ge-ne-ran
% h
he-rra-mien-ta
he-rra-mien-tas
% i
i-dio-ma
i-dio-mas
im-pres-cin-di-ble
im-pres-cin-di-bles
in-de-xa-do
in-de-xa-dos
in-de-xa-da
in-de-xa-das
in-di-vi-dual
in-fe-ren-cia
in-fe-ren-cias
in-for-ma-ti-ca
in-gre-dien-te
in-gre-dien-tes
in-me-dia-ta-men-te
ins-ta-la-do
ins-tan-cias
% j
% k
% l
len-gua-je
li-be-ra-to-rio
li-be-ra-to-rios
li-be-ra-to-ria
li-be-ra-to-rias
li-mi-ta-do
li-te-ra-rio
li-te-ra-rios
li-te-ra-ria
li-te-ra-rias
lo-tes
% m
ma-ne-ra
ma-nual
mas-que-ra-de
ma-yor
me-mo-ria
mi-nis-te-rio
mi-nis-te-rios
mo-de-lo
mo-de-los
mo-de-la-do
mo-du-la-ri-dad
mo-vi-mien-to
% n
na-tu-ral
ni-vel
nues-tro
% o
obs-tan-te
o-rien-ta-do
o-rien-ta-dos
o-rien-ta-da
o-rien-ta-das
% p
pa-ra-le-lo
pa-ra-le-la
par-ti-cu-lar
par-ti-cu-lar-men-te
pe-da-gó-gi-ca
pe-da-gó-gi-cas
pe-da-gó-gi-co
pe-da-gó-gi-cos
pe-rio-di-ci-dad
per-so-na-je
plan-te-a-mien-to
plan-te-a-mien-tos
po-si-ción
pre-fe-ren-cia
pre-fe-ren-cias
pres-cin-di-ble
pres-cin-di-bles
pri-me-ra
pro-ble-ma
pro-ble-mas
pró-xi-mo
pu-bli-ca-cio-nes
pu-bli-ca-do
% q
% r
rá-pi-da
rá-pi-do
ra-zo-na-mien-to
ra-zo-na-mien-tos
re-a-li-zan-do
re-fe-ren-cia
re-fe-ren-cias
re-fe-ren-cia-da
re-fe-ren-cian
re-le-van-tes
re-pre-sen-ta-do
re-pre-sen-ta-dos
re-pre-sen-ta-da
re-pre-sen-ta-das
re-pre-sen-tar-lo
re-qui-si-to
re-qui-si-tos
res-pon-der
res-pon-sa-ble
% s
se-pa-ra-do
si-guien-do
si-guien-te
si-guien-tes
si-guie-ron
si-mi-lar
si-mi-la-res
si-tua-ción
% t
tem-pe-ra-ments
te-ner
trans-fe-ren-cia
trans-fe-ren-cias
% u
u-sua-rio
Unreal-Ed
% v
va-lor
va-lo-res
va-rian-te
ver-da-de-ro
ver-da-de-ros
ver-da-de-ra
ver-da-de-ras
ver-da-de-ra-men-te
ve-ri-fi-ca
% w
% x
% y
% z
}
% Variable local para emacs, para que encuentre el fichero
% maestro de compilación
%%%
%%% Local Variables:
%%% mode: latex
%%% TeX-master: "./Tesis.tex"
%%% End:


% Marcamos  el inicio  del  documento para  la  numeración de  páginas
% (usando números romanos para esta primera fase).
\frontmatter
\pagestyle{empty}

%---------------------------------------------------------------------
%
%                          configCover.tex
%
%---------------------------------------------------------------------
%
% cover.tex
% Copyright 2009 Marco Antonio Gomez-Martin, Pedro Pablo Gomez-Martin
%
% This file belongs to the TeXiS manual, a LaTeX template for writting
% Thesis and other documents. The complete last TeXiS package can
% be obtained from http://gaia.fdi.ucm.es/projects/texis/
%
% Although the TeXiS template itself is distributed under the 
% conditions of the LaTeX Project Public License
% (http://www.latex-project.org/lppl.txt), the manual content
% uses the CC-BY-SA license that stays that you are free:
%
%    - to share & to copy, distribute and transmit the work
%    - to remix and to adapt the work
%
% under the following conditions:
%
%    - Attribution: you must attribute the work in the manner
%      specified by the author or licensor (but not in any way that
%      suggests that they endorse you or your use of the work).
%    - Share Alike: if you alter, transform, or build upon this
%      work, you may distribute the resulting work only under the
%      same, similar or a compatible license.
%
% The complete license is available in
% http://creativecommons.org/licenses/by-sa/3.0/legalcode
%
%---------------------------------------------------------------------
%
% Fichero que contiene la configuración de la portada y de la 
% primera hoja del documento.
%
%---------------------------------------------------------------------


% Pueden configurarse todos los elementos del contenido de la portada
% utilizando comandos.

%%%%%%%%%%%%%%%%%%%%%%%%%%%%%%%%%%%%%%%%%%%%%%%%%%%%%%%%%%%%%%%%%%%%%%
% Título del documento:
% \tituloPortada{titulo}
% Nota:
% Si no se define se utiliza el del \titulo. Este comando permite
% cambiar el título de forma que se especifiquen dónde se quieren
% los retornos de carro cuando se utilizan fuentes grandes.
%%%%%%%%%%%%%%%%%%%%%%%%%%%%%%%%%%%%%%%%%%%%%%%%%%%%%%%%%%%%%%%%%%%%%%
\tituloPortada{%
Deep Reinforcement Learning en juegos
}

%%%%%%%%%%%%%%%%%%%%%%%%%%%%%%%%%%%%%%%%%%%%%%%%%%%%%%%%%%%%%%%%%%%%%%
% Autor del documento:
% \autorPortada{Nombre}
% Se utiliza en la portada y en el valor por defecto del
% primer subtítulo de la segunda portada.
%%%%%%%%%%%%%%%%%%%%%%%%%%%%%%%%%%%%%%%%%%%%%%%%%%%%%%%%%%%%%%%%%%%%%%
\autorPortada{Ricardo Arranz Janeiro\\
Lidia Concepción Echeverría\\
Juan Ramón Del Caño Vega\\
Francisco Ponce Belmonte\\
Juan Luis Romero Sánchez}

%%%%%%%%%%%%%%%%%%%%%%%%%%%%%%%%%%%%%%%%%%%%%%%%%%%%%%%%%%%%%%%%%%%%%%
% Fecha de publicación:
% \fechaPublicacion{Fecha}
% Puede ser vacío. Aparece en la última línea de ambas portadas
%%%%%%%%%%%%%%%%%%%%%%%%%%%%%%%%%%%%%%%%%%%%%%%%%%%%%%%%%%%%%%%%%%%%%%
\fechaPublicacion{\today}

%%%%%%%%%%%%%%%%%%%%%%%%%%%%%%%%%%%%%%%%%%%%%%%%%%%%%%%%%%%%%%%%%%%%%%
% Imagen de la portada (y escala)
% \imagenPortada{Fichero}
% \escalaImagenPortada{Numero}
% Si no se especifica, se utiliza la imagen TODO.pdf
%%%%%%%%%%%%%%%%%%%%%%%%%%%%%%%%%%%%%%%%%%%%%%%%%%%%%%%%%%%%%%%%%%%%%%
\imagenPortada{Imagenes/Vectorial/escudoUCM}
\escalaImagenPortada{.2}

%%%%%%%%%%%%%%%%%%%%%%%%%%%%%%%%%%%%%%%%%%%%%%%%%%%%%%%%%%%%%%%%%%%%%%
% Tipo de documento.
% \tipoDocumento{Tipo}
% Para el texto justo debajo del escudo.
% Si no se indica, se utiliza "TESIS DOCTORAL".
%%%%%%%%%%%%%%%%%%%%%%%%%%%%%%%%%%%%%%%%%%%%%%%%%%%%%%%%%%%%%%%%%%%%%%
\tipoDocumento{Trabajo de Fin de Grado}

%%%%%%%%%%%%%%%%%%%%%%%%%%%%%%%%%%%%%%%%%%%%%%%%%%%%%%%%%%%%%%%%%%%%%%
% Institución/departamento asociado al documento.
% \institucion{Nombre}
% Puede tener varias líneas. Se utiliza en las dos portadas.
% Si no se indica aparecerá vacío.
%%%%%%%%%%%%%%%%%%%%%%%%%%%%%%%%%%%%%%%%%%%%%%%%%%%%%%%%%%%%%%%%%%%%%%
\institucion{%
Grado en Ingeniería Informática\\[0.2em]
Facultad de Informática\\[0.2em]
Universidad Complutense de Madrid
}

%%%%%%%%%%%%%%%%%%%%%%%%%%%%%%%%%%%%%%%%%%%%%%%%%%%%%%%%%%%%%%%%%%%%%%
% Director del trabajo.
% \directorPortada{Nombre}
% Se utiliza para el valor por defecto del segundo subtítulo, donde
% se indica quién es el director del trabajo.
% Si se fuerza un subtítulo distinto, no hace falta definirlo.
%%%%%%%%%%%%%%%%%%%%%%%%%%%%%%%%%%%%%%%%%%%%%%%%%%%%%%%%%%%%%%%%%%%%%%
\directorPortada{Antonio Alejandro Sánchez Ruiz-Granados}

%%%%%%%%%%%%%%%%%%%%%%%%%%%%%%%%%%%%%%%%%%%%%%%%%%%%%%%%%%%%%%%%%%%%%%
% Texto del primer subtítulo de la segunda portada.
% \textoPrimerSubtituloPortada{Texto}
% Para configurar el primer "texto libre" de la segunda portada.
% Si no se especifica se indica "Memoria que presenta para optar al
% título de Doctor en Informática" seguido del \autorPortada.
%%%%%%%%%%%%%%%%%%%%%%%%%%%%%%%%%%%%%%%%%%%%%%%%%%%%%%%%%%%%%%%%%%%%%%
\textoPrimerSubtituloPortada{%
\textbf{Trabajo de Fin de Grado en Ingeniería Informática}  \\ [0.3em]
\textbf{Departamento de Ingeniería de Software e Inteligencia Artificial} \\ [0.3em]
}

%%%%%%%%%%%%%%%%%%%%%%%%%%%%%%%%%%%%%%%%%%%%%%%%%%%%%%%%%%%%%%%%%%%%%%
% Texto del segundo subtítulo de la segunda portada.
% \textoSegundoSubtituloPortada{Texto}
% Para configurar el segundo "texto libre" de la segunda portada.
% Si no se especifica se indica "Dirigida por el Doctor" seguido
% del \directorPortada.
%%%%%%%%%%%%%%%%%%%%%%%%%%%%%%%%%%%%%%%%%%%%%%%%%%%%%%%%%%%%%%%%%%%%%%
\textoSegundoSubtituloPortada{%
\textbf{Convocatoria: }\textit{Junio \the\year} \\ [0.2em]
\textbf{Calificación: }\textit{}
}

%%%%%%%%%%%%%%%%%%%%%%%%%%%%%%%%%%%%%%%%%%%%%%%%%%%%%%%%%%%%%%%%%%%%%%
% \explicacionDobleCara
% Si se utiliza, se aclara que el documento está preparado para la
% impresión a doble cara.
%%%%%%%%%%%%%%%%%%%%%%%%%%%%%%%%%%%%%%%%%%%%%%%%%%%%%%%%%%%%%%%%%%%%%%
\explicacionDobleCara

%%%%%%%%%%%%%%%%%%%%%%%%%%%%%%%%%%%%%%%%%%%%%%%%%%%%%%%%%%%%%%%%%%%%%%
% \isbn
% Si se utiliza, aparecerá el ISBN detrás de la segunda portada.
%%%%%%%%%%%%%%%%%%%%%%%%%%%%%%%%%%%%%%%%%%%%%%%%%%%%%%%%%%%%%%%%%%%%%%
%\isbn{978-84-692-7109-4}


%%%%%%%%%%%%%%%%%%%%%%%%%%%%%%%%%%%%%%%%%%%%%%%%%%%%%%%%%%%%%%%%%%%%%%
% \copyrightInfo
% Si se utiliza, aparecerá información de los derechos de copyright
% detrás de la segunda portada.
%%%%%%%%%%%%%%%%%%%%%%%%%%%%%%%%%%%%%%%%%%%%%%%%%%%%%%%%%%%%%%%%%%%%%%
\copyrightInfo{\autor}


%%
%% Creamos las portadas
%%
\makeCover

% Variable local para emacs, para que encuentre el fichero
% maestro de compilación
%%%
%%% Local Variables:
%%% mode: latex
%%% TeX-master: "../Tesis.tex"
%%% End:

% \chapter*{Autorización de difusión}

   
El abajo firmante, matriculado en el Máster en Ingeniería en Informática de la Facultad de Informática, autoriza a la Universidad Complutense de Madrid (UCM) a difundir y utilizar con fines académicos, no comerciales y mencionando expresamente a su autor el presente Trabajo Fin de Máster: ``TITULO DEL TRABAJO'', realizado durante el curso académico CURSO bajo la dirección de DIRECTORES en el Departamento de XXXXXXXXXXXXXXXXXXXXXXXX, y a la Biblioteca de la UCM a depositarlo en el Archivo Institucional E-Prints Complutense con el objeto de incrementar la difusión, uso e impacto del trabajo en Internet y garantizar su preservación y acceso a largo plazo.

\vspace{5cm}

% +--------------------------------------------------------------------+
% | On the line below, replace "Enter Your Name" with your name
% | Use the same form of your name as it appears on your title page.
% | Use mixed case, for example, Lori Goetsch.
% +--------------------------------------------------------------------+
\begin{center}
	\large Nombre Del Alumno\\
	
	\vspace{0.5cm}
	
	% +--------------------------------------------------------------------+
	% | On the line below, replace Fecha
	% |
	% +--------------------------------------------------------------------+
	
	\today\\
	
\end{center}

% +--------------------------------------------------------------------+
% | Dedication Page (Optional)
% +--------------------------------------------------------------------+

\chapter*{Dedicatoria}


Texto de la dedicatoria...
% +--------------------------------------------------------------------+
% | Acknowledgements Page (Optional)                                   |
% +--------------------------------------------------------------------+

\chapter*{Agradecimientos}

Texto de los agradecimientos












\chapter*{Resumen}

Resumen en español del trabajo


\section*{Palabras clave}
   
\noindent Máximo 10 palabras clave separadas por comas

   



\begin{otherlanguage}{english}
\chapter*{Abstract}

In this project we will study the Deep Reinforcement Learning field in order to achieve an stable application for classic control problems. To do this we will investigate its fundamentals: Reinforcement Learning and Neural Networks, learning which are their strengths and weaknesses. Finally, we will merge both to progressivly improve our agent's performance and stability.

In order to gain a better insight we will personally implement the agents and algorithms. All of this will be tested through the popular framework OpenAI Gym. 

This project's source code can be found in the repository:

\url{https://github.com/delcanovega/TFG-DRL}

\section*{Keywords}

\begin{itemize}
    \item Reinforcement Learning
    \item Q-Learning
    \item Markov decision process
    \item Neural Networks
    \item Deep Reinforcement Learning
    \item DeepMind 
    \item OpenAI
\end{itemize}
   




% Si el trabajo se escribe en inglés, comentar esta línea y descomentar
% otra igual que hay justo antes de \end{document}
\end{otherlanguage}

\ifx\generatoc\undefined
\else
%---------------------------------------------------------------------
%
%                          TeXiS_toc.tex
%
%---------------------------------------------------------------------
%
% TeXiS_toc.tex
% Copyright 2009 Marco Antonio Gomez-Martin, Pedro Pablo Gomez-Martin
%
% This file belongs to TeXiS, a LaTeX template for writting
% Thesis and other documents. The complete last TeXiS package can
% be obtained from http://gaia.fdi.ucm.es/projects/texis/
%
% This work may be distributed and/or modified under the
% conditions of the LaTeX Project Public License, either version 1.3
% of this license or (at your option) any later version.
% The latest version of this license is in
%   http://www.latex-project.org/lppl.txt
% and version 1.3 or later is part of all distributions of LaTeX
% version 2005/12/01 or later.
%
% This work has the LPPL maintenance status `maintained'.
% 
% The Current Maintainers of this work are Marco Antonio Gomez-Martin
% and Pedro Pablo Gomez-Martin
%
%---------------------------------------------------------------------
%
% Contiene  los  comandos  para  generar los  índices  del  documento,
% entendiendo por índices las tablas de contenidos.
%
% Genera  el  índice normal  ("tabla  de  contenidos"),  el índice  de
% figuras y el de tablas. También  crea "marcadores" en el caso de que
% se esté compilando con pdflatex para que aparezcan en el PDF.
%
%---------------------------------------------------------------------


% Primero un poquito de configuración...


% Pedimos que inserte todos los epígrafes hasta el nivel \subsection en
% la tabla de contenidos.
\setcounter{tocdepth}{2} 

% Le  pedimos  que nos  numere  todos  los  epígrafes hasta  el  nivel
% \subsubsection en el cuerpo del documento.
\setcounter{secnumdepth}{3} 


% Creamos los diferentes índices.

% Lo primero un  poco de trabajo en los marcadores  del PDF. No quiero
% que  salga una  entrada  por cada  índice  a nivel  0...  si no  que
% aparezca un marcador "Índices", que  tenga dentro los otros tipos de
% índices.  Total, que creamos el marcador "Índices".
% Antes de  la creación  de los índices,  se añaden los  marcadores de
% nivel 1.

\ifpdf
   \pdfbookmark{Índices}{indices}
\fi

% Tabla de contenidos.
%
% La  inclusión  de '\tableofcontents'  significa  que  en la  primera
% pasada  de  LaTeX  se  crea   un  fichero  con  extensión  .toc  con
% información sobre la tabla de contenidos (es conceptualmente similar
% al  .bbl de  BibTeX, creo).  En la  segunda ejecución  de  LaTeX ese
% documento se utiliza para  generar la verdadera página de contenidos
% usando la  información sobre los  capítulos y demás guardadas  en el
% .toc
\ifpdf
   \pdfbookmark[1]{Tabla de Contenidos}{tabla de contenidos}
\fi

\cabeceraEspecial{\'Indice}

\tableofcontents

\newpage 

% Índice de figuras
%
% La idea es semejante que para  el .toc del índice, pero ahora se usa
% extensión .lof (List Of Figures) con la información de las figuras.

\ifpdf
   \pdfbookmark[1]{Índice de figuras}{indice de figuras}
\fi

\cabeceraEspecial{\'Indice de figuras}

\listoffigures

\newpage

% Índice de tablas
% Como antes, pero ahora .lot (List Of Tables)

\ifpdf
   \pdfbookmark[1]{Índice de tablas}{indice de tablas}
\fi

\cabeceraEspecial{\'Indice de tablas}

\listoftables

\newpage

% Variable local para emacs, para  que encuentre el fichero maestro de
% compilación y funcionen mejor algunas teclas rápidas de AucTeX

%%%
%%% Local Variables:
%%% mode: latex
%%% TeX-master: "../Tesis.tex"
%%% End:

\fi

% Marcamos el  comienzo de  los capítulos (para  la numeración  de las
% páginas) y ponemos la cabecera normal
\mainmatter

\pagestyle{fancy}
\restauraCabecera

%%%%%%%%%%%%%%%%%%%%%%%%%%%%%%%%%%%%%%%%%%%%%%%%%%%%%%%%%%%%%%%%%%%%%%%%%%%
% Si el TFM se escribe en ingles, comentar las siguientes líneas 
% porque no hace falta incluir nuevamente la Introducción en inglés
\begin{otherlanguage}{english}
\chapter{Introduction}

\chapterquote{The young man or the young woman must possess or teach himself, train himself, in infinite patience, which is to try and to try and to try until it comes right}{William Faulkner}


\section{Motivation}

\textbf{Artificial Intelligence} is one of the computation fields that most interest has generated, both among experts and general public, who is more interested in the leisure side of it. Been able to grant a machine the hability to reason and perform functions only associated with the human intellect has always been considered Sci-Fi. And nonetheless this very idea has experienced a huge technological leap in the last years.

To better understand this concept from its roots, we need to reference the \textbf{Turing Test}, formulated by Alan Turing \citep{Turing1950-TURCMA}, which consists in performing a series of questions to a machine. The test is considered passed if the evaluator cannot discern if the answers were given by a human or a machine.

Trying to define Artificial Intelligence quickly leads us to concepts like \textit{thought process} or \textit{reasoning}, which end up driving to more complex ones like \textit{behaviour}. From this ideas we can find other definitions, classified in the matrix \ref{fig:tabla_IA_EN}.

\figura{Bitmap/Introduccion/tabla_IA_EN}{width=1\textwidth}{fig:tabla_IA_EN}%
       {Artificial Intelligence definitions, \citet{Russell:2009:AIM:1671238}}

% TODO JCA TRADUCIR UNA VEZ AMPLIADO
En base a estas clasificaciones podríamos diferenciar dos corrientes de interpretación:
\begin{itemize}
    \item Una visión empírica (columna izquierda) con el ser humano como centro de la investigación.
    \item Una visión racionalista (columna derecha) que involucra una combinación de matemáticas e ingeniería.
\end{itemize}

Lots of experts have studied both approaches in different ways. We will focus in the empiric approach, pursuing the goal of our \textit{agent} being able to take the ``right choices'' with its available knowledge. In order to achieve this, we will study an Artificial Intelligence's field in particular, called Reinforcement Learning. We will also combine this field with Neural Networks, resulting in Deep Reinforcement Learning.


\section{Goals}

\begin{enumerate}
    \item Understand why is Reinforcement Learning different from other Machine Learning methods, and in which situations it can be applied.
    \item Study the fundamentals of Reinforcement Learning, understanding its components, implementations and limits.
    \item Test what we have learned with practical simulations, on which we will implement Reinforcement Learning algorithms, and study the results.
    \item Dive into the Deep Learning field, where we will see the fundamentals of Neural Networks.
    \item Learn how is it possible to combine Neural Networks with Reinforcement Learning techniques, trying to avoid the limitations of both. It will be a journey where, step by step, we will find solutions to the problems that arise, until we come with a stable Deep Reinforcement Learning solution.
    \item After every milestone we will evaluate the results, contrasting if our solutions improve stability and performance.
\end{enumerate}


\section{Structure of the memory}

Our project intercalates theorical chapters with practical applications of what we have seen, resulting in two big blocks: One about Reinforcement Learning and other about Deep Reinforcement Learning.

\begin{itemize}
    \item \textbf{Chapter 1, Introduction.} Motivation and goals of our proyect.
    \item \textbf{Chapter 2, Reinforcement Learning.} The chapter begins with a comparison between different Machine Learning techniques. Afterwards, the needed Reinforcement Learning theorical background is provided. Finally, we will thoroughly explain Q-Learning, one of the most commonly used Reinforcement Learning algorithms.
    \item \textbf{Chapter 3, Q-Learning in action.} We will introduce OpenAI Gym, the framework used in our tests. Later we will apply the acquired Reinforcement Learning knowledge into CartPole, an environment that will allow us to measure results and experience Reinforcement Learning's limitations.
    \item \textbf{Chapter 4, Neural Networks and Q-Learning.} We will give the necessary theorical background to understand Neural Networks, followed by how is possible to combine them with Reinforcement Learning techniques, obtaining the so called Deep Q-Networks.
    \item \textbf{Chapter 5, DQNs in action.} We will solve CartPole again, this time applying the new learned approaches. Then we will face MountainCar, a new and challenging environment for our agent.
    \item \textbf{Chapter 6, Conlusions.} Summary of everything we have achieved so far. Lessons learned, highlights and future work.    
\end{itemize}

\end{otherlanguage}
\addtocounter{chapter}{-1} 
%%%%%%%%%%%%%%%%%%%%%%%%%%%%%%%%%%%%%%%%%%%%%%%%%%%%%%%%%%%%%%%%%%%%%%%%%%%

\chapter{Introducción}
\label{cap:introduccion}

\chapterquote{Los jóvenes deben enseñarse a sí mismos, entrenarse a sí mismos, con infinita paciencia, intentarlo una y otra y otra vez hasta que salga bien}{William Faulkner}


\section{Motivación}

La \textbf{inteligencia artificial} es una de las ramas de la computación que más interés ha generado, tanto entre expertos de la materia como en otro tipo de público, más interesado en la parte lúdica de este concepto. Dotar a una máquina de la capacidad de realizar funciones asociadas sólo al intelecto humano ha sido siempre considerado ciencia ficción. Y sin embargo esta misma idea ha supuesto un enorme avance tecnológico en los últimos años.

Para conocer mejor este concepto desde sus orígenes, es necesario hacer referencia al \textbf{test de Turing}, propuesto por Alan \citet{Turing1950-TURCMA}. Éste consiste en realizar una serie de preguntas a un ente y el test se considerará superado si el interrogador no es capaz de discernir si las respuestas provienen de una máquina o una persona.

Intentar definir la inteligencia artificial nos lleva directamente a conceptos como el \textit{proceso del pensamiento} o el \textit{razonamiento}, los cuales terminan por conducir a otros más complejos como es el \textit{comportamiento}. A partir de estas ideas podemos encontrar otras definiciones, clasificadas en la matriz~\ref{fig:tabla_IA}.

\figura{Bitmap/Introduccion/tabla_IA}{width=1\textwidth}{fig:tabla_IA}%
       {Definiciones de inteligencia artificial, \citet{Russell:2009:AIM:1671238}}

En base a estas clasificaciones podríamos diferenciar dos corrientes de interpretación:
\begin{itemize}
    \item Una visión empírica (columna izquierda) con el ser humano como centro de la investigación. Involucra principalmente observaciones e hipótesis sobre cómo debería comportarse un humano. Esta vertiente tiene cierta relevancia a día de hoy, sobre todo gracias al auge de proyectos como el coche autónomo. Estos agentes autónomos, puestos en una situación límite, podrían verse obligados a decidir entre dos opciones que pongan en peligro vidas humanas. En este caso, deberíamos basar nuestra respuesta en qué elegiría un conductor real.
    \item Una visión racionalista (columna derecha), que implica una combinación de matemáticas e ingeniería. En esta vertiente se engloban proyectos como los asistentes de voz o los robots de Boston Dynamics. Ninguno de ellos necesita valorar las órdenes e información que se les provee a un nivel ``humano''. Los asistentes no necesitan ser asertivos, sólo necesitan ser capaces de procesar la información de forma correcta para actuar consecuentemente, mientras que los robots sólo deben ser capaces de aprender a adaptarse a cualquier terreno con el fin de realizar la función que se les ha encomendado.
\end{itemize}

Multitud de expertos han abordado ambos acercamientos de distintas formas. Nosotros buscaremos que nuestro \textit{agente} tome las ``decisiones correctas'' en función del conocimiento que posea. En particular buscaremos que un agente sin ningún conocimiento previo sea capaz de aprender a realizar tareas sencillas mediante la interacción constante con el entorno, quien le proporcionará nuevas experiencias de las que extraer conocimiento.

Esta forma de aprendizaje es un campo de la inteligencia artificial llamado \textbf{aprendizaje por refuerzo}. También investigaremos una combinación del mismo con \textbf{redes neuronales}, resultando en el llamado \textbf{aprendizaje por refuerzo profundo}. La popularidad de estas técnicas no ha parado de crecer en los últimos años. Koray Kavukcuoglu, director de investigación en Deepmind, explica su potencial de la siguiente forma:

\begin{quote}
    El aprendizaje por refuerzo es un sistema muy general para aprender a tomar decisiones secuenciales. Por otra parte, el aprendizaje profundo es el mejor conjunto de algoritmos disponibles para aprender representaciones. Combinar estos dos modelos diferentes es la mejor opción que tenemos disponible para lograr buenas representaciones de estados en tareas complejas, no sólo para resolver juegos sencillos si no también complicados problemas reales.
\end{quote}

En definitiva, el potencial y las posibilidades de esta metodología convierten al aprendizaje por refuerzo profundo en un campo muy interesante, que puede que nos lleve un paso más cerca al mundo de la inteligencia artificial general.


\section{Objetivos}

\begin{enumerate}
    \item Comprender en qué es el aprendizaje por refuerzo distinto a otros métodos de aprendizaje automático, y en qué situaciones puede ser usado.
    \item Estudiar los fundamentos del aprendizaje por refuerzo, entendiendo sus características, componentes y limitaciones.
    \item Poner a prueba lo aprendido con simulaciones prácticas, en las que implementemos algoritmos de aprendizaje por refuerzo y estudiemos sus resultados.
    \item Adentrarnos en el campo del aprendizaje profundo, donde veremos los fundamentos de las redes neuronales.
    \item Estudiar cómo es posible combinar las redes neuronales y el aprendizaje por refuerzo con el objetivo de sortear las limitaciones de ambos. Será un camino en el que, paso a paso, encontraremos soluciones a los problemas que surjan hasta lograr un modelo estable de aprendizaje por refuerzo profundo.
    \item En cada hito del camino evaluaremos los resultados obtenidos, para comprobar que las soluciones mejoran en rendimiento y estabilidad.
\end{enumerate}


\section{Estructura de la memoria}

Nuestro trabajo intercala capítulos teóricos con aplicaciones prácticas de lo visto en dichos capítulos, resultando en dos bloques diferenciables: el primero sobre aprendizaje por refuerzo y el segundo sobre aprendizaje por refuerzo profundo.

\begin{itemize}
    \item \textbf{Capítulo 1, Introducción.} Motivación y objetivos de nuestro proyecto.
    \item \textbf{Capítulo 2, Aprendizaje por Refuerzo.} El capítulo comienza con una comparación del aprendizaje por refuerzo con otros métodos de aprendizaje automático. Después, se proporciona toda la base teórica necesaria para comprender el aprendizaje por refuerzo. Para terminar, explicaremos en profundidad Q-Learning, el algoritmo que utilizaremos en nuestras pruebas.
    \item \textbf{Capítulo 3, Q-Learning en acción.} Introduciremos OpenAI Gym, la herramienta utilizada durante nuestras pruebas. Aplicaremos los conocimientos de aprendizaje por refuerzo en CartPole, un problema que nos permitirá evaluar resultados y limitaciones.
    \item \textbf{Capítulo 4, Redes Neuronales y Q-Learning.} Proporcionaremos el marco teórico necesario para comprender las redes neuronales. Después explicaremos cómo es posible aplicarlas a técnicas de aprendizaje por refuerzo, obteniendo las DQN.
    \item \textbf{Capítulo 5, DQNs en acción.} Volveremos a resolver CartPole y nos enfrentaremos a MountainCar, un problema que presenta nuevos retos para nuestro agente.
    \item \textbf{Capítulo 6, Conclusiones.} Síntesis de todo lo aprendido durante el camino. Puntos clave de las distintas etapas, limitaciones y oportunidades de cara al futuro.
    
\end{itemize}

\chapter{Aprendizaje por Refuerzo}
\label{cap:reinforcementLearning}

\chapterquote{El comportamiento es modelado y mantenido por sus consecuencias}{B. F. Skinner}


\section{Aprendiendo a aprender}

Una de las funciones humanas de las que necesitaremos dotar a nuestra máquina en busca de este \textit{rendimiento ideal} es el \textbf{aprendizaje}.

El campo del \textbf{aprendizaje automático} (o \textit{Machine Learning}) se encarga de esta tarea, a través de la generalización y la búsqueda de patrones en experiencias pasadas. En función del tipo de \textit{realimentación} obtenido existen diferentes técnicas de aprendizaje, diseñadas para distintos casos y objetivos:
\begin{itemize}
    \item \textbf{Aprendizaje supervisado} (\textit{Supervised Learning}): el agente observa una serie de ejemplos de entradas y salidas, aprendiendo una función que es capaz de asignar a una entrada su salida correspondiente. Se llama \textbf{supervisado} porque esta serie de ejemplos, llamada conjunto de entrenamiento, debe estar correctamente clasificada desde un primer momento. Podría decirse que el agente aprende en base a estas experiencias, hasta que llegado un punto es capaz de clasificar una entrada completamente nueva, de modo que la exactitud de esta clasificación dependerá del entrenamiento recibido.
    \item \textbf{Aprendizaje no supervisado} (\textit{Unsupervised Learning}): a diferencia del supervisado, en el aprendizaje no supervisado los ejemplos no cuentan con una etiqueta que los clasifica inequívocamente. En su lugar el agente busca patrones en el conjunto de entrada, intentando extraer características comunes de sus elementos. Uno de los usos más comunes del aprendizaje no supervisado es la agrupación o \textbf{clustering}: la unión de ejemplos como grupos o \textit{clústeres} que comparten características comunes. Por ejemplo, la agrupación de películas con características similares, de modo que si a un usuario le resultan interesantes varias de un mismo \textit{clúster}, es muy probable que también disfrute de otros elementos de esa misma agrupación. 
    \item \textbf{Aprendizaje por refuerzo} (\textit{Reinforcement Learning}): el agente aprende a partir de una serie de refuerzos, recompensas si son positivos y penalizaciones en caso de ser negativos. Por ejemplo, cuando una mascota cumple una orden y recibe una galleta como recompensa, o un músico desafina en directo y recibe un abucheo por parte del público. Además, el agente puede ``pensar'' a largo plazo, de forma que su comportamiento le permita conseguir una recompensa mayor en un futuro, así como adaptarse a entornos totalmente nuevos para él. Esta técnica de aprendizaje será el punto de partida de nuestro proyecto y explicaremos su funcionamiento a lo largo de este capítulo. 
\end{itemize}


\subsection{¿Por qué aprendizaje por refuerzo?}

Como muchos otros campos pertenecientes a la rama de inteligencia artificial, el aprendizaje por refuerzo \citep{Watkins1992} se inspira en estudios sobre el comportamiento en humanos y animales. De esta forma toma su base en la psicología conductista \citep{Skinner1953}, en la que un algoritmo decide si una acción tomada ha sido positiva o negativa, y la refuerza consecuentemente.

Pero antes de profundizar en ello, analicemos por qué es el aprendizaje por refuerzo una opción interesante y por qué será la que utilicemos en nuestro proyecto.

Echemos un vistazo a las técnicas de aprendizaje vistas en la sección anterior. ¿Qué es interesante de cada una de ellas? Empecemos por el \textit{aprendizaje no supervisado}: este es quizá el caso más sencillo, ya que lo único que necesitaremos es algo de lo que vivimos rodeados, \textbf{información}. En internet disponemos de grandes cantidades de ella. Podemos servirnos de las APIs de servicios de música, series, entretenimiento, organismos públicos... En la mayoría de los casos, únicamente tendremos que normalizar y limpiar la información antes de alimentarla a nuestro algoritmo. Sin embargo, el \textit{aprendizaje supervisado} necesita una entrada de datos normalizada. A la hora de presentar nueva información necesitamos que esté correctamente clasificada y, aunque en internet hay conjuntos de entrenamiento de gran calidad y variedad, podría darse el caso de que ninguno se adapte a nuestras necesidades. En esta situación podríamos construir uno nosotros mismos, pero la tarea sería larga y tediosa.

El caso del \textit{aprendizaje por refuerzo} es distinto. El agente no necesita grandes cantidades de casos de prueba, sino un entorno con el que interactuar por sí mismo, de forma que los casos de prueba se generen de forma automática. La manera más rápida de hacernos con un entorno en el que dejar a nuestro agente y verlo actuar es, claramente, simulándolo. Una simulación es la forma más sencilla de controlar las condiciones y el progreso de nuestro experimento, además de quitarnos otros problemas que podrían darse en un entorno real: variables que no se tuvieron en cuenta, cambios inesperados en dicho entorno, o cualquier tipo de error no controlado... El mundo real, en general.

Un dominio muy interesante sobre el que realizar simulaciones de aprendizaje por refuerzo son los videojuegos \citep{Rodriguez2018}. Durante los últimos años han tenido bastante éxito debido a varios motivos; entre ellos, porque son entornos controlados y reproducibles. De esta forma, no habrá ninguna diferencia en nuestras pruebas más allá de las propias acciones que realice nuestro agente sobre la simulación, disponiendo así de unos resultados fiables y fácilmente comparables. Además, los videojuegos resultan ser un dominio accesible, de diversa complejidad y ya preparados para nuestro tipo de agente: muchos de ellos disponen de un sistema de puntuación que sirve perfectamente como refuerzo, ya sea por recolección de objetos, número de enemigos derrotados, tiempo... Nuestro agente podrá utilizar fácilmente sus experiencias pasadas para decidir cuál es la mejor decisión a tomar en cada nueva partida.

Podemos ver el funcionamiento del aprendizaje por refuerzo en el pseudocódigo \ref{code:reinf-learning}:

\begin{minipage}{0.9\linewidth}%
    \begin{lstlisting}[frame=tb, caption=Pseudocódigo Aprendizaje por Refuerzo, mathescape, label={code:reinf-learning}]
    
    Initialize state
    
    repeat (for each step of the episode):
        select action
        perform action; observe reward and next-state
        state $\leftarrow$ next-state
    \end{lstlisting}%
\end{minipage}

Pero hay mucho más. En las próximas secciones describiremos los distintos componentes de nuestros problemas y las técnicas usadas sobre los diferentes dominios para su resolución, además de cuestiones más complejas: cómo definir la interacción entre agente y entorno, cómo representar el entorno de una forma eficiente o cómo discernir si una acción tomada ha sido buena o mala.


\section{Elementos del Aprendizaje por Refuerzo}

A lo largo de este capítulo el término ``agente'' ha aparecido en repetidas ocasiones. Un \textbf{agente} no es más que el sujeto de nuestro experimento. Si nos encontrásemos en el mundo real, un agente podría ser un ratón intentando salir de un laberinto. En nuestro contexto software, llamamos agente a un programa capaz de interactuar con el \textbf{entorno}, aprender de él y \textbf{tomar decisiones}.

El \textbf{entorno} es el otro elemento fundamental del aprendizaje por refuerzo. Es lo que proporciona información al agente, quien la usará para tomar una decisión. Esta decisión en forma de acción provocará cambios en el entorno, los cuales servirán para que éste proporcione la recompensa consecuente al agente.

Más allá del agente y el entorno, existen cuatro subelementos en un sistema de aprendizaje por refuerzo: una \textit{política}, una \textit{recompensa}, una \textit{función de utilidad} y un \textit{modelo} del entorno.

La \textbf{política} define la forma en la que el agente afronta un determinado problema. Por ejemplo, si nos encontrásemos a un enemigo en un videojuego, una política válida sería enfrentarnos a él. Huir podría ser otra política perfectamente válida. Generalizando esto, la política es una función que asocia el estado en que se encuentra el entorno y la acción que tomará el agente.

La \textbf{recompensa} es la forma por la cual el agente obtiene realimentación del entorno. Después de cada acción tomada, el agente obtiene un valor numérico que le ayude a saber si dicha acción fue buena o mala. A través de la recompensa se modela y modifica, de cara a futuras acciones, la política del agente, pudiendo éste aprender a maximizar dicha recompensa a lo largo de una simulación y aprender a jugar mejor.

Pero la recompensa no es suficiente para lograr un comportamiento óptimo por parte del agente. Es aquí donde entra en juego la \textbf{función de utilidad}. Si nos paramos a pensarlo, existen acciones que no nos proporcionan una gran recompensa por si solas, pero facilitan el camino a otras que generan una recompensa aún mayor. En una partida de \textit{Tetris} podríamos apilar piezas a un lado mientras esperamos la ficha en forma de barra vertical, completando así cuatro líneas en un solo movimiento y obteniendo muchos más puntos que si hubiésemos completado una única línea cuatro veces seguidas. Desde un punto de vista más teórico, podemos entender la utilidad como la recompensa máxima a la que se podría llegar desde un estado. Es decir, una recompensa a largo plazo.

Dependiendo del problema que afrontemos tendremos que ajustar la importancia que damos a la recompensa y a la utilidad, para que nuestro agente sea capaz de lograr buenos resultados.

Llegamos al último elemento por definir, el \textbf{modelo}. El modelo es una representación del entorno que el agente construye y mediante la cual es capaz de aprender a optimizar la política, así como a predecir las transiciones y recompensas obtenidas. No todos los agentes usan un modelo, también existen aquellos \textit{libres de modelo} en los cuales el agente depende únicamente de la prueba y error. 

Podemos apreciar la relación de estos elementos en la figura \ref{fig:rl-diagram}.

\figura{Bitmap/AprendizajePorRefuerzo/MDP-Diagram}{width=0.6\textwidth}{fig:rl-diagram}{Modelo de interacción del aprendizaje por refuerzo}


\subsection{Problemas de Decisión de Markov}
\label{sec:mdp}

El Proceso de Decisión de Markov \citep{Puterman1994}, al cual nos referiremos como $MDP$, llamado así por el matemático ruso Andrei Markov, es un modelo matemático mediante el cual podemos modelar la resolución de cierta clase de problemas relacionados con la toma de decisiones. Para comprender mejor cómo funciona, definiremos los elementos que componen nuestro MDP:

\begin{itemize}
    \item \textbf{S}: conjunto de estados finitos en los que nos podemos encontrar en un determinado momento.
    \item \textbf{A}: conjunto de acciones que pueden ser tomadas.
    \item \textbf{Modelo de transición}: función que define el estado futuro (\textbf{s'}) basándose en el estado actual (\textbf{s}) y la acción tomada en dicho estado (\textbf{a}). Es representado mediante la función \textbf{P(s, a, s’)}. El modelo de transición puede ser de dos tipos:
    \begin{itemize}
        \item Determinista: ejecutar una acción en un estado determinado siempre lleva al mismo estado siguiente.
        \item Probabilista: representa la probabilidad de que un estado siguiente sea alcanzado si se toma una acción en un estado.
    \end{itemize}
    \item \textbf{Recompensa}: “puntuación” positiva o negativa que se obtiene al tomar una acción en un determinado estado. Se representa mediante la función \textbf{R(r | s, a)}.
\end{itemize}

Podemos relacionar todos estos elementos de modo que para un estado actual (\textbf{s}) se toma determinada acción (\textbf{a}), llevándonos así a un estado siguiente (\textbf{s'}) definido en el modelo de transición, y obteniendo una recompensa (\textbf{r}) asociada a la toma de dicha decisión en ese estado. Cabe destacar que el MDP relaciona estos elementos de forma que, para obtener los resultados futuros, sólo se toma en cuenta el estado inmediatamente anterior junto la última acción escogida; es decir, no se guarda memoria de las transiciones anteriores. 

Pongamos el ejemplo de un laberinto, como podemos ver representado en la figura \ref{fig:mdp-01}. Nuestro conjunto de \textbf{estados} serían todas las posibles posiciones dentro del laberinto donde podemos situarnos y las \textbf{acciones} serían las direcciones en las que nos podemos mover en cada posición del laberinto. Nuestro \textbf{modelo de transición} estaría definido de modo que si nos encontramos en una posición $[x, y]$ y decidimos tomar la acción de movernos hacia arriba, nuestra siguiente posición (estado) sería $[x, y+1]$. Finalmente podríamos definir la \textbf{recompensa} como una puntuación de +100 si conseguimos llegar a la salida, 0 si conseguimos movernos a una posición buena dentro del laberinto o -1 si nos chocamos contra una pared.

\figura{Bitmap/AprendizajePorRefuerzo/MDP-Maze.png}{width=0.6\textwidth}{fig:mdp-01}{MDP como laberinto}

El hecho de que un agente tome una acción y haga cambiar el estado del entorno lo llamamos \textbf{paso}. Un paso está compuesto por el estado en el que el agente se encuentre, la acción tomada y la recompensa obtenida: $$paso = (estado, accion, recompensa)$$


Los pasos se suceden hasta que el entorno llega a un estado de “finalizado”, de forma que todos los pasos recorridos hasta dicho estado componen un \textbf{episodio}. Un entorno puede llegar a su estado “finalizado” si perdemos el juego o, si por el contrario, lo ganamos. Podemos diferenciar 2 tipos de entornos y sus diferentes objetivos para conseguir ganar:

\begin{itemize}
    \item \textbf{Entornos finitos}: finalizan cuando el agente consigue alcanzar un estado objetivo o si el número de pasos que componen cada episodio hasta alcanzarlo es limitado. Un ejemplo de entorno finito sería el laberinto antes planteado, en el cual existe el objetivo de llegar a la salida.
    \item \textbf{Entornos infinitos}: son aquellos entornos que carecen de un estado objetivo. De este modo, el objetivo que se plantea en este caso es el de mantenerse de forma infinita en un estado bueno. Dicho de una forma mucho más simple, el agente debe aprender a no perder. 
\end{itemize}

Pero, ¿cómo sabemos o cómo decidimos qué acción es mejor tomar en cada estado? Éste es el objetivo de nuestro agente, el de encontrar una política óptima que nos ayude a tomar buenas decisiones acerca de qué acciones tomar en cada momento basándonos en el estado actual y conseguir maximizar nuestra “recompensa futura” (\textit{future return}).

Entendemos como \textbf{recompensa futura} la recompensa a largo plazo que el agente puede acumular siguiendo su política a partir del estado actual. De este modo, el objetivo de nuestro agente es optimizar su política para buscar obtener una mayor recompensa a la larga, aunque eso conlleve obtener recompensas inmediatas más pequeñas. Este concepto se conoce como \textbf{recompensas retrasadas}, ya que no las obtendremos instantáneamente tras ejecutar una acción, sino después de la sucesión de una serie de acciones.

Gracias a esto conseguimos que nuestra política aprenda a pensar, no sólo en qué acción tomar en este momento, sino en qué acción tomar después de la misma. De esta manera, un inconveniente que podemos encontrar es que nuestra política piense demasiado a largo plazo y le lleve un elevado número de pasos encontrar una recompensa que merezca la pena, de modo que nuestra recompensa futura diverge. Para evitar que esto suceda, introduciremos los factores de descuento.

Un \textbf{factor de descuento} es un pequeño valor que se descuenta de la recompensa con el fin de evitar que, como acabamos de ver, nuestra política tome demasiados episodios antes de encontrar una recompensa positiva, de forma que se equilibre la búsqueda de recompensas cercanas en el tiempo y se maximice la recompensa total, del mismo modo que evite que la recompensa futura tienda a infinito en los entornos infinitos. Teniendo esto en cuenta, podemos definir nuestra función de recompensa como:

$$R_{t} = \sum^{T}_{k = 0} \gamma^{t}r_{t + k + 1}$$

El factor de descuento, \( \gamma \), representa cuánto se descuenta de la recompensa en cada paso, y su valor siempre se encuentra entre 0 y 1, de forma que valores más altos corresponden a una penalización menor. Los factores de descuento más comúnmente usados se encuentran entre 0,97 y 0,99.

Esta búsqueda del balance entre las recompensas cercanas y futuras está muy relacionado con el problema de \textbf{Exploración vs. Explotación}. Nuestro agente contará con un hiperparámetro \textit{exploración}, el cual determina con qué frecuencia el agente decide tomar una acción aleatoria (exploración) en lugar de aquella que le proporcione el valor más prometedor (explotación), con la esperanza de encontrar un nuevo estado en el que nunca haya estado y que pueda resultar más beneficioso. En otras palabras, el agente decide si sacar provecho de lo ya aprendido o arriesgarse e intentar aprender algo mejor.

Siguiendo con esta idea, nuestra estrategia para equilibrar el dilema de la explotación vs. exploración se basa en el llamado \textbf{\( \epsilon \)-Greedy} \citep{Tokic:egreedy}. \( \epsilon \)-Greedy es una estrategia que implica tomar una decisión en cada paso para tomar la acción registrada por el agente con una mayor recompensa o tomar una acción al azar. La probabilidad de que el agente tome una acción aleatoria se rige por el parámeto epsilon (\( \epsilon \)). Podemos hacer un acercamiento de esta idea en el código \ref{code:e_greedy}.

De esta manera, se busca obtener un balance entre la exploración y explotación, con el objetivo de evitar que se explore demasiado, lo que conllevaría que nuestro agente no optimizase lo suficiente su política, ni que explote demasiado, lo que significaría caer en un mínimo local.

Para ayudarnos a conseguir este balanceo entre exploración y explotación, lo que haremos será reducir dinámicamente el valor de epsilon, de manera que al principio explore más opciones y después explote los conocimientos adquiridos para terminar convergiendo a una política concreta.

\begin{minipage}{0.9\linewidth}%
    \begin{lstlisting}[frame=tb, caption=Pseudocódigo \( \epsilon \)-Greedy, label={code:e_greedy}]

    if random() < epsilon:
        # explore
        choose a random action

    else:
        # exploit
        choose best action for current state
    \end{lstlisting}%
\end{minipage}

\section{Q-Learning}

Hasta ahora hemos hablado de múltiples conceptos y elementos: agente, entorno, recompensa, política... Llega el momento de aplicarlos de manera conjunta para conseguir una técnica de aprendizaje tangible. \textbf{Q-Learning} es una técnica de Aprendizaje por Refuerzo que tiene como objetivo enseñar al agente qué acción es mejor tomar en cada circunstancia. Se trata de un algoritmo \textit{off-policy} con \textit{diferenciación temporal} que busca encontrar una función acción-utilidad que nos dirá cómo de bueno es ejecutar una acción es un determinado estado. Los algoritmos \textit{off-policy} son capaces de encontrar una política óptima independientemente de la politica utilizada por el agente para elegir acciones, siempre que pase por todos los estados suficientes veces \citep{PooleMackworth17}.

El algoritmo de Q-Learning cuenta con una tabla-Q con los estados posibles contemplados a partir del MDP, en la que se van almacenando las sumas de las posibles recompensas futuras. También conocidas como \textbf{valores-Q}, se predicen o actualizan usando el valor-Q del estado futuro $s'$ y la acción $a'$ que más utilidad produzca. Siendo $Q(s, a)$ el valor de aplicar una acción $a$ en el estado $s$, los valores devueltos por nuestra función-Q serán la utilidad máxima alcanzable de forma:

$$U(s) = \max_{a}Q(s, a)$$

Con \textbf{diferenciación temporal} hacemos referencia al hecho de que, en busca de un aprendizaje correcto, nuestro agente tendrá que tener en cuenta las recompensas actualizadas obtenidas en un momento futuro para actualilzar los valores-Q registrados anteriormente. Reformulando esto, el agente deberá actualizar las estimaciones obtenidas en un momento $Q(s, a)$ con el valor actualizado en pasos siguientes $Q(s+k, a)$. Para poder hacerlo, debemos encontrar un equilibrio entre las recompensas inmediatas y las futuras. Como podemos imaginar, potenciar la importancia de las recompensas directas puede perjudicar el valor de las futuras.

Existen varias fórmulas para calcular la utilidad de los estados. Uno de los métodos más utilizados es a través de la \textbf{Ecuación de Bellman} \citep{Baird1995}, que nos indica que la máxima recompensa futura de tomar una acción es la suma de la recompensa actual y la máxima recompensa futura del siguiente episodio, permitiéndonos así obtener una relación entre valores-Q.

$$Q^*(s_{t}, a_{t}) = r_{t} + \gamma\ max_{a'}\ Q^*(s_{t+1}, a')$$

De este modo, podemos calcular el valor-Q para un par estado-acción, del mismo modo que con cada nuevo episodio podremos actualizar valores-Q previamente calculados con nueva información obtenida.

Q-Learning cuenta con tres hiperparámetros, dos de los cuales ya hemos visto: la \textit{exploración} y el \textit{factor de descuento}, introducido al final de la sección anterior. El tercer hiperparámetro en cuestión corresponde a la \textbf{tasa de aprendizaje}. Este hiperparámetro, que toma valores entre 0 y 1, nos indica cuánta nueva información actualizamos sobre los valores-Q previamente calculados. De este modo, si nuestra tasa de aprendizaje es igual a 0, con cada nuevo episodio no se actualizarán valores-Q con nueva información, y si es igual a 1, los valores previos se sobreescribirán por completo con la nueva información obtenida.

Podemos relacionar esta tasa de aprendizaje ($\alpha$) y la ecuación antes propuesta del siguiente modo:

$$Q^*(s_{t}, a_{t}) = (1-\alpha ) Q(s_{t}, a_{t}) + \alpha [r(s_{t}, a_{t}) + \gamma\ max_{a'}\ Q^*(s_{t+1}, a')]$$

Combinando todos estos factores, podemos definir el pseudocódigo del funcioniamiento de Q-Learning como apreciamos en el Listing \ref{code:q-learning}, de manera que relacionemos todos los hiperparámetros y conceptos vistos hasta ahora. Cabe destacar la variable \textbf{\texttt{env}}, que corresponte al entorno sobre el que estamos trabajando, siendo la función \textbf{\texttt{env.reset()}} la cual nos proporciona un estado inicial aleatorio dentro del entorno, la función \textbf{\texttt{env.step()}} la que simula la ejecución de la acción tomada y devolviendo el estado-siguiente y la recompensa, el booleano \textbf{\texttt{end}} que nos indica si el episodio ha terminado o no, y la variable \textbf{\texttt{next\_max}} que se corresponde al valor-Q máximo en el estado-siguiente.

\begin{minipage}{0.9\linewidth}%
\begin{lstlisting}[frame=tb, language=python, caption=Pseudocódigo Aprendizaje por Refuerzo, label={code:q-learning}]
import numpy as np
import gym

env = gym.make('gym-environment')

# Initialize Q-table, states and actions
states  = env.state_space
actions = env.action_space
q_table = np.zeros((states, actions))

for i in range(episodes):
    # Initialize current-state
    state = env.reset()
    
    while not end:
        choose an action
        
        # perform action
        next_state,reward,end,_=env.step(action)
        
        # update Q-table
        Q'(state,action) = 
            (1 - LEARNING_RATE) * Q(state,action) + 
            LEARNING_RATE * (reward + 
            DISCOUNT_FACTOR * next_max)
        
        state = next_state
\end{lstlisting}%
\end{minipage}

Llegados a este punto, podemos observar cómo se crea la tabla-Q que relacionará los pares estado-acción con su correspondiente valor-Q. En consecuencia, nos encontramos con que esta tabla crece alarmantemente rápido al tener que almacenar todas las combinaciones estado-acción. El número de posibles estados puede ser inabarcable incluso para problemas con un pequeño grado de complejidad, haciendo excesivamente costoso el hecho de visitar todas las experiencias recogidas, así como de actualizarlas, y convirtiendo el problema en algo inmanejable desde un punto de vista computacional.
\chapter{Q-Learning en acción}
\label{cap:q-learning}

En el que descubriremos un aliado en la búsqueda de la AGI, y pondremos a prueba las capacidades del Aprendizaje por Refuerzo.

\chapterquote{Si no tienes esperanza, no encontrarás lo que está más allá de tus esperanzas}{Clemente de Alejandría}

\section{OpenAI}
OpenAI es una iniciativa cuyo objetivo es asegurar que la \textbf{Inteligencia Artificial General} (AGI) beneficie a toda la humanidad. En su web (\citet{OpenAI_charter}) se encuentra el siguiente manifiesto:

\begin{quote}
    La misión de OpenAI es asegurar que la inteligencia artificial general - refiriéndose a sistemas altamente autónomos que superen el rendimiento humano en el trabajo más valioso económicamente - beneficie a toda la humanidad. Intentaremos desarrollar AGI de una forma segura y beneficiosa, pero también consideraremos nuestra misión completa si nuestro trabajo ayuda a otros a conseguir esta meta. 
\end{quote}

En busca de este objetivo, OpenAI sigue tres caminos:
\begin{itemize}
    \item \textbf{Investigación y desarrollo:} sus contribuciones al Aprendizaje Automático como estado del arte durante los últimos años no han sido pocas: desde nuevos algoritmos de Aprendizaje por Refuerzo (\citet{baselines}) hasta demostraciones de cómo una Inteligencia Artificial puede sobrepasar a jugadores expertos (\citet{OpenAI_dota}).
    \item \textbf{Recursos académicos} \citet{spinningup} puestos a disposición de todo aquél que quiera aprender sobre el Aprendizaje por Refuerzo Profundo.
    \item \textbf{Herramientas y plataformas} mediante las cuales facilitar la labor de investigación y pruebas del Aprendizaje Automático.
\end{itemize}

Es nuestra investigación nos serviremos de una de estas plataformas, \textbf{OpenAI Gym}, para experimentar y descubrir los puntos fuertes y débiles de las distintas técnicas y algoritmos de aprendizaje. 

Gym es un conjunto de herramientas para desarrollar y comparar algoritmos de aprendizaje de refuerzo. En él los agentes son capaces de aprender cualquier cosa, desde caminar hasta jugar a juegos como Pong o Pinball.

Para ello la librería te aporta una interfaz sencilla, a través de la cual un entorno de simulación te es provisto. Tú tendrás que aportar el algoritmo, en nuestro caso se tratará del agente que interaccionará con el entorno para aprender de él e intentar lograr unos objetivos.

\section{CartPole}
\begin{quote}
    Un mástil está unido por una bisagra a un carro, el cual se mueve a lo largo de una pista sin rozamiento. El sistema es controlado aplicando una fuerza de +1 o -1 al carro. El péndulo (mástil) comienza en posición vertical, el objetivo es prevenir que caiga. Una recompensa de +1 es otorgada por cada \textit{timestep} que el mástil permanece erguido. El episodio finaliza cuando el mástil se encuentra a más de 15 grados de su posición vertical, o el carro se mueve más de 2.4 unidades del centro.
\end{quote}

Este es el enunciado del primer problema que resolveremos mediante técnicas de Aprendizaje por Refuerzo. La implementación del problema nos viene dada como parte de la librería \textit{OpenAI Gym}. Una vez importado el problema, dispondremos de un entorno con el que interactuar y nuestro objetivo será crear un agente que interaccione con el mismo.

Nuestro conjunto de acciones, como se indica en el enunciado, tiene dos elementos: Aplicar fuerza hacia la derecha o aplicar fuerza hacia la izquierda. El agente deberá decidir cuál de estas dos acciones tomar y se lo comunicará al entorno. Éste responderá con cuatro elementos:

\begin{itemize}
    \item \textbf{Observación} (objeto): Consiste en un array que contiene la información necesaria para describir el estado actual (tras aplicar la última acción). Los valores posibles pueden verse en la tabla \ref{obs-cartpole}.
    \item \textbf{Recompensa} (float): Refuerzo para el agente, +1.0 mientras la simulación continúe.
    \item \textbf{Fin} (boolean): Señal que indica si la simulación ha concluido, bien porque el mástil ha caído o porque lo hemos mantenido en pie durante el tiempo suficiente.
    \item \textbf{Info} (diccionario): Información de diagnóstico útil para la depuración del agente. No obstante, esta información no debe ser usada por el agente para aprender.
\end{itemize}
\begin{table}[]
    \centering
    \begin{tabular}{|l|l|l|l|}
    \hline
    \textbf{Num} & \textbf{Observación} & \textbf{Min} & \textbf{Max} \\ \hline
    0            & Posición del carro   & -4.8         & 4.8          \\ \hline
    1            & Velocidad del carro  & -Inf         & Inf          \\ \hline
    2            & Ángulo del poste     & -24º         & 24º          \\ \hline
    3            & Velocidad del poste  & -Inf         & Inf          \\ \hline
    \end{tabular}
    \caption{Observación del entorno para CartPole}
    \label{obs-cartpole}
\end{table}

Como podemos ver a través de las observaciones, este es uno de esos problemas en los que el número de estados distintos es potencialmente infinito, debido a que estamos tratando con valores continuos. Por ello tendremos que encontrar una buena \textbf{función de discretización} que simplifique la representación del estado y nos permita trabajar con un número óptimo de estados. Por suerte para nosotros, disponer de un número tan limitado de acciones posibles disminuye la complejidad del problema.

\subsection{Discretizando el estado}
A la hora de discretizar una observación, una de las mejores formas de hacerlo es a través de una simplificación de la misma. Este acercamiento es una adaptación del \textit{Constraint Relaxation} que se aplica en muchos problemas de satisfacción de restricciones. De esta forma podemos comprender el problema desde un punto de partida mucho más sencillo, para después pulir la fórmula y tener en cuenta más variables. Además, podemos ir evaluando el rendimiento de nuestro agente a lo largo del camino, y así poder comprobar si las modificaciones que vamos añadiendo realmente mejoran su comportamiento o simplemente añaden ruido.

Centrémonos por un momento en el objetivo más básico del problema: Mantener el mástil en pie. Podemos dividir el problema en dos estados muy simples, uno en el que el mástil está cayendo hacia la derecha y otro en el cae hacia la izquierda, como podemos ver en la Figura \ref{fig:cartpole_01}. Esto coincide con nuestras dos acciones disponibles y, de forma algo ingenua, podemos considerar esto un punto de partida válido.

\figura{Bitmap/Capitulo2/cartpole_01}{width=0.7\textwidth}{fig:cartpole_01}%
       {Simplificación de estados}

Por supuesto, esta discretización es demasiado ingenua para resolver el problema de una forma eficiente; es necesario probar otras divisiones de estados y tener en cuenta las demás observaciones. Para ello creamos una función parametrizable, la cual decide en cuántos estados dividir cada observación. Así podemos decidir a qué observaciones damos más importancia, creando más o menos estados en función a éstas, como podemos apreciar en la Figura \ref{fig:cartpole_02}.

\figura{Bitmap/Capitulo2/cartpole_02}{width=0.7\textwidth}{fig:cartpole_02}%
       {División de las observaciones}

Este aumento en el número de estados permite que nuestro agente sea capaz de especializarse más en acciones concretas. Pero esta mejora viene con un precio a pagar: Cuantos más estados distintos tengamos, más tiempo tardará en visitarlos todos (en repetidas ocasiones) durante su etapa de entrenamiento. Esto quiere decir que nuestro agente tardará más en tener un buen rendimiento. Es importante entonces que evaluemos el problema al que nos enfrentamos para encontrar el equilibrio ideal y así poder ahorrar recursos de memoria y tiempo.

\subsection{Resultados}
\begin{quote}
    CartPole define como solucionado el problema al obtener una recompensa media de 195.0 durante 100 episodios consecutivos.
\end{quote}

Para realizar las mediciones, creamos una cola con una longitud máxima de 100 elementos. En ella añadimos el resultado de cada simulación, de forma que el resultado número 101 elimina el número 1. De esta forma podemos calcular la media sobre la estructura de datos y comprobar si el problema se consideraría resuelto por OpenAI Gym. Además dibujaremos esta media de la estructura tras cada episodio en una gráfica, con el fin de que los resultados sean más fáciles de apreciar y analizar. Las gráficas resultantes tienen variaciones tras cada ejecución, pero el caso más común es el mostrado en la Figura \ref{fig:cartpole_03}.

\figura{Bitmap/Capitulo2/cartpole_03}{width=0.7\textwidth}{fig:cartpole_03}%
       {Visualización de los resultados del agente RL}

En dicha gráfica se pueden apreciar tres etapas distintas en el rendimiento del agente:
\begin{itemize}
    \item Episodios 1-50: durante las primeras simulaciones el agente aún ha de rellenar la tabla-Q. Se encuentra ante situaciones nuevas ante las que aún no sabe como reaccionar. Además hay que tener en cuenta que su rendimiento se calcula en base a la media con los primeros resultados, por lo que en la etapa final de esta fase aún es lastrado por esos primeros episodios.
    \item Episodios 50-300: suponen el despegue en rendimiento del agente. En ellos se empieza a librar del lastre de las primeras simulaciones, además de que su tabla interna ya se encuentra en un estado bastante estable y es capaz de generar mejores resultados.
    \item A partir del episodio 300: el agente alcanza su meta y el algoritmo comienza a estabilizarse.
\end{itemize}

Cabe destacar que estos resultados han sido obtenidos con una configuración en la discretización de \texttt{(1, 1, 3, 6)}. ¿Por qué optamos por esta configuración? Mientras probábamos distintas configuraciones descubrimos que el principal motivo de finalización de una simulación era la caída del mástil. En la mayoría de las ocasiones el mástil cae mucho antes de que el carro alcance los límites laterales. Esto nos permite ignorar las dos primeras observaciones (posición del carro y velocidad del carro) reduciendo notablemente la dimensionalidad del problema.

De esta forma, contamos con un espacio dimensional bastante reducido ($1*1*3*6=18$ estados). Si por el contrario hubiésemos optado por un número mayor de estados, la primera fase descrita arriba se alargaría considerablemente, pero a cambio podríamos esperar una mayor estabilidad del agente en la tercera fase.

\section{Los límites del Aprendizaje por Refuerzo}
A lo largo de este capítulo hemos introducido el \textit{Aprendizaje por Refuerzo} como método de aprendizaje, las ventajas que tiene sobre otros métodos, sus mecánicas y elementos, algunos de sus algoritmos... Pero también hemos dejado entrever ciertos problemas.

El más importante es el que vimos durante la discretización del estado. Este problema es conocido como la \textbf{maldición de la dimensión}(\textit{curse of dimensionality}) y hace referencia al aumento exponencial en el tamaño de las tablas de estados en la memoria de los agentes. El problema puede ser contenido a través de buenas aproximaciones en la discretización; no obstante, un entorno continuo lo bastante complejo acaba resultando inviable de afrontar.

Otra solución ha surgido durante los últimos años y consiste en combinar el \textit{Aprendizaje por Refuerzo} con técnicas de \textit{Deep Learning}, que estudiaremos en los próximos capítulos.

\chapter{Redes Neuronales y Q-Learning}
\label{cap:deepLearning}

\chapterquote{En otras actividades más allá del puro pensamiento lógico, nuestras mentes funcionan mucho más rápido que cualquier ordenador jamás concebido.}{Daniel Crevier}

\section{Redes Neuronales: definición y elementos}
En vistas a solucionar problemas cada vez más complejos, como por ejemplo reconocimiento de patrones, surgió la necesidad de buscar diferentes modelos computacionales. Una rama a destacar sería la de \textbf{Aprendizaje Profundo} (o \textit{Deep Learning}), englobada dentro de Aprendizaje Automático, que decidió diseñar estos nuevos modelos basándose en la estructura del cerebro humano, desarrollando de esta forma las primeras neuronas artificiales. Éstas se crearon para conseguir la resolución de dichos problemas a partir del diseño de Redes Neuronales y procesos de auto-aprendizaje vistos en capítulos anteriores.

\subsection{Neuronas}
La neurona es el elemento básico de una red neuronal. Al igual que las existentes en el cerebro humano, estas neuronas son células de procesamiento; reciben varias entradas de datos y devuelven una única salida. Cada neurona contiene uno o varios pesos propios, en función del número de variables que reciba, que se relacionan con éstas de la siguiente forma.

$$Y = f(x\ w\ + b)$$

Echando un vistazo a la representación de la neurona en la imagen \ref{fig:esquema_neurona}, distinguiremos las $x$ como las variables de entrada y $w$ los pesos que les asigna la propia neurona. La variable $b$ es el \textit{bias}, o desplazamiento, que se suma a la multiplicación entre las variables y sus respectivos pesos y $f$ es la \textbf{función de activación}, que nos devolverá la salida de la neurona $Y$.

\figura{Bitmap/AprendizajeProfundo/Neuron}{width=1\textwidth}{fig:esquema_neurona}%
       {Esquema de una neurona artificial.}

Sin embargo, dentro de estas neuronas existen varios tipos. Las que nos interesan y trataremos en profundidad son las especializadas en no linealidad, manteniendo ésta característica en su función de activación. Este tipo de neuronas produce respuestas acotadas, terminando así con los problemas de fluctuación y la falta de adecuación a diversos tamaños de señales. A continuación, listaremos las más utilizadas y las que podremos observar en la gráfica \ref{fig:func_neurona}.

\begin{itemize}
    \item Sigmoide: $f(z) = \frac{1}{(1+e^{-z})}$: esta función logit marca una diferenciación clara entre valores pequeños y grandes, agrupándolos en 0 o 1 respectivamente. Entre estos dos extremos, las salidas de las neuronas adoptarán una forma de S.
    \item Tanh: $f(z)= tanh(z)$: este tipo de neurona tiene una salida muy parecida a las tipo sigmoide, manteniendo la forma de S, pero con la diferencia de que el marco esta entre -1 y 1. Suele ser usada con más asiduidad debido a que está centrada en 0.
    \item ReLu: $f(z) = max(0,z)$: la neurona de linealidad restringida o ReLu (Restricted Linear Unit) utiliza un distinto tipo de no linealidad. A pesar de su simplicidad, ha sido la función de activación utilizada para la resolución de los diferentes problemas debido a su mayor facilidad de convergencia y facilidad computacional \citep{Enyinna2018}.
\end{itemize}

\figura{Bitmap/AprendizajeProfundo/Func_Neuron}{width=1\textwidth}{fig:func_neurona}%
       {Comparación entre funciones Sigmoide, Tanh y ReLu.}

\subsection{Estructura y funcionamiento}
Ya que las funcionalidades de una neurona son limitadas, nos resultaría imposible resolver un problema de Aprendizaje Automático con ella. Para ello, de nuevo basándose en el modelo de un cerebro humano, las neuronas se encuentran agrupadas por capas, formando así una red neuronal. Cada capa consta de un número finito de neuronas del mismo tipo, recibiendo todas los mismos tipos y cantidades de datos devueltos por las neuronas de la capa anterior pero, como hemos mencionado antes, devolviendo resultados diferentes cada una.

Las Redes Neuronales están compuestas generalmente por varias capas, de forma que las salidas de la anterior forman las entradas de la siguiente, tal y como podemos observar en la imagen \ref{fig:red_neuronal}. Entre ellas, podemos distinguir la capa de entrada, que es la que recibe los datos a analizar, la capa de salida, que es la que devuelve los resultados en el formato que nos interesa en función del problema, y las capas intermedias u ocultas. Éstas son las encargadas de abstraer todos los datos recibidos por la capa anterior para que sea más fácil procesarlos y analizarlos. 

\figura{Bitmap/AprendizajeProfundo/Red_Neuronal}{width=1\textwidth}{fig:red_neuronal}%
       {Esquema de estructura de una red neuronal profunda \citep{NIPS2013_5207}}

Las redes con múltiples capas intermedias son conocidas como \textbf{Redes Neuronales Profundas}, o \textit{Deep Neural Networks} (DNNs). El ejemplo más conocido para el que podemos utilizar estas redes es el de reconocimiento de objetos en imágenes. Existen múltiples formas de utilizar las DNNs en este ámbito, así que utilizaremos el ejemplo que podemos ver en la imagen \ref{fig:dnn}. 

La capa de entrada comienza recibiendo un conjunto de bits de distintos colores. Las capas intermedias se encargan de distinguir el objeto, ya sea diferenciando los píxeles según sus colores y su proximidad, distinguiendo siluetas o bien reconociendo patrones que puedan distinguir al coche del resto de la fotografía. Con una sola capa intermedia nos sería imposible resolver este problema, ya que el proceso requiere varios pasos entre neuronas (por ejemplo, reconocer los faros, las ventanas, la matrícula, etc.). Finalmente, la capa de salida se encarga de marcar las regiones en las que el coche es visible, o más bien, en las que existe el objeto coche. En otros casos, las capas intermedias también pueden ser utilizadas para disminuir las dimensiones del problema, como reduciendo el número de píxeles de una imagen desde una magnitud de miles a una de decenas.

\figura{Bitmap/AprendizajeProfundo/DNN_ejemplo}{width=1\textwidth}{fig:dnn}%
       {Ejemplo de uso de una DNN.}    

\subsubsection{Descenso de gradiente y retropropagación}

Como ya hemos visto en este mismo capítulo, la neurona dispone de unos pesos para asignar a sus entradas y así producir una salida que nos resulte útil. Sin embargo, el proceso de asignar un valor óptimo a esos pesos no resulta para nada trivial, así que necesitamos un método que nos ayude a encontrar una configuración óptima de todas las neuronas de nuestra red.
 
El método del \textbf{descenso de gradiente} (o \textit{gradient descent}) \citep{Buduma:grad-dec} es un algoritmo de optimización que permite converger hacia el valor mínimo de una función de manera iterativa. En este caso, la función a minimizar es la de coste, que calcula la diferencia entre la salida estimada (\textit{y}) y la salida real (\textit{t}), consiguiendo mejores resultados. Para identificar el mínimo de la función, el método del descenso del gradiente calcula la derivada parcial respecto a cada parámetro (\textit{x}) en el punto de evaluación. La derivada indica el valor y la dirección en los que se encuentra el mínimo más próximo. No obstante, uno de los problemas que surgen en este punto es que éste puede ser tanto un mínimo local como global, siendo posible que nunca lleguemos a unos valores óptimos.

El resultado de la derivada se resta a cada uno de los parámetros, multiplicando finalmente por la \textbf{tasa de aprendizaje} ($\alpha$). Aunque hemos hablado en capítulos anteriores sobre ella, aplicada a este caso indica lo rápido que converge el algoritmo hasta un mínimo local, manteniendo un valor entre 0 y 1. Es importante ver que la tasa influye mucho a la hora de utilizar el descenso de gradiente y es necesario escoger un valor adecuado para evitar problemas. Finalmente, la fórmula de la diferencia de gradiente quedaría expresada como la siguiente:

$$\Delta w_{k} = \sum_{i} \alpha\ x_{k}^{(i)} (t^{(i)} - y^{(i)} )$$

El descenso de gradiente resulta útil a la hora de asignar pesos a una neurona, pero para aplicarlo a una red neuronal completa hay que dar un paso más allá. Es por ello que existe un algoritmo conocido como \textbf{retropropagación} (o \textit{backpropagation}) \citep{Buduma:backprop}. Al igual que en el descenso de gradiente, compara la salida real de la red neuronal con la salida deseada y minimiza su error, con la diferencia de que cambia los pesos de la capa anterior. Este proceso se lleva a cabo por toda la red hasta alcanzar la capa de entrada.

\subsubsection{Optimizadores}
Antes hemos hablado de la tasa de aprendizaje, pero no hemos llegado a ver qué supone un menor o mayor valor. Concretamente, la tasa influye sobre todo dependiendo del tamaño del problema: mientras que un ratio pequeño puede aproximarse mejor al error mínimo, en problemas de gran magnitud puede resultar demasiado lento. Con un ratio de aprendizaje demasiado grande, ocurre exactamente lo contrario: aprenderá más rápidamente pero puede resultar muy difícil converger en un mínimo error local. 
Para evitar que ocurran este tipo de problemas, se estudiaron múltiples algoritmos de optimización, de forma que dicho ratio se pudiera modificar dinámicamente durante el entrenamiento. A continuación explicaremos los optimizadores más utilizados \citep{NIPS2017_7003}:

\begin{itemize}
    \item \textbf{AdaGrad}: intenta adaptar la tasa de aprendizaje global a partir de la acumulación del historial de gradientes. Para ser específicos, mantiene un seguimiento de la tasa de cada parámetro, de forma que éste sea inversamente proporcional a la raíz de la suma de los cuadrados del historial de gradiente de todos los estados. 
    \item \textbf{RMSProp}: este algoritmo utiliza el de descenso de gradiente con momento que, en resumen, se trata de un descenso de gradiente donde los pasos para reducir el error son cada vez más cortos, de forma que se pueda converger en un error mínimo sin problemas. Combinando este algoritmo con el historial de gradientes de AdaGrad, obtenemos el optimizador RMSProp. 
    \item \textbf{Adam}: mientras que RMSProp utiliza la media de los gradientes que guarda Adagrad (primer momento), Adam utiliza la media de la varianza (segundo momento). Hay que tener en cuenta que ya que los momentos están inicializados a cero, es necesario un factor de corrección para obtener el valor real de ellos. Adam será el optimizador que utilizaremos, por ser más eficiente y poder adaptarse a la mayoría de problemas. 
\end{itemize}

\subsection{Problemas de clasificación}
Los problemas de clasificación consisten, a grandes rasgos, en asignar un dato recibido a una categoría de un conjunto ya establecido. El ejemplo más claro de esta categoría sería proveer de imágenes de perros y gatos a una red neuronal, y que sea ésta la que decida de qué animal se trata en cada caso. Modificando el comportamiento del ojo humano, la red recibiría dichas imágenes como un conjunto de píxeles y a partir ellos, crearía contornos y buscaría patrones para distinguir entre un animal u otro.

En nuestro ejemplo de aprendizaje supervisado sobre clasificación, hemos elegido otro conjunto de datos conocido: reconocimiento de cifras escritas a mano \citep{lecun-mnisthandwrittendigit-2010}. Este consiste en un conjunto de 4500 imágenes de diferentes cifras escritas a mano de distintas formas, como se ve en la imagen \ref{fig:cifras}, incluyendo la respuesta correcta para comprobar los resultados de nuestro clasificador.

\figura{Bitmap/AprendizajeProfundo/Ejemplo_Cifras}{width=1\textwidth}{fig:cifras}%
       {Imagen del conjunto de datos de cifras escritas a mano de MNIST \citep{Lecun98gradient-basedlearning}} 

Para este caso, hemos establecido una red neuronal de tres capas, la de entrada, la de salida y una capa intermedia u oculta para que la red realice los cálculos necesarios. Hemos utilizado un optimizador Adam, de forma que el descenso de gradiente se realice de forma más efectiva y los valores de los pesos en la red se reajusten más rápidamente. Además, puede regular automáticamente el valor de la tasa de aprendizaje, comenzando en un valor de 0.001.

Tanto la capa de entrada como la oculta, son del tipo \textit{ReLu}, visto anteriormente. La capa de entrada se encarga de procesar los 400 píxeles que representan la imagen de una cifra escrita a mano. A modo de salida, hemos utilizado un tipo especial de capa llamado \textit{softmax}. En ella, se da la condición de que la suma de todas las salidas sea igual a 1. De esta forma, las probabilidades que reflejan las distintas salidas se encuentran normalizadas y son excluyentes entre sí, dejando que haya una probabilidad que destaque por encima de las demás en su conjunto; ésa será nuestra predicción. 

Además, para comprobar que nuestra red neuronal entrena y aprende correctamente, hemos utilizado validación cruzada en K-iteraciones (\textbf{K-Fold cross-validation}). La técnica de validación cruzada consiste en dividir en varios grupos el conjunto de datos proporcionado y entrenar varias veces el modelo sobre ellos, alternando la función de dichos grupos entre entrenamiento y prueba. El algoritmo K-Fold nos permite barajar el conjunto de datos y dividirlo en k grupos, todos para entrenamiento a excepción de uno que se reserva para el prueba. De esta forma, podemos comprobar las distintas formas en las que ha aprendido, en función de los grupos formados durante la aplicación del K-Fold.

\subsection{Problemas de regresión}
Los problemas de regresión se distinguen de los de clasificación en que la respuesta de la red neuronal no pertenece a un grupo, sino que se intenta predecir un valor real. La forma de entender este tipo de problemas sería tomando como ejemplo una red neuronal que sea capaz de predecir el precio de una casa en función de su tamaño, número de habitaciones, localización, etc.

Para este ejemplo, hemos utilizado el conjunto de datos \textbf{Boston Housing}, incluido en uno de los paquetes de Keras \citep{BostonKeras}. Éste consta de 13 características distintas de un número determinado de viviendas (506 casos), a partir de las cuales se puede averiguar su precio. 

Esta vez los datos son muy diversos y los atributos pueden representar tanto unidades en escalas de miles, como de unos o ceros, para representar si una característica se da o no. Por ello, antes de crear la red neuronal, es necesario estandarizar los valores mediante la función \textit{StandardScaler} de \textit{sklearn}. De esta forma, todos los atributos estarán normalizados y la red podrá trabajar con ellos de forma más eficiente.

La red neuronal se compone de tres capas, igual que en el ejemplo visto de clasificación, utilizando el optimizador Adam. De nuevo, las capas de entrada y la oculta, serán de tipo \textbf{ReLU}, con un tamaño de 13 y 26 neuronas cada una. La capa de salida consta de una única neurona, de tipo \textbf{ReLU}, de forma que nos devolverá directamente el precio predicho por la red neuronal. 

Además, para asegurarnos de la fiabilidad de la red neuronal y comprobar que no se ha dado una situación de \textbf{sobreentrenamiento} (\textit{overfitting}), en el conjunto de datos disponemos dos partes separadas, una para entrenamiento y otra para la prueba de validación, en una proporción de 80\% frente a un 20\%, respectivamente. De esta forma, mientras entrenamos la red con la primera, se puede comprobar la diferencia de errores con la segunda, como podemos apreciar en la imagen \ref{fig:regresion}.

\figura{Bitmap/AprendizajeProfundo/regresion}{width=1\textwidth}{fig:regresion}%
       {Disminución del error sobre el entrenamiento respecto a la validación.} 

En los casos de regresión haremos uso del \textbf{error cuadrático medio} (ECM), el más utilizado de los múltiples métodos de cálculo de error existentes. En nuestro caso, $Ypred$ se refiere al valor calculado por la red neuronal mientras que $Yreal$ es el valor real del precio de las casas. El cálculo del ECM nos indicará la diferencia entre ambos, pudiendo así comprobar lo cerca que han estado las predicciones de la red neuronal de los valores reales.

$$ECM = 1/n \sum^n_{i=1}(Ypred_i - Yreal_i)^2$$

\section{Redes Neuronales y Q-learning}

Al hablar de Aprendizaje por Refuerzo anteriormente, nos centramos en el método de Q-Learning. Observamos que las soluciones planteadas eran válidas, pero el mayor problema que presentaba se debía al tamaño de las tabla-Q: ésta terminaba con tamaños intratables incluso con problemas de complejidad baja. 

Buscando la forma de arreglar este problema, se generó la posibilidad de realizar aproximaciones a la función-Q a través del uso de una red neuronal, de forma que no fuera necesario recorrer todos los pares estado-valor constantemente. Es así como surgieron las Redes-Q Profundas (\textit{Deep Q-Networks} o \textbf{DQNs}, como nos referiremos a ellas). 

Las transiciones creadas a partir de las DQNs se pueden representar como un vector de valores de la siguiente forma: 

$$Q(s, · ; T )$$ 

Donde $T$ son los parámetros de la red y $s$ es el estado actual. Esto significa que la red es capaz de calcular múltiples valores-Q de forma dinámica para todas las acciones posibles en un estado, sin abandonar la representación que suponía la tabla-Q. Además, permite generalizar estados no visitados pero sí muy parecidos a otros que ya ha visitado, ya que la red ya tendrá calculadas las decisiones para éstos también.

\chapter{DQNs en acción}
\label{cap:dqnEnAccion}

\chapterquote{Frase célebre dicha por alguien inteligente}{Autor}

\section{CartPole}

\subsection{Reemplazar la tabla-Q}

\subsection{Añadir memoria al agente}

\subsection{Romper la cohesión de la memoria}

\subsection{Estabilizar la red}

\section{MountainCar}

\subsection{Recompensas insuficientes}

\chapter{Conclusiones y trabajo futuro}
\label{cap:conclusiones}

\chapterquote{Somos lo que hacemos repetidamente. La excelencia, entonces, no es un acto, es un hábito}{Aristóteles}

\section{Conclusiones}

Este proyecto nos ha servido como un extenso estudio sobre los orígenes y la evolución de la que actualmente es una de las tecnologías más exploradas dentro de la inteligencia artificial: aprendizaje automático, concretamente la subcategoría de aprendizaje por refuerzo profundo. 

Cuando empezamos el proyecto, sólo teníamos una leve idea de la complejidad que implicaba este término o del alcance de las utilidades que se pueden conseguir a través de esta tecnología. Pero por suerte, teníamos claro nuestro objetivo y nuestra investigación fue dirigida a conseguirlo: conseguir que una IA pudiera jugar por sí sola a un videojuego.

Comenzamos estudiando las bases del aprendizaje automático, tomando como punto de partida la técnica de aprendizaje por refuerzo. Ésta encajaba perfectamente en nuestro proyecto, ya que tratándose de videojuegos, era fácil encontrar recompensas en ese entorno. Investigando sobre ello, descubrimos el método de Q-Learning, que parecía ser el que mejor se adecuaba a nuestro caso. Nos permitía tomar decisiones en una representación fiable de cualquier entorno, ya que no era necesario establecer un modelo del mismo, así que la IA podría jugar con normalidad independientemente del juego.

Tras el estudio previo, era el momento de hacer nuestras primeras pruebas y aprender cómo utilizar el entorno. Empezamos con un juego sencillo, el CartPole disponible en la librería de OpenAI Gym. El entorno era sencillo y las acciones que debía tomar el agente, reducidas. Sin embargo, el modelo utilizado en Q-Learning nos podía dar problemas en entornos mayores, a pesar de que el enfoque era bueno en un principio: el tamaño que requería la representación de los estados posibles crecía de forma casi exponencial cada vez que se introducían nuevas variables en el entorno, o aumentaba el rango de valores de las existentes. Por ello, necesitaríamos ir un paso más allá. 

No fue fácil acostumbrarse al uso del método de Q-Learning. El concepto resultaba curioso a la vez que simple, debido a su parecido con una máquina de estados. No obstante, fue durante la implementación cuando más comenzamos a notar la complejidad que implicaba su uso, incluso en un problema a primera vista sencillo, como resultó ser el de CartPole. El hecho de plantear la representación del problema supuso un reto, ya que tratábamos con valores continuos, y fueron necesarias varias configuraciones para la discretización de dichos valores, hasta ver con cuál trabajaba mejor el algoritmo.

Como primera toma de contacto con los elementos que conformarían nuestras futuras investigaciones, en este caso todo lo aplicado sobre el ejemplo del CartPole en Q-Learning, podemos decir que los resultados fueron satisfactorios, y no solo porque el algoritmo consiguiese predecir correctamente los movimientos en el juego. Nos sirvió para prepararnos para los errores que podríamos encontrarnos más adelante, además de comprobar el funcionamiento de este método y crear lo que sería una base teórica para saber cómo aplicarlo a otros entornos en un futuro.

Simultáneamente, comenzamos buscando otros caminos por los que desarrollar nuestro proyecto, optando finalmente por el uso de redes neuronales. Comprobamos con varios ejemplos que éstas eran compatibles con nuestro problema, llegando a la conclusión de que era un buen punto por el que avanzar. Explorando más allá de lo que ya conocíamos de ellas, encontramos la forma perfecta de combinar la idea y resultados del método de Q-Learning con la rapidez y comodidad que suponían las redes neuronales: las DQNs (o \textit{Deep Q-Networks}). 

A pesar de que los conceptos que implicaban la creación de las DQNs ya nos eran familiares, los resultados en un primer momento nos pillaron totalmente por sorpresa. El agente no aprendía correctamente, y aunque esperábamos peores resultados por ser nuestro primer intento, no nos habíamos planteado unos tan malos. Fue en este punto cuando la tarea de investigación tomó un papel vital para nosotros, obligándonos a todos a volcarnos en ella, hasta ver nuestros fallos y posibles soluciones a ellos. 

Finalmente, gracias al uso de una memoria para que nuestro agente también aprendiese de sus propios errores y a una segunda red neuronal para estabilizar sus resultados, conseguimos un correcto aprendizaje. A partir de estos elementos, entendimos que no sólo importaba el volumen de los datos de los que disponíamos, o el número de episodios que dejásemos al agente aprendiendo. Lo importante era lo relevantes y representativos que debían ser éstos datos para que sirviera de algo aprender de ellos. De la misma forma, comprendimos que no siempre que nuestro agente aprendía algo nuevo, tenía que ser bueno. La segunda red nos fue muy útil para "guiar" al agente por la mejor ruta de aprendizaje, consiguiendo buenos resultados en pocos episodios, en comparación con todas las pruebas anteriores.  

El último paso al que llegamos en nuestro proyecto fue a trasladar todo lo aprendido y lo conseguido en el CartPole a otro entorno para comprobar que el agente seguía aprendiendo correctamente. Esta vez se trataba de el juego de MountainCar, también disponible en la librería de OpenAI Gym. 

El principal problema al que nos enfrentamos aquí fue a la velocidad de aprendizaje. Tras observar cómo se desenvolvía nuestro agente en el nuevo entorno, nos dimos cuenta de que la recompensa no trasmitía una información útil para el aprendizaje del modelo; o por lo menos, no hasta que se alcanzaba el objetivo final, lo cual implicaba un entrenamiento muy largo. 

Fue por ello que decidimos centrarnos en cambiar la recompensa. Tras diferentes enfoques y pruebas para modificarla, conseguimos acelerar el aprendizaje de nuestro agente. Sorprendentemente, las soluciones más simples fueron las más efectivas y las que nos llevaron a concluir que habíamos conseguido desarrollar una IA capaz de jugar a videojuegos.

Tal vez en un primer momento la idea de hacer un trabajo de investigación, con una meta que cada vez pueda resultar más lejana, no suene muy atractiva. Sin embargo, para nosotros esto ha resultado mucho más provechoso que centrarnos en terminar un producto final. Este proyecto nos ha abierto una puerta hacia el aprendizaje por refuerzo profundo, las técnicas que lo componen y los avances que se siguen haciendo en este campo. 

La investigación y el tiempo invertidos nos han llevado a comprender el trabajo que supone construir una IA de este tipo, incluso teniendo una pequeña base inicial en la que sostenerse. De la misma forma, hemos podido experimentar de primera mano las facilidades que supone su uso, sobre el nuestro y sobre otros proyectos. Además de nuevos conocimientos, hemos adquirido nuevas ideas y una base sobre la que construirlas. 

El aprendizaje por refuerzo profundo es un campo que todavía está en crecimiento y del que nos queda mucho por aprender. Pero si algo podemos decir con seguridad es que su evolución todavía no ha acabado y que con el tiempo tendrá mucha más importancia de la que ya tiene a día de hoy. Y, con suerte para nosotros, estaremos preparados para seguir su avance.

\section{Cumplimiento de objetivos}

Al principio de este documento, planteábamos unos objetivos que explorar durante la realización de este proyecto, los cuales nos han servido para organizar el trabajo y su exposición en el presente documento:

\begin{enumerate}
        \item El primer objetivo era comprender qué es el aprendizaje por refuerzo y en qué se diferencia de otras ramas del aprendizaje automático. A lo largo del capítulo \ref{cap:reinforcementLearning} vimos una explicación de dicho concepto, así como las ideas en las que se basaba y sus diferencias respecto a otras ramas del aprendizaje automático y por qué la elegimos como punto de partida. Siguiendo con esta idea, explicamos todos los elementos que lo componen y la forma que tienen de relacionarse.
        \item Una vez visto cómo se relacionan todos los elementos del aprendizaje por refuerzo, decidimos poner dichos conceptos a prueba durante el capítulo \ref{cap:q-learning}, en el que explicamos la librería Gym de OpenAI. Elegimos el entorno de CartPole para llevar a cabo nuestros experimentos, logrando superar el objetivo planteado para dicho entorno, pudiendo experimentar los problemas que suponía la implementación de dicho algoritmo en referencia al aumento exponencial de la tabla-Q, así como la complejidad que implicaba mantenerla completamente actualizada, y conseguir el correcto aprendizaje de nuestro agente.
        \item Otro objetivo era el de adentrarnos en el campo del aprendizaje profundo y los fundamentos de las redes neuronales. Lo tratamos a lo largo del capítulo \ref{cap:deepLearning}, durante el cual explicamos los principios de las redes neuronales, su composición, estructura y funcionamiento. De este modo, jugamos con distintos problemas de clasificación y regresión que nos sirvieron para tener una primera toma de contacto que más adelante nos ayudaría para manejar problemas dentro del ámbito del aprendizaje por refuerzo profundo.
        \item Nuestro último objetivo era el de combinar las redes neuronales y el aprendizaje por refuerzo, de forma que consiguiéramos obtener los beneficios de ambos, así como encontrar soluciones a los inconvenientes que tenían por separado, como vimos durante el capítulo \ref{cap:dqnEnAccion}. Estudiamos paso a paso cómo combinar los dos ámbitos, primero reemplazando la tabla-Q del aprendizaje por refuerzo por una red neuronal y viendo que no nos otorgaba un aprendizaje suficiente en nuestro agente. Tras ello introdujimos el concepto de reproductor de experiencias y más tarde el uso de dos redes, gracias a los cuales conseguimos estabilizar el aprendizaje y una convergencia de los resultados.
        \item Cumplidos todos los objetivos y en vista a los resultados obtenidos, hemos conseguido ver las limitaciones que tiene este campo de estudio actualmente, el cual sigue en evolución y sobre el que queda mucho trabajo por hacer.
\end{enumerate}

\section{Trabajo futuro}

El proyecto iba dirigido a enseñar a una IA a jugar y, aunque ésta ha aprendido de forma exitosa, apenas ha sido probada en un par de entornos con pocas variables. El camino de nuestra investigación podía ser el correcto, pero sin más avances, o pruebas en otros entornos, no podemos estar seguros de ello. 

%TODO JCA referencias
%https://en.wikipedia.org/wiki/Transfer_learning
Dicho esto, en caso de poder continuar nuestro proyecto, nos centraríamos en continuar con más pruebas en distintos juegos. Tras comprobar que nuestra IA es adaptable, empezaríamos actualizándola para que resultase escalable a problemas con más acciones a elegir o un mayor volumen de variables, en caso de que no lo fuera ya. Un punto a considerar sería si el agente podría aplicarse directamente a problemas parecidos a los que ya conoce (\textit{Transfer Learning}), ya entrenado en ellos, y explorar cómo adaptarlo para mejorar sus resultados.

%TODO JCA referencias
%https://en.wikipedia.org/wiki/Multi-agent_system
A partir de estas modificaciones, tendríamos que considerar los posibles elementos nuevos que pudieran surgir, en función del problema o juego que tratásemos. El reto consistiría en adaptar el agente a elementos que se salen de lo visto anteriormente y que pueden provocar que la resolución o victoria no dependa sólo del agente. El caso más común en el que podemos pensar es en entornos multijugador, donde el agente tenga que enfrentarse a otros jugadores, o agentes (entornos multiagente - INSERTAR REF AQUÍ). 

Entre estos casos, también podríamos considerar los entornos con información incompleta: situaciones en las que no se pueda ver todo el mapa y en las que el agente tendría que actuar sin conocer el problema en su completitud. Los entornos probabilísticos, fáciles de ver en juegos de cartas, dados o cualquier \textit{RPG} con encuentros y daños no deterministas, también serían algo a tener en cuenta. Además de tener una estrategia ya aprendida, el agente debería aprender a adaptarla o a optar por otra totalmente distinta en función de resultados que no siempre son los que puede esperar. 

El trabajo más a largo plazo podría seguir muchos caminos. Por una parte se podría experimentar con otro tipo de algoritmos, buscando una mayor eficiencia. Prueba de ello es ``Baselines'', el repositorio de OpenAI que ya mencionamos \citep{baselines}. Otra opción sería dar un paso más allá y empezar a interpretar directamente imágenes como entrada, para así no depender de observaciones en forma de variable como hemos hecho hasta ahora.

DeepMind ya demostró esto hace algunos años. En \citet{mnih2013playing} demostraron que su IA era capaz de aprender a jugar a una serie de juegos de Atari a nivel profesional, todo esto únicamente a través de los píxeles de cada imagen. Para lograr esto habría que especializarse en redes neuronales convolucionales, lo cual sería todo un reto. Ya no sólo por ser más complejas sino porque, al necesitar mayores recursos a la hora de entrenarse, el sistema debe ser mucho más preciso para ser viable.

Pero todo esfuerzo trae su recompensa: una vez se dispone de una IA capaz de reconocer imágenes, las posibilidades se disparan. Al no estar atado a unas observaciones provistas a través de variables, deja de ser necesario depender de frameworks o tediosas implementaciones manuales para realizar pruebas. Se podría trabajar a un nivel mucho más ``real'', en el sentido de que tal vez ya no sería necesario describir un problema a la perfección, con todas sus leyes físicas, para trabajar en él. Tal vez simplemente se necesitaría proveer al agente de una serie de imágenes para que fuese capaz de aprender sobre ellas, entendiendo comportamientos y patrones y aprendiendo de forma mucho más humana. No obstante, esta idea todavía queda algo lejana para nosotros.

\chapter{Aportación de los participantes}


\section{Juan Ramón del Caño Vega}


\subsection{Antecedentes}

Antes de empezar el proyecto ya contaba con un conocimiento básico sobre el Aprendizaje por Refuerzo. En la asignatura que cursé de Inteligencia Artificial se le daba bastante importancia a este apartado, especialmente de forma práctica. En los laboratorios trabajamos con Q-Learning en un entorno Java. Se trataba de una simulación en la que teníamos que estabilizar una nave espacial con tres motores, no obstante tan sólo tuvimos que implementar las funciones de recompensa y discretización.

Respecto a Redes Neuronales, recuerdo que no entramos en profundidad. Se nos explicaron, pero ni llegamos a utilizarlas de forma práctica ni se consideraba materia de examen, por lo que quedaron bastante de lado. Lo mismo ocurrió con el Aprendizaje por Refuerzo Profundo, el cual se nos mencionó al final del curso junto con sus posibles usos en campos como el reconocimiento de imágenes.

También tenía experiencia en otras áreas del Aprendizaje Automático, tanto aprendizaje supervisado como no supervisado, con los que he trabajado en las librerías \texttt{scikit-learn}, \texttt{pandas} o \texttt{numpy} de Python. Esto, a pesar de no estar directamente relacionado con nuestro trabajo, me facilitó acostumbrarme a trabajar con \texttt{Keras}.


\subsection{Aportación}

Inicialmente me dediqué a la parte de Aprendizaje por Refuerzo. Puesto que ya tenía una buena base teórica pudimos empezar a hacer pruebas con bastante rapidez.

Empecé por programar el simulador para poder ejecutar CartPole. El objetivo era hacerlo de forma modular, principalmente por un motivo: encapsular toda la lógica del agente en una clase propia nos permitiría mantener el ``bucle de ejecución'' lo más limpio y simple posible, de esta forma se asemejaba mucho al pseudocódigo que veíamos en los libros, como puede verse en los fragmentos de código de la sección \ref{sec:cartpoleDQN}).

Una vez conseguido eso, sólo quedaba por implementar la lógica del agente. Quizá lo más complicado fue la función de discretización, explicada en \ref{sec:disc}. Una vez implementada la función parametrizada sólo fue cuestión de probar algunas configuraciones e hiperparámetros hasta dar con la solución que más nos gustase. Eso y corregir algún que otro bug, como que la ecuación de Bellman no sumase recompensas futuras si el agente se encontraba en un estado final, lo cual hacía que nuestro algoritmo divergiese.

A la hora de documentar este primer bloque recurrí al libro \textit{Artificial Intelligence: A Moddern Approach} \citep{Russell:2009:AIM:1671238}, el cuál también fue mi libro de referencia durante la asignatura de Inteligencia Artificial y proporciona explicaciones bastante concisas de distintos campos. Aproveché este libro para escribir la introducción del proyecto y la primera parte de Aprendizaje por Refuerzo. Para los apartados más técnicos cambié a \textit{Reinforcement Learning: An Introduction} \citep{Sutton:2018:RLI:3312046}, manual por excelencia del Aprendizaje por Refuerzo. No obstante fue Juan Luis Romero (quien también ayudó e hizo pruebas con CartPole) el encargado de rematar el capítulo con las secciones de Q-Learning y Markov Decision Process.

En este punto el resto del equipo ya había acabado de investigar Redes Neuronales (especialmente Lidia Concepción y Francisco Ponce), y nos preparábamos para empezar con el Aprendizaje por Refuerzo Profundo. Para ponerme al día con Redes Neuronales repetí uno de los ejemplos de clasificación que ellos ya habían hecho, MNIST \citep{MNISTKeras}, pero esta vez usando Jupyter Notebooks y el conjunto de datos propio de Keras.

Una vez hecho esto todos nos pusimos a volver a resolver CartPole utilizando las DQN que vimos en libros como \textit{Fundamentals of Deep Learning} \citep{Buduma:dnn} y los artículos de DeepMind. Mientras el resto del equipo salto directamente a las implementaciones vistas en \ref{sec:cartpoledqn3} y \ref{sec:DA}, yo empecé desde \ref{code:dqn}. Estas implementaciones, a pesar de que sabíamos que no funcionarían demasiado bien, nos permitieron comprender el proceso de mejora del agente mucho mejor, y por supuesto a documentarlo mejor, en lo que también participé.

Finalmente, también dediqué tiempo al problema de MountainCar. Especialmente a darnos cuenta de por qué es un problema tan especial y cuáles eran los motivos por los que presentaba nuevos retos. Finalmente fue Ricardo Arranz quien se enfrentó con el problema hasta el final.

\section{Francisco Ponce Belmonte}


\subsection{Antecedentes}

Al principio tenía conocimientos bastante ligeros sobre temas como Aprendizaje por Refuerzo y Redes Neuronales, todos ellos lo aprendí en la asignatura de Inteligencia Artificial. Sin embargo, considerando esta base insuficiente, decidí cursar la optativa de Aprendizaje Automático, con el fin de ganar más conocimientos y soltura en el uso y funcionamiento de las Redes Neuronales.

Por otro lado, también cursé Minería de Datos. Aunque la asignatura no estaba directamente relacionada con el objetivo de este proyecto, sí que utilizaba algunas herramientas y principios que me resultaron útiles para empezar con mis aportaciones. Entre ellos, cabría destacar las librerías \texttt{scikit-learn}, \texttt{pandas} o \texttt{numpy} de Python. 


\subsection{Aportación}

Mientras algunos de mis compañeros se encargaban de la base para Aprendizaje por Refuerzo, yo me centré en el desarrollo de la Red Neuronal. Aunque ya tenía unos conocimientos base, además de lo que iba aprendiendo paralelamente en Aprendizaje Automático, decidí empezar desde abajo para ir analizando paso a paso las posibilidades de esta rama.

Debido a ello, comencé con el desarrollo de una red neuronal de clasificación \ref{sec:classif_NN} usando las herramientas con las que ya estaba familiarizado, consiguiendo resultados rápidamente y sin ningún problema. Una vez más acostumbrado al funcionamiento de las Redes Neuronales, empecé a preparar lo que realmente necesitaría para nuestro proyecto, una red neuronal de regresión \ref{sec:regres_NN}. Para ello, dejé atrás las herramientas conocidas y empecé a utilizar \texttt{Keras}. Éste ya poseía unos cuantos ejemplos que podía aprovechar, aparte de automatizar muchos de los procesos necesarios para la resolución del problema.

Para entonces, el resto del grupo ya había concluido con su parte y pudimos empezar a unir nuestros aportes para empezar con el Aprendizaje por Refuerzo Profundo. Por mi parte, empecé directamente con las implementaciones vistas en \ref{sec:cartpoledqn3}, en busca de obtener resultados que analizar. Sin embargo, ante la tesitura de que éramos muchos trabajando individualmente sobre el mismo problema, y que algunos de mis compañeros estaban consiguiendo mejores resultados, decidí centrarme más en la memoria del proyecto.

Durante esta parte \ref{cap:deepLearning} comencé referenciando lo aprendido en las pruebas de los primeros meses\ref{sec:classif_NN} \ref{sec:regres_NN}, para luego basarme en el libro de \textit{Fundamentals of Deep Learning}, especialmente en sus segundo\citep{Buduma:backprop} y cuarto \citep{Buduma:dnn} capítulos. En ellos se tratan de manera profunda los fundamentos y mecánicas de Redes Neuronales y su aplicación y uso en el Aprendizaje por Refuerzo Profundo.

En última instancia, al igual que el resto de mis compañeros, revisé el trabajo completo tanto para búsqueda de errores como para una mayor comprensión de lo conseguido en el proyecto.


%%%%%%%%%%%%%%%%%%%%%%%%%%%%%%%%%%%%%%%%%%%%%%%%%%%%%%%%%%%%%%%%%%%%%%%%%%%
% Si el TFM se escribe en inglés, comentar las siguientes líneas 
% porque no es necesario incluir nuevamente las Conclusiones en inglés
\setcounter{chapter}{\thechapter -1} 
\begin{otherlanguage}{english}
\chapter{Conclusions and future work}
\label{cap:conclusions}

\chapterquote{We are what we do repeatedly. Excellence, then, is not an act, it is a habit}{Aristotle}

\section{Conclusions}

This project has served us as an extensive study about the origins and evolution of one of the most explored technologies within the Artificial Inteligence: Machine Learning, concretely the Deep Learning field.

When we started this project, we just had a slight idea about this term's complexity involvements or the utilities's scope that can be reached with this technology. But fortunately, our goal was clear and our research was aimed for a single goal: To make it possible for an AI to play a videogame on its own.

We began by studying the basics of Machine Learning, taking as our starting point the technique of Reinforcement Learning. This one fitted perfectly our project, since when it comes to videogames, it is easy to find rewards in that environment. Investigating about it, we discovered the Q-Learning method, which seemed to be the best suited to our case. It allowed us to make decisions at every moment, maintaining a reliable representation of any possible environment, in a way that the AI could play normally regardless of the game.

After the previous study, it was time to do our first tests and learn how to use the environment. We started with a simple game: CartPole, available in the OpenAI Gym library. The environment was simple and the actions to be taken by the agent were reduced. The results were good, but we needed to go one step further. The model used in Q-Learning could give us a lot of problems in larger environments, even though the idea was good at first.

Almost simultaneously, we started looking for other ways to develop our project, finally opting for Neural Networks. We verified with several examples that these were compatible with our problem, concluding that it was a good point to advance. Exploring beyond what we already knew about them, we found the perfect way to combine the idea and results of Q-Learning with the speed and comfort that Neural Networks provide: DQNs.

Testing DQNs was somehow more tedious to perform, as it was something totally new for us and required extra research time. The results were not good at the beginning, but after stabilizing them with a second network and adding a memory for them to learn from their own mistakes, we achieve the correct learning of our agent, giving us better results.

The last step we reached in our project was to transfer everything we learned and accomplished in CartPole to another environment to check that the agent was still learning correctly. This time it was the MountainCar game, also available in the OpenAI Gym library. After several adaptations and some new minor components, we got successful results, so we concluded that we had managed to develop an AI capable of playing.

What has been achieved in this project has shown us the amount of possibilities that exist when it comes to solving a problem of this type. The research and time invested, has led us to understand the work involved in building an AI, even having a small initial base on which to stand, and the amount of information we still have to learn about this topic.

\section{Achievement of objectives}

At the beginning of this document, we proposed some objectives to explore during the course of this project, which have helped us to organize the work and its presentation in this document:

\begin{enumerate}
        \item The first objective was to understand what Reinforcement Learning is and how it differs from other branches of Machine Learning. Throughout the chapter \ref{cap:reinforcementLearning} we saw an explanation of this concept, the ideas on which it was based and their differences from other branches of Machine Learning, aswell as why we chose it as a starting point. Continuing with this idea, we explain all the elements that compose it and the way in which they relate to each other.
        \item Having seen how all the elements of Reinforcement Learning are related, we decided to test these concepts during the chapter \ref{cap:q-learning}, in which we explained the OpenAI Gym library. We chose the CartPole environment to carry out our experiments, managing to overtake the objective set for this environment and experimenting the problems involved in the implementation of this algorithm, in reference to the exponential increase of the Q-table and the complexity involved in keeping it completely updated, and achieving the correct learning of our agent.
        \item Another objective was to get into the field of Deep Learning and the fundamentals of Neural Networks. We discussed it throughout the chapter \ref{cap:deepLearning}, during which we explained the principles of Neural Networks, their composition, structure and behaviour. This way, we played with different problems of classification and regression that served us to have a first contact that later would help us to manage problems within the scope of Deep Reinforcement Learning.
        \item Our last goal was to combine Neural Networks and Reinforcement Learning, in a way that we could get the benefits of both, as well as finding solutions to their different issues, as we saw during the chapter \ref{cap:dqnEnAccion}. We studied step by step how to combine the two scopes, first replacing the Q-table of Reinforcement Learning with a Neuronal Network, which we saw that it didn't give our agent enough learning skills. After that, we introduced the concept of experience replay and the use of two networks, thanks to which we managed to stabilize learning and convergence.
        \item Having fulfilled all the objectives, and in view of the obtained results, we have been able to see the limitations of this field of study as of today, which is quickly evolving with new techniques every day. There definetly are lots of ways to follow if you are eager and interested in it.
\end{enumerate}

\section{Future work}

The project was aimed for teaching an AI to play and, although it has learned successfully, it has hardly been tested in a couple of environments with few variables. The path of our research could be the right one, but without further progress, or testing in other environments, we cannot be sure of that.

That said, if we could continue our project, we would focus on continuing with more tests on different games. After verifying that our AI is adaptable, we would start by adding more options to the possible actions that the player can perform, updating it to make it scalable, in case it already wasn't.

From there, work could follow several paths. On one hand we could experiment with other types of algorithms, looking for a greater efficiency. Proof of this is ``Baselines'', the OpenAI repository that we have already mentioned \citep{baselines}. Another option would be to go a step further and start interpreting images directly as input, so as not to depend on observations in the form of a variable as we have done so far.

DeepMind already demonstrated this a few years ago. In \citet{mnih2013playing} they proved that their AI was able to learn to play some Atari games at a professional level, all of this only through the pixels of each image. To achieve this we would have to specialize in Neural Convolutional Networks, which would be quite a challenge. Not only because they are more complex, but also because, since they need more resources when it comes to training, the system must be much more precise to be viable.

But every effort pays off: Once you have an AI capable of recognizing images, the possibilities soar. By not being tied to observations provided through variables, it is no longer necessary to rely on frameworks or tedious manual implementations to perform tests. One could work at a much more ``real'' level, in the sense that perhaps it would no longer be necessary to describe a problem perfectly, with all its physical laws, in order to work on it. Perhaps it would simply be necessary to provide the agent with a series of images so that it would be able to learn about them, understanding behaviors and patterns and learning in a much more humane way. However, this idea is still a long way off for us.

\end{otherlanguage}
%%%%%%%%%%%%%%%%%%%%%%%%%%%%%%%%%%%%%%%%%%%%%%%%%%%%%%%%%%%%%%%%%%%%%%%%%%%


% Apéndices
\appendix
% 
\chapter{Título}
\label{Appendix:Key1}

En continuación describiremos cronológicamente el desarrollo de prototipos durante el ejercicio de MountainCar:

\section{Prototipo 0}
Este fue nuestro primer intento en el que simplemente probamos a entrenar directamente con la recompensa del entorno.

\figura{Bitmap/ApendiceA/Primer_paso.PNG}{width=1\textwidth}{fig:mountaincar_00}%
       {Prototipo 0}

En las primeras ejecuciones de 200 y 500 episodios no conseguimos ningún resultado \ref{fig:mountaincar_00} así que probamos con 1500 y obtuvimos nuestro primer paso, era posible resolver el problema.

\figura{Bitmap/ApendiceA/laFuerzaBrutaSiempreGana.jpg}{width=1\textwidth}{fig:mountaincar_001}%
{Prototipo 00}

En este momento iniciamos nuestra investigación y análisis del entorno. Identificamos como posible causa la recompensa del entorno, y decidimos cambiarla por otra modificada para favorecer el aprendizaje.

\section{Prototipo 1}
Desarrollamos nuestra primera función de recompensa, ``ourReward'', que recibe un estado y devuelve la recompensa en función de la posición absoluta añadiendo una bonificación de +1 en caso de que la posición sea mayor a 0.5 y la velocidad sea positiva. Inicialmente planteamos este prototipo con un factor de descuento \ref{fig:mountaincar_01} que para penalizar las soluciones lentas frente a las rápidas.

\figura{Bitmap/ApendiceA/CapturaCambiandoRewardAPosicion.PNG}{width=1\textwidth}{fig:mountaincar_01}%
{Prototipo 1}

Más tarde nos percatamos de que este factor de descuento entraba en conflicto con nuestro principio de basar la recompensa enteramente en el estado. De todas formas, pudimos comprobar un factor de penalización menor de 0.997 era excesivo. Y por otro lado caímos en la cuenta de que la posición absoluta no reflejaba el comportamiento que queríamos premiar. Nuestro objetivo era premiar al agente por alejar el avatar del centro, el punto más bajo entre las montañas, pero dado que este centro se encuentra en la posición -0.5, estábamos premiando la distancia respecto a la mitad de la montaña derecha. Lo cual claramente conlleva a evitar subir la ladera derecha.





\section{Prototipo 2}
Este prototipo fue desarrollado al mismo tiempo que el Prototipo 1 y más tarde descartado por las mismas razones.
En este caso basamos la recompensa en la velocidad absoluta, añadiendo una bonificación si la posición absoluta es mayor que 0.5. Como ya explicamos antes, el centro no es 0 sino -0.5 por lo que esa bonificación se aplicará siempre que no alcance la mitad de la ladera derecha. Además, lo insignificantemente pequeña que es la velocidad comparada con el valor de la bonificación, hace que reciba siempre la misma recompensa. Por otro lado, al igual que Prototipo 1, esta cuenta con un factor de descuento.

\figura{Bitmap/ApendiceA/FigureMain2.png}{width=1\textwidth}{fig:mountaincar_02}%
{Prototipo 2}

Observando más detenidamente los resultados y el comportamiento del agente son percatamos de que esta recompensa se ajustaba bastante al comportamiento de la recompensa original del problema, la razón de esto es que las recompensas intermedias son invariantes durante toda la ejecución.

\section{Prototipo 3}
Este prototipo fue creado junto a Prototipo 1 y Prototipo 2, pero fuimos actualizando lo a medida que íbamos detectando fallos en los dos anteriores y puntos que mejorar en los siguiente.
Comenzó como otro prototipo basado en la Velocidad absoluta, sin factor de penalización y con una bonificación de +10 en caso de superar su mayor posición centro, de esta forma pretendíamos que al quedarse sin impulso hacia un lado buscara superar dicha altura por la ladera contraria y así favorecer el balanceo \ref{fig:mountaincar_03}.

\figura{Bitmap/ApendiceA/Main_3SinPenalizacionAlcanzadoIndistin.png}{width=1\textwidth}{fig:mountaincar_030}%
{Prototipo 30}

En este punto nos dimos cuenta de que, aunque en ocasiones si llegaba a alcanzar la meta, estas ejecuciones pasaban inadvertidas para el aprendizaje dado que en estos casos la recompensa no era especialmente mayor que las demás partidas. Incluso en otros prototipos que desarrollábamos al mismo tiempo, llegamos a ver ejemplos que priorizaban seguir con el balanceo y así ganar más puntos.
Por ello decidimos dar una recompensa desproporcionada a las partidas que alcanzaran la meta \ref{fig:mountaincar_031}.

\figura{Bitmap/ApendiceA/Main3SinPenalAlcanzaDisting.png}{width=1\textwidth}{fig:mountaincar_031}%
{Prototipo 31}

Tras las observaciones del Prototipo 2, optamos por dar más visibilidad a la velocidad, esto lo haríamos en primera instancia multiplicando la por 10.

\figura{Bitmap/ApendiceA/main3_rewardX10.png}{width=1\textwidth}{fig:mountaincar_032}%
{Prototipo 32}

Tras los buenos resultados del Prototipo 4, probamos a aplicar el mismo método en el Prototipo 3, pero esta vez le daríamos muchos más episodios para aprender \ref{fig:moutaincar_033}

\figura{Bitmap/ApendiceA/1500itMain3RewardX100.png}{width=1\textwidth}{fig:mountaincar_033}%
{Prototipo 33}

\section{Prototipo 4}
Desarrollado poco después de comenzar con las pruebas del Prototipo 3 e implementando las correcciones de los errores de los prototipos 1 y 2.
La función de recompensa consiste en la velocidad absoluta por 100, para así darle más visibilidad a la velocidad frente a las bonificaciones. Por otro lado, y al igual que Prototipo 3, este premia con un +10 cada vez que el avatar supera su posición más alejada del centro. Además, este fue nuestro primer prototipo que no tenía el factor de descuento que invalidaba casi todas las pruebas anteriores.

\figura{Bitmap/ApendiceA/Main_4SolorecompSiGTmaxMas0_1.png}{width=1\textwidth}{fig:mountaincar_04}%
{Prototipo 4}

\section{Prototipo 5}
Viendo el poco éxito que habíamos tenido hasta el momento, decidimos regresar a los orígenes, para ver si con pequeñas modificaciones de la recompensa original conseguíamos alguna mejora notable.
La recompensa volvía a ser -1 para todo estado no final, pero si supera su posición más alejada del centro en 0.1, la recompensa pasa a ser 1. De esta forma planteamos una serie de recompensas intermedias como los prototipos anteriores, pero reduciendo el número de estas. Esta estrategia premia solo superar las metas y penaliza por el paso del tiempo.

\figura{Bitmap/ApendiceA/main5PremiarGTmaxMas0_1else-1.png}{width=1\textwidth}{fig:mountaincar_05}%
{Prototipo 5}

En este momento fue cuando caímos en la cuenta de que todos los bonus, que habíamos estado basando en la posición máxima alcanzada, eran erróneos, dado que la posición máxima alcanzada varía según el momento de la partida, es decir, la recompensa no se basaba únicamente en el estado y por tanto no era valida.

\section{Prototipo 6}
Ante las últimas revelaciones optamos por volver a modelos mucho más simples. Comenzando como una nueva variante del Prototipo 3, basado en la velocidad absoluta y dando una recompensa exageradamente alta para reforzar los caminos correctos. Además de empotrar el ``Doble Agente'' en este ejercicio.

\figura{Bitmap/ApendiceA/main3_1_prueba3ExageraRecompensa.png}{width=1\textwidth}{fig:mountaincar_06}%
{Prototipo 6}

Comenzamos a explorar las posibles causas de las bajadas en el aprendizaje que se ven en algunas de las gráficas.
Hasta el momento siempre habíamos utilizado el Optimizador Adam\ref{fig:mountaincar_061} pero hicimos algunas pruebas con Adadelta\ref{fig:mountaincar_062}.

\figura{Bitmap/ApendiceA/main_3_2FinalSinBonusAdam200eps.png}{width=1\textwidth}{fig:mountaincar_061}%
{RELLENAR}

\figura{Bitmap/ApendiceA/main_3_2FinalSinBonusAdadelta200Mejor.png}{width=1\textwidth}{fig:mountaincar_062}%
{RELLENAR}

\section{Prototipo 7}
Ahora pasamos a crear unas gráficas más completas para poder ver bien la evolución de las ejecuciones.
Retomamos la recompensa basada en la posición respecto al centro \ref{fig:mountaincar_070}. Y hacemos la correspondiente versión para la velocidad \ref{fig:mountaincar_071}.

\figura{Bitmap/ApendiceA/mainPosCentradaGraficaCompleta.png}{width=1\textwidth}{fig:mountaincar_070}%
{Prototipo Gráfica mejorada Posición Centro}

\figura{Bitmap/ApendiceA/mainVelocidadGraficaCompleta.png}{width=1\textwidth}{fig:mountaincar_071}%
{Prototipo Gráfica mejorada velocidad absoluta}

La primera grafica representa los datos referentes a la recompensa, tanto la recompensa del \textbf{entorno} en \textbf{rojo} y como la que nosotros le asignamos en \textbf{azul}.
Las gráficas \textbf{moradas} muestran respectivamente la \textbf{velocidad} máxima alcanzada y la velocidad acumulada durante toda cada episodio.
Las gráficas \textbf{verdes} muestran la información referente a la \textbf{posición} siendo la primera la posición máxima alcanzada, si es 0.5 implica terminar la partida y la posición acumulada, para así medir claramente el desplazamiento del avatar durante sus episodios.

\subsection{Prototipo 8}
Se puede ver claramente como el aprendizaje de los prototipos anteriores fluctúa bastante. Por ello creamos un nuevo agente más estable que añade una tercera red neuronal en la que guardara la mejor configuración encontrada hasta el momento de esa forma podemos prevenir la regresión en el aprendizaje.

\figura{Bitmap/ApendiceA/MAIN10AGENT6.png}{width=1\textwidth}{fig:mountaincar_08}%
{RELLENAR}

La gráfica inferior izquierda muestra los resultados de las competiciones entre las redes neuronales, las marcas verdes indican el resultado de las grandes competiciones, en las que se renueva la mejor red guardada.
Además, cambiamos la configuración de las gráficas para prestar información más interesante.
También añadimos una competición final al terminar la ejecución para comparar los resultados finales.

\figura{Bitmap/ApendiceA/mountainCar_08.png}{width=1\textwidth}{fig:mountaincar_081}%
{RELLENAR}

\figura{Bitmap/ApendiceA/mountainCar_09.png}{width=1\textwidth}{fig:mountaincar_082}%
{RELLENAR}

\figura{Bitmap/ApendiceA/mountainCar_10.png}{width=1\textwidth}{fig:mountaincar_083}%
{RELLENAR}

En estas gráficas se puede ver la evolución de los resultados de las tres redes neuronales, así como la posición máxima alcanzada en cada episodio y por ultimo los resultados de las tres redes en una prueba para evaluar su aprendizaje.
Las dos primeras corresponden al doble agente, y la siguiente muestra el resultado obtenido de Prototipo 8.




%\chapter{Título}
\label{Appendix:Key2}

%\include{Apendices/appendixC}
%\include{...}
%\include{...}
%\include{...}
\backmatter

%
% Bibliografía
%
% Si el TFM se escribe en inglés, editar TeXiS/TeXiS_bib para cambiar el
% estilo de las referencias
%---------------------------------------------------------------------
%
%                      configBibliografia.tex
%
%---------------------------------------------------------------------
%
% bibliografia.tex
% Copyright 2009 Marco Antonio Gomez-Martin, Pedro Pablo Gomez-Martin
%
% This file belongs to the TeXiS manual, a LaTeX template for writting
% Thesis and other documents. The complete last TeXiS package can
% be obtained from http://gaia.fdi.ucm.es/projects/texis/
%
% Although the TeXiS template itself is distributed under the 
% conditions of the LaTeX Project Public License
% (http://www.latex-project.org/lppl.txt), the manual content
% uses the CC-BY-SA license that stays that you are free:
%
%    - to share & to copy, distribute and transmit the work
%    - to remix and to adapt the work
%
% under the following conditions:
%
%    - Attribution: you must attribute the work in the manner
%      specified by the author or licensor (but not in any way that
%      suggests that they endorse you or your use of the work).
%    - Share Alike: if you alter, transform, or build upon this
%      work, you may distribute the resulting work only under the
%      same, similar or a compatible license.
%
% The complete license is available in
% http://creativecommons.org/licenses/by-sa/3.0/legalcode
%
%---------------------------------------------------------------------
%
% Fichero  que  configura  los  parámetros  de  la  generación  de  la
% bibliografía.  Existen dos  parámetros configurables:  los ficheros
% .bib que se utilizan y la frase célebre que aparece justo antes de la
% primera referencia.
%
%---------------------------------------------------------------------


%%%%%%%%%%%%%%%%%%%%%%%%%%%%%%%%%%%%%%%%%%%%%%%%%%%%%%%%%%%%%%%%%%%%%%
% Definición de los ficheros .bib utilizados:
% \setBibFiles{<lista ficheros sin extension, separados por comas>}
% Nota:
% Es IMPORTANTE que los ficheros estén en la misma línea que
% el comando \setBibFiles. Si se desea utilizar varias líneas,
% terminarlas con una apertura de comentario.
%%%%%%%%%%%%%%%%%%%%%%%%%%%%%%%%%%%%%%%%%%%%%%%%%%%%%%%%%%%%%%%%%%%%%%
\setBibFiles{%
nuestros,latex,otros%
}

%%%%%%%%%%%%%%%%%%%%%%%%%%%%%%%%%%%%%%%%%%%%%%%%%%%%%%%%%%%%%%%%%%%%%%
% Definición de la frase célebre para el capítulo de la
% bibliografía. Dentro normalmente se querrá hacer uso del entorno
% \begin{FraseCelebre}, que contendrá a su vez otros dos entornos,
% un \begin{Frase} y un \begin{Fuente}.
%
% Nota:
% Si no se quiere cita, se puede eliminar su definición (en la
% macro setCitaBibliografia{} ).
%%%%%%%%%%%%%%%%%%%%%%%%%%%%%%%%%%%%%%%%%%%%%%%%%%%%%%%%%%%%%%%%%%%%%%
% \setCitaBibliografia{
% \begin{FraseCelebre}
% \begin{Frase}
%   Y así, del mucho leer y del poco dormir, se le secó el celebro de
%   manera que vino a perder el juicio.
% \end{Frase}
% \begin{Fuente}
%   Miguel de Cervantes Saavedra
% \end{Fuente}
% \end{FraseCelebre}
% }

%%
%% Creamos la bibliografia
%%
\makeBib

% Variable local para emacs, para  que encuentre el fichero maestro de
% compilación y funcionen mejor algunas teclas rápidas de AucTeX

%%%
%%% Local Variables:
%%% mode: latex
%%% TeX-master: "../Tesis.tex"
%%% End:

%
% Índice de palabras
%

% Sólo  la   generamos  si  está   declarada  \generaindice.  Consulta
% TeXiS.sty para más información.

% En realidad, el soporte para la generación de índices de palabras
% en TeXiS no está documentada en el manual, porque no ha sido usada
% "en producción". Por tanto, el fichero que genera el índice
% *no* se incluye aquí (está comentado). Consulta la documentación
% en TeXiS_pream.tex para más información.
\ifx\generaindice\undefined
\else
%%---------------------------------------------------------------------
%
%                        TeXiS_indice.tex
%
%---------------------------------------------------------------------
%
% TeXiS_indice.tex
% Copyright 2009 Marco Antonio Gomez-Martin, Pedro Pablo Gomez-Martin
%
% This file belongs to TeXiS, a LaTeX template for writting
% Thesis and other documents. The complete last TeXiS package can
% be obtained from http://gaia.fdi.ucm.es/projects/texis/
%
% This work may be distributed and/or modified under the
% conditions of the LaTeX Project Public License, either version 1.3
% of this license or (at your option) any later version.
% The latest version of this license is in
%   http://www.latex-project.org/lppl.txt
% and version 1.3 or later is part of all distributions of LaTeX
% version 2005/12/01 or later.
%
% This work has the LPPL maintenance status `maintained'.
% 
% The Current Maintainers of this work are Marco Antonio Gomez-Martin
% and Pedro Pablo Gomez-Martin
%
%---------------------------------------------------------------------
%
% Contiene  los  comandos  para  generar  el índice  de  palabras  del
% documento.
%
%---------------------------------------------------------------------
%
% NOTA IMPORTANTE: el  soporte en TeXiS para el  índice de palabras es
% embrionario, y  de hecho  ni siquiera se  describe en el  manual. Se
% proporciona  una infraestructura  básica (sin  terminar)  para ello,
% pero  no ha  sido usada  "en producción".  De hecho,  a pesar  de la
% existencia de  este fichero, *no* se incluye  en Tesis.tex. Consulta
% la documentación en TeXiS_pream.tex para más información.
%
%---------------------------------------------------------------------


% Si se  va a generar  la tabla de  contenidos (el índice  habitual) y
% también vamos a  generar el índice de palabras  (ambas decisiones se
% toman en  función de  la definición  o no de  un par  de constantes,
% puedes consultar modo.tex para más información), entonces metemos en
% la tabla de contenidos una  entrada para marcar la página donde está
% el índice de palabras.

\ifx\generatoc\undefined
\else
   \addcontentsline{toc}{chapter}{\indexname}
\fi


% Generamos el índice
\printindex

% Variable local para emacs, para  que encuentre el fichero maestro de
% compilación y funcionen mejor algunas teclas rápidas de AucTeX

%%%
%%% Local Variables:
%%% mode: latex
%%% TeX-master: "./tesis.tex"
%%% End:

\fi

%
% Lista de acrónimos
%

% Sólo  lo  generamos  si  está declarada  \generaacronimos.  Consulta
% TeXiS.sty para más información.


\ifx\generaacronimos\undefined
\else
%---------------------------------------------------------------------
%
%                        TeXiS_acron.tex
%
%---------------------------------------------------------------------
%
% TeXiS_acron.tex
% Copyright 2009 Marco Antonio Gomez-Martin, Pedro Pablo Gomez-Martin
%
% This file belongs to TeXiS, a LaTeX template for writting
% Thesis and other documents. The complete last TeXiS package can
% be obtained from http://gaia.fdi.ucm.es/projects/texis/
%
% This work may be distributed and/or modified under the
% conditions of the LaTeX Project Public License, either version 1.3
% of this license or (at your option) any later version.
% The latest version of this license is in
%   http://www.latex-project.org/lppl.txt
% and version 1.3 or later is part of all distributions of LaTeX
% version 2005/12/01 or later.
%
% This work has the LPPL maintenance status `maintained'.
% 
% The Current Maintainers of this work are Marco Antonio Gomez-Martin
% and Pedro Pablo Gomez-Martin
%
%---------------------------------------------------------------------
%
% Contiene  los  comandos  para  generar  el listado de acrónimos
% documento.
%
%---------------------------------------------------------------------
%
% NOTA IMPORTANTE:  para que la  generación de acrónimos  funcione, al
% menos  debe  existir  un  acrónimo   en  el  documento.  Si  no,  la
% compilación  del   fichero  LaTeX  falla  con   un  error  "extraño"
% (indicando  que  quizá  falte  un \item).   Consulta  el  comentario
% referente al paquete glosstex en TeXiS_pream.tex.
%
%---------------------------------------------------------------------


% Redefinimos a español  el título de la lista  de acrónimos (Babel no
% lo hace por nosotros esta vez)

\def\listacronymname{Lista de acrónimos}

% Para el glosario:
% \def\glosarryname{Glosario}

% Si se  va a generar  la tabla de  contenidos (el índice  habitual) y
% también vamos a  generar la lista de acrónimos  (ambas decisiones se
% toman en  función de  la definición  o no de  un par  de constantes,
% puedes consultar config.tex  para más información), entonces metemos
% en la  tabla de contenidos una  entrada para marcar  la página donde
% está el índice de palabras.

\ifx\generatoc\undefined
\else
   \addcontentsline{toc}{chapter}{\listacronymname}
\fi


% Generamos la lista de acrónimos (en realidad el índice asociado a la
% lista "acr" de GlossTeX)

\printglosstex(acr)

% Variable local para emacs, para  que encuentre el fichero maestro de
% compilación y funcionen mejor algunas teclas rápidas de AucTeX

%%%
%%% Local Variables:
%%% mode: latex
%%% TeX-master: "../Tesis.tex"
%%% End:

\fi

%
% Final
%
%---------------------------------------------------------------------
%
%                      fin.tex
%
%---------------------------------------------------------------------
%
% fin.tex
% Copyright 2009 Marco Antonio Gomez-Martin, Pedro Pablo Gomez-Martin
%
% This file belongs to the TeXiS manual, a LaTeX template for writting
% Thesis and other documents. The complete last TeXiS package can
% be obtained from http://gaia.fdi.ucm.es/projects/texis/
%
% Although the TeXiS template itself is distributed under the 
% conditions of the LaTeX Project Public License
% (http://www.latex-project.org/lppl.txt), the manual content
% uses the CC-BY-SA license that stays that you are free:
%
%    - to share & to copy, distribute and transmit the work
%    - to remix and to adapt the work
%
% under the following conditions:
%
%    - Attribution: you must attribute the work in the manner
%      specified by the author or licensor (but not in any way that
%      suggests that they endorse you or your use of the work).
%    - Share Alike: if you alter, transform, or build upon this
%      work, you may distribute the resulting work only under the
%      same, similar or a compatible license.
%
% The complete license is available in
% http://creativecommons.org/licenses/by-sa/3.0/legalcode
%
%---------------------------------------------------------------------
%
% Contiene la última página
%
%---------------------------------------------------------------------


% Ponemos el marcador en el PDF
\ifpdf
   \pdfbookmark{Fin}{fin}
\fi

\thispagestyle{empty}\mbox{}

\vspace*{4cm}

\small

\hfill \emph{--¿Qué te parece desto, Sancho? -- Dijo Don Quijote --}

\hfill \emph{Bien podrán los encantadores quitarme la ventura,}

\hfill \emph{pero el esfuerzo y el ánimo, será imposible.}

\hfill 

\hfill \emph{Segunda parte del Ingenioso Caballero} 

\hfill \emph{Don Quijote de la Mancha}

\hfill \emph{Miguel de Cervantes}

\vfill%space*{4cm}

\hfill \emph{--Buena está -- dijo Sancho --; fírmela vuestra merced.}

\hfill \emph{--No es menester firmarla -- dijo Don Quijote--,}

\hfill \emph{sino solamente poner mi rúbrica.}

\hfill 

\hfill \emph{Primera parte del Ingenioso Caballero} 

\hfill \emph{Don Quijote de la Mancha}

\hfill \emph{Miguel de Cervantes}


\newpage
\thispagestyle{empty}\mbox{}

\newpage

% Variable local para emacs, para  que encuentre el fichero maestro de
% compilación y funcionen mejor algunas teclas rápidas de AucTeX

%%%
%%% Local Variables:
%%% mode: latex
%%% TeX-master: "../Tesis.tex"
%%% End:

%\end{otherlanguage}
\end{document}
